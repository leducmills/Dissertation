
INTRODUCTION

what is this thesis about?

what are devices for embodied fabrication?

why should we care? what's interesting and novel?

what problems does this work aim to solve?

what are the motivations for this work?

-----


PROTOTYPES

intro to the devices, re-hash of motivation

design principles, goals of the devices


UCUBE

overview

technical
single arduino mega

typical use case scenario
- basic convex hull construction and export

discussion of the device

- positives - it works
- limitations, design flaws, leading up to..

SNAPCAD

overview
- motivations from ucube feedback
- bigger, more expressive, more sturdy
- multi color, thus multi player, multi shape -- although these last two are
more about software capabilities

technical
- pcb design - chainable input shift registers
- magnetic snap, towers, shift register boards, inverting hex buffer circuit
- 3d printed enclosures and brackets
- software reads only a string of 1's and 0's, not a complete coordinate

use case
- convex hull shape matching game

discussion

- 343 > 64
- size, weight, price, portability, non-diy


POPCAD

overview

- why paper? (price, availability, growing synergy with electronics)
- portability, possible to reproduce
- no loose pieces

technical

what makes this possible now? touch briefly on jie qi, etc.

v1 & v2

-capacitive sensing without additional IC

v1

- copper tape, wires, paper struts, construction paper

v2

- conductive tape all the way, laser cut circuit on paper, MC soldered directly
onto paper
- focus on being more 'paper-like'
-heavy weight watercolor paper
- no more struts -- pull-tab system instead

use case

- bring to a school? (w/ makerbot a mobile digital fabrication studio?) provide
building plans / make open source?

discussion

early example of 'next gen' paper electronics -- goes beyond aesthetic and
simple interactions to a fully functional peripheral interface

3x3x3 is small, true, but their may be educational advantages to starting
simply. (if study data bears this out)


-----

RELATED WORK

this work draws on previous work in several areas:

theories of children's education, from froebel and montessori to piaget, who
influenced papert, who really brought computation in a tangible sense into the
world of children's education

works discussing embodied cognition, embodied mathematics, why the body matters
in interface design, the relation of gesture to children's learning

--look for studies on whether play with tangible objects improve spatial
reasoning ability in children

history of tangibles in computer science -- lego mindstorms, programmable
bricks, hiroshi ishii's works, roblocks, topobo

recent tangible interfaces are paying attention to digital fabrication
technicques -- easigami, kidcad, constructable, interactive fabrication devices
from  CMU


------

EVALUATION


UCUBE PILOT

circumstances (who, what, where, when, why)

procedure

results

discussion 



UCUBE MODELING & MATCHING

circumstances (who, what, where, when, why)

procedure

results

discussion 



POPCAD / SNAPCAD A/B TEST

circumstances (who, what, where, when, why)

procedure

results

discussion 


-----

DISCUSSION 

- what has been accomplished? 
- what are the strengths and weaknesses of the work?
- how does this work fit in with the other themes and trends -- maker movement,
the personalization of fabrication devices, the push for education reforms in
the USA around STEM, e-textiles and other computational crafting activities
- fablabs in schools
- popcad as future paper electronics
- changing the way children interact with digital fabrication technology - our
devices as a stepping stone

-----

VISION
- the future of devices for embodied fabrication -- what is it?
- this is one small slice of the devices that could be created with similar
 aims
- only one paradigm (towers and nodes) was truly explored - there are so many
other possibilities, just in embodied 3d modeling devices, not too mention other
outputs - embodied devices for laser cutting, CNC machines, sewing machines,
looms, other traditional shop tools, and the coming desktop-ification of
industrial machinery


-----

CONCLUSIONS 

sum it all up!







