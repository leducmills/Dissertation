\chapter{Introduction}
\label{intro}

A number of computer scientists, technologists, and educators have declared that
the era of personal fabrication is upon
us\cite{anderson2012makers}\cite{Gershenfeld:2007:FCR:1211574}. New devices
aimed at increasing the ability of the individual to physically manufacture
their own ideas are being released at breakneck speed. The cultural and
technological shifts caused by this change are taking many forms, yet few
technologies associated with the `maker' movement have received as much
attention as 3D printing - the ability (by various means) to digitally design
and then print out physical 3-dimensional objects. Media outlets from
Forbes\cite{forbes} to The Economist\cite{economist} have extolled the
disruptive and democratizing possibilities that 3D printing offers - at least as
it affects the traditional manufacturing supply chain. Less examined has been
how to introduce novices, specifically pre-teens and early adolescents, to 3D
printing - and perhaps more importantly - discussing what (and how) they might
learn by being exposed to it.
While the variety of desktop 3D printers continues to increase and the cost of
adding a `fab lab' of digitally-based manufacturing tools in the home or
classroom steadily declines, the types of interfaces by which children can
easily and intuitively design and explore the capabilities of 3D printers still
remains a barren landscape consisting primarily of software-only solutions. It
is this landscape that we are interested in seeding, following the best
practices in computational and cognitive science with particular attention to
children-centered design.

To this end, we present a class of tangible user interfaces (TUIs) designed to
scaffold a child's ability to design, explore, and play in three dimensions,
with a particular focus on enabling output for 3D printing.
Significant work has been done on two devices that allow users to specify points
on a physical, interactive, volumetric interface that simultaneously displays
active points in real-time on a computer. The software on the computer allows
for certain modeling operations on the set of input points (e.g., taking the
convex hull), as well as exporting shapes to stereolithography (.stl) format,
the preferred format for 3D printing. We propose that these designs form a new
class of tangible input devices, and present early work on a portable pop-up
book building upon the ideas expressed in the two earlier designs. The rest of
the paper proceeds as follows: discussion of related work, description of
completed work, the proposed work, a timeline for completion, and concluding
thoughts.
