% Paulo
% - Great work overall
% - Why did you display the statistics the way you did?
%      - Perhaps add error bars to the graphs
% - Add transcriptions from the interactions and describe them in detail
%      - Perhaps send link to video
%      - (Mike) Put it into an Appendix so you don't break the flow of the presentation of the dissertation
% - Tools make a big difference, imagine child programming without Logo or Scratch
%      - But that's not really the focus of the paper
%      - Add a focus on the design of tools
% - Don't stop working in this area
% 
% 
% Kylie
% - Dissertation does 2 things
%      - Adds design and shifts the landscape of tools
%      - Embraces idea of testing and thinking of malleable questions that the tools might pose
% - Most campuses have a statistical consulting office, they're very helpful
%      - (Jeff) Perhaps: http://www.colorado.edu/ibs/crs/statcons.html
% - Talk about second half, contribution to research literature
%      - Where do you see yourself publishing?
%      - Could be a learning sciences contribution
%      - (Ben) Not fond of CHI. IDC, TEI, ICLS
%      - (Mike) TOCHI
%      - (Paulo) IJCCI
% - The gender data is too close and the sample is too small
%      - Don't make a mountain out of mole hill
% - Like classifying the complexity of 3D objects
% - What do you see your work doing to extend the speech and gesturing landscape?
%      - (Ben) Not about extending the landscape, more about seeing if it existed. See if design can impact gesturing
%      - Being too modest
%      - 3d modeling, learning and growth over time, articulate the contribution of the core questions that the study has asked
%      - 2-3 papers of the design work and how the design allows you to ask interesting questions
% - Fluidity between the construction and the modeling / 2d presentation. What affordances does that have?
%      - (Ben) 2d/3d fluidity in the paper
%      - The distance between media can really cement learning
% 
% 
% C-Dawg
% - Second some of the statistics comments
% - Leave some of the smaller differences alone, i.e. gender variations
% - Another stats point, people started w, s, or p then switched, so there's a fairly standard analyzes you can try to assess differences in s vs. p and order effects. There's a graphical way to display the data that may suggest this.
% - Modeling question
%      - (Ben) Modeled in the first session
%      - Didn't show which they made with different tool, if you separated that, would there be an obvious difference in what they made?
%      - (Ben) With the simpler shapes, they could have made them with either devices. With the more abstract ones, it matters
% - What can you say of the cost of the modeling device?
%      - (Ben) Not a whole lot: construction paper, conductive tape, led's, micro controller, time, xacto knife
%      - This might be an aspect of the work you're pursuing, sounds like it's an affordable option, that's pretty cool.
%      - (Ben) Want to make the designs as sturdy as possible and make them an open source project
% - Did you record the sessions, can review and see if there was difference between the snap device
%      - Review number of vertices, etc
% - What would be involved in making the pop-device out of wood? Balsa wood
%      - (Ben) Doesn't necessarily have to have any paper engineering
% - Interesting work!
% 
% 
% Tom Yeh
% - Having kids work in groups was nice, did you observe any group dynamics with pop-cad? Any anecdotal observations of what kids do together?
%      - (Ben) Observed interesting dynamics, one controls device, one controls software, turn-taking
%      - One advantage is the tangible nature and allowing multiple people to work together, traditional 3d modeling software can't do this
% - Is there accessible 3d modeling software, can blind kids use this?
%      - (Ben) Talked to Amy and Shaun Kane, not the major thrust, but it's possible
%      - What would you say to people that would want to adapt your device? What advice would you give?
%      - (Ben) Need a way to differentiate which axis, which point in space they're dealing with. Audio feedback
%      - How would you replace the light?
%      - (Ben) Multiple vibrations of a singular column, with dampening material
% - I'm a teacher in a high school, how do I start?
%      - (Ben) Design a sturdier open-source version
%      - Could it come as a kit?
%      - (Ben) Yes, but the cost may be more than it's worth
% 
% 
% Mike Eisenberg
% - Thesis notes in particular
% - Wanted to see more of a description of the free-hand modeling, that's a success, they were making stuff and playing with it
%      - Add what they said, what they were doing with it
%      - 2d writing device, kids were spontaneously were doing that, hadn't thought of that in those terms
% - Mentioned some initial credit to Julie Debiosie, 3d geo board is a stodgy instrument
%      - Not much of a mention of the promise for teaching
%      - Talk about it as a stodgy teaching device for x, y, z coordinates, maybe less free-wheeling but useful things people could do
% - You don't want to be the uCube guy or the pop-cad guy, you want to be the embodied interfaces guy
%      - What are these instances of?
%      - What can we learn about embodied interface design from this?
%      - What do we know now that we didn't know before?
%      - What would be inspired by this 5-10 years down the road?
%      - (Ben) Embodied fabrication device
% 
% Audience Questions
% - (Jeff) Why not switches for blind students?
% - (Jeff) What happens as the n-dimensions increase?
% - (Tom) Point and line becomes important in a high dimensional space. Kids are good at building in the minecraft world
% - (Ben) Could be interesting to have a minecraft mode



\documentclass[defaultstyle,11pt]{thesis}

\usepackage{amssymb}		% to get all AMS symbols
\usepackage{graphicx}		% to insert figures

\usepackage{rotating}
\usepackage{graphics}
\usepackage{colortbl}
\usepackage{color}
%\usepackage{hyperref}
\usepackage{subfig}
\usepackage{amsmath}

%%%%%%%%%%%%   All the preamble material:   %%%%%%%%%%%%

%\title{Devices for Embodied Fabrication}
\title{Embodied Fabrication:\\ Body-Centric Devices for Novice Designers}

\author{Benjamin A.}{Leduc-Mills}

\otherdegrees{B.A., University of California, Santa Cruz, 2003 \\
	      M.P.S., New York University, 2008 \\
	      M.S., University of Colorado, 2013}

\degree{Doctor of Philosophy}		%  #1 {long descr.}
	{Ph.D., Computer Science}		%  #2 {short descr.}

\dept{Department of}			%  #1 {designation}
	{Computer Science}		%  #2 {name}

\advisor{Prof.}				%  #1 {title}
	{Michael Eisenberg}			%  #2 {name}

\reader{Clayton Lewis}		%  2nd person to sign thesis
\readerThree{Tom Yeh}		%  3rd person to sign thesis

 % because it is very short
\abstract{  \OnePageChapter We present a class of devices under the umbrella
moniker of ``embodied fabrication''. These devices and the development of the
term embodied fabrication is rooted not only in computer science, but in
cognitive science, childhood educational theory, emerging digital fabrication
technology, and the convergence of these strands present in the do-it-yourself
community known as the ``maker movement''. As such, we operate under a certain
set of premises that guide and direct this work.
First, that embodied cognition - which places the body at the center of our
cognitive operations - provides a framework from which to ground our decisions
to design physical peripheral devices as opposed to purely screen-based
software. Second, that a strong line of pedagogical research supports providing
children with tangible, ``manipulative'' objects to learn with. Third, that
digital fabrication technologies - 3D printing in particular - provide a
wonderful new opportunity for children and novice designers in general to make,
play, and explore creatively - and that the current design options for 3D
printers are not suited to meaningful design and creation of objects by
non-expert 3D modelers. Finally, that by following the best traditions of
body-centric interaction design for children, devices can be created to provide
an educationally and technically rich environment that connects kids to the
creative potential of 3D printing. We unpack these ideas more in the
introduction, followed by an overview of three prototype devices belonging to
this class of ``embodied'' interfaces, a chapter on related work followed by a
chapter on the three user studies we performed with our devices, a discussion of
the presented studies, and finally we present a vision of the future of this
work and of embodied fabrication devices as a whole before concluding.
}

\dedication[Dedication]{	% NEVER use \OnePageChapter here.
	This work is dedicated to Red Burns, for the opportunity, the honesty, and the
	paper-smacking. The world is a duller place without you. Thank you.}

\acknowledgements{ \OnePageChapter % *MUST* BE ONLY ONE PAGE! 

I was incredibly fortunate to have found the supporting cast that I did during
my studies. My advisor, Michael Eisenberg is one of the true gems of humanity.
An unbelievable scholar, gifted writer, and a wonderfully supportive and warm
person. Our conversations were always inspiring; may there be many more to come.
My father, for putting Mike on my radar as I was applying to grad school and for
always letting me know how proud you were. It meant a lot. You and I turned out
more alike than I ever would have imagined. My mom, for her unwavering faith
that I would be excellent. You were much better at convincing me than I was at
convincing myself. Your visits were always a chance to recharge my batteries,
feel loved, and be fed extremely well. Monika, for taking the risk of leaving
your world behind to join me in Colorado, and for supporting me so well during a
stressful time. You're the best cheerleader a guy could ask for. I'm a lucky
man. Many others supported this work, academically and socially. Credit goes to
Julie DiBiase, Yingdan Huang, and Kate Starbird for their early work on a
3DGeoboard. I owe an enormous debt of gratitude to Nathan Seidle, whose
generosity allowed this work to reach its completion. My committee members were
all amazingly insightful and supportive. Ann Eisenberg kept the lab from
self-destruction more times than I can count. Present and past CU students for
their support (especially Jeff and Swamy in CS and all the MFA kids). Lindsay
Levkoff Diamond, for being an incredibly understanding and flexible boss during
my studies. The DoE and all my friends at SparkFun Electronics - without you, I
don't know if I would have made it. Finally, the staff and students at Gold
Crown for their amazing support during my user study. Thank you all.}

\IRBprotocol{13-0476}	%
% optional!


%\ToCisShort	% use this only for 1-page Table of Contents

%\LoFisShort	% use this only for 1-page Table of Figures
% \emptyLoF	% use this if there is no List of Figures

%\LoTisShort	% use this only for 1-page Table of Tables
% \emptyLoT	% use this if there is no List of Tables

%%%%%%%%%%%%%%%%%%%%%%%%%%%%%%%%%%%%%%%%%%%%%%%%%%%%%%%%%%%%%%%%%
%%%%%%%%%%%%%%%       BEGIN DOCUMENT...         %%%%%%%%%%%%%%%%%
%%%%%%%%%%%%%%%%%%%%%%%%%%%%%%%%%%%%%%%%%%%%%%%%%%%%%%%%%%%%%%%%%



\begin{document}

% \tableofcontents
% \listofigures
% \listoftables

\input{macros.tex}

\chapter{Introduction}
\label{introchap}


% what is embodied fabrication?
% what are the goals of this work? (democratization of 3d printing)

Ten years ago, 3-dimensional printing was solely the purview of large
fabrication studios and industrial manufacturing; five years ago the first
desktop ``homebrew'' 3D printers hit the market, though few people seemed to pay
much attention; today, desktop 3D printing is one of the big headlines at the
annual Consumer Electronics Showcase in Las Vegas, media outlets from
Forbes\cite{forbes} to the Economist\cite{economist} are discussing it, and most
of the teenagers we talked to during our user studies know what 3D printing is.
3D printing is part of a forceful trend often referred to as the ``maker
movement''\cite{anderson2012makers}, that puts do-it-yourself ethics and
emerging technology together in a way that has inspired people the world over to
put down the TV remote and pick up a soldering iron. A subset of this movement
has focused on ``digital fabrication'' technology, of which 3D printers, laser
cutters, CNC mills, and vinyl plotters (among other devices) belong. Digital
fabrication machines take computer-generated files as input and fabricate
physical objects from those files. With the help of the maker movement and
visionary works on the upcoming age of ``personal
fabrication''\cite{Gershenfeld:2007:FCR:1211574}, 3D printing machines that once
cost tens of thousands of dollars are now available as DIY kits for less than
one thousand. Do not misunderstand us - this is a wonderful thing; cost is one,
if not the main barrier to the spread of most technologies.
However, the maker movement is not without its blind spots. Most innovators
behind these desktop 3D printers are of a very privileged socioeconomic
background; many of them retired engineers or otherwise possessing technical
training far beyond the average person. For the most part, they have not (nor is
it necessarily their responsibility to have) truly thought about how to make
their low-cost 3D printers accessible to the average Joe and Sally - much less
Joe Jr. - and to be fair, they are not the only actors contributing to the
barrier of entry for 3D novices who wish to design for 3D printers.

For those readers who may be unfamiliar with 3D printing, it is indeed what it
sounds like - an umbrella term for one of several processes capable of creating
3-dimensional objects (usually fine layers of extruded plastic filament) from
digital files, much like a laser printer prints 2D images on paper. 3D printers
primarily take in a file format called stereolithography - or .STL for short.
Normally, to create an .STL file one needs a rather complicated,
professional-level piece of 3D-modeling software, such as Rhino\cite{Rhino} or
Solidworks\cite{Solidworks}; programs which are rich with features that only
seasoned users will find a need for, with sub-menus upon sub-menus and decidedly
particular behaviors that any novice - especially a young one - would find quite
intimidating. As anyone who has used these software programs knows, one must be
very precise and conscious of every operation for a model to turn out properly -
and this order of operations is learned slowly (often agonizingly) over time. It
is a user interface nightmare; hardly the soft of intuitive environment one
might want to learn with. It should be noted that some efforts have been made to
create entry-level 3D modeling software - most notably Google
SketchUp\cite{SketchUp} - although last we checked SketchUp did not export
directly into .STL format (there are some rather troublesome looking workarounds
however), leaving the average newcomer facing an incredibly steep learning curve
in order to produce any original, 3D-printable objects. We emphasize
``original'' because there are several fairly simple ways to download and print
out a pre-created .STL file from the Internet (most notably from the on-line
repository Thingiverse\cite{thingiverse}).

\begin{figure}[!ht]
\begin{center}$
\begin{array}{cc}
\includegraphics[width=.4\linewidth]{images/makerbot}&
\includegraphics[width=.5\linewidth]{images/MB05_REP_Group}
\end{array}$
\end{center}
\caption{Left: One of the first popular desktop 3D printers, the MakerBot
``Cupcake CNC'', released in 2009. Right: The latest group of MakerBot models,
released at the Consumer Electronics Showcase in January 2014.}
\label{makerbot}
\end{figure}

Although printing out dozens of army men or barnyard animal figurines may be
satisfying for a time, and indeed speaks to the compelling nature of 3D
printing, it seems fair to say that children do not learn much about 3D modeling
from a ``download and print'' paradigm. Herein lies the crux of the problem - 3D
printing offers a wonderfully rich new platform for design, creativity, and
exploration, but neither the 3D printer manufacturers nor the companies who
produce the software necessary to author files suitable for 3D printing have
made accessibility for novices a priority. This is where our journey begins:
the desire to democratize 3D printing in a way that empowers newcomers,
particularly youngsters, in designing their own objects for 3D printing;
engaging them in such a way that intuitively introduces many of the core
concepts of 3D modeling, while helping to solidify cognitive processes around
spatial reasoning and 2D/3D translations, by building a set of devices that act
as a new genus amongst an ecosystem of next-generation digital fabrication
interfaces.

How, then, did we arrive at the term ``embodied fabrication'' to describe this
new genus? The simple answer, at the risk of over-extending the genealogical
metaphor, is that we selected what we deemed to be the best, most relevant
traits from a number of related areas (computer science, cognitive science,
developmental psychology, pedagogical theory, and digital fabrication
technology, amongst others) and attempted to splice them together in such a way
as to meaningfully address the issues with 3D printing outlined above. 

We derive the term ``embodied'' from cognitive science, and the fairly recent
advances in an area known as ``embodied cognition''. Embodied cognition posits
that our physical bodies and their interactions with the world are more closely
bound to our cognitive processes than previously thought. Evidence from research
in this area (discussed more thoroughly in Chapter 3 on related work) points to
cognitive benefits in basic arithmetic, ratios, proportions, and spatial
reasoning - all of which are useful (if not essential) tools in 3D modeling,
simply by involving the body more closely in the learning process.
This evidence, combined with the simple intuition that learning the skill of
3-dimensional modeling ought to be done in 3-dimensions as much as possible and
not solely on a 2-dimensional screen, provided the impetus for us to look toward
a physical solution that involves the body more than a typical piece
of software.

Physical, or ``tangible'' user interfaces are nothing new; wooden blocks have
been a part of children's education in a pedagogical sense since the beginning
of kindergarten over 150 years ago\cite{froebel}. Montessori ``manipulatives''
developed in the early part of last century inspired some of the first attempts
at creating physical, computationally-enhanced construction kits for children in
the 1980's\cite{Resnick:1998:DMN:274644.274684}. Tangible user interfaces, or
TUIs have been a growing part of human-computer interaction in a formal way
since Hiroshi Ishii's work on ``tangible
bits''\cite{Ishii:1997:TBT:258549.258715} in the mid 1990's, and of course the
influence of icons such as Doug Engelbart\cite{engelbart1968research} - one
might argue the mouse was the first ``embodied'' peripheral for a computer, in
the 1960's - and Mark Weiser\cite{weiser1991computer} (who presaged many of the
devices we take for granted today) as well as many others, should not be
overlooked - we give a more detailed account of this lineage when discussing
related work. For us, the longevity, breadth of applications, and numerous
achievements of mediating human-computer interaction though different physical
interfaces further suggests that a tangible user interface, coupled with the
proper software is more than capable of providing an accessible and embodied
foundation for our work.

Taking design principles from the lexicon of tangible user interfaces, adapting
them to better fit an embodied cognition world-view, and focusing on enabling 3D
modeling specifically for 3D printers, we designed and built a suite of
functional prototype devices for an embodied mode of digital fabrication; hence
the title of our work. To this end, we present a class of tangible user
interfaces designed to scaffold a child's ability to design, explore, and play
in three dimensions, with a particular focus on enabling original output for 3D
printing. We present three prototype devices (called UCube, SnapCAD,
and PopCAD) as well as piece of companion software that translates the physical
actions performed on the devices into screen-based content in real-time.

To give a brief preview of our creations; with their hands, users manipulate a
device to specify points (as coordinates in 3-space) that simultaneously display
as active dots against a ghosted 3D grid in real-time on a computer. The
software on the computer allows for certain modeling operations on this set of
input points (e.g., taking the convex hull, making a path through space),
exporting shapes to stereolithography (.STL) format with the click of a button,
the preferred format for 3D printers, as well as other functionality that we
explore more thoroughly in the next chapter.

We propose that these designs form a novel class of embodied input devices aimed
at enabling novice output for digital fabrication machines. Over three separate
user studies with 11 to 18 year olds, we investigate the ability for children to use
our devices to model a given shape (with and without the companion software) and
to match configurations on our device to a printed 3-dimensional object (without
the aid of the software). In our last study we compare two of our devices over a
multi-session study, while also administering a set of spatial reasoning tasks
as a pre and post test. We video record the subjects (with parental consent)
and analyze the gesture and speech expressions the participants make when
explaining a modeling strategy to reproduce a given object.

Through our studies, we show evidence that our suite of devices can be used
effectively by young adolescents with very minimal instruction, that a wide
variety of shapes can be recreated by the majority of subjects who used our
devices, that spatial test scores and modeling performance tends to improve over
multiple sessions with our devices, and that the kinds of gestures produced
while explaining modeling strategy correlates to modeling success on our
devices, a finding which supports prior research on gesture analysis by other
authors.

By providing a feedback loop between the bodily interaction with
tangible interfaces and the observed changes in real-time on a computer screen,
this body of work presents strong new motives for the inclusion of embodied
cognition in tangible interface design, while tackling the lack of appropriate
tools for novices to create for 3D printers, and evaluating the efficacy of our
devices as modeling tools and as devices for strengthening spatial reasoning and
cognition. We continue in Chapter 2 to present our prototype devices and the
software they operate with, explaining the evolution of our design choices as
well as the technical details behind their operation. Chapter 3 details the lineage
of related work, hinted at somewhat in this introduction, drawing connections
between the childhood developmental theories and conceptions of space developed
by Piaget and refined by Papert, the notions of cognitive development and
embodied mathematics discussed by Lakoff and Nu\~nez, the democratization of
digital fabrication technologies discussed by Gershenfeld and Lipson, and the
previous adaptation of these achievements into computer science. Chapter 4 is
devoted to the evaluation of our work, presenting three user studies, their
procedures and results. Chapter 5 delves deeper into the discussions which
surround the observations from our studies, comparing them with prior research,
and against each other. Finally, Chapter 6 provides a vision for immediate
future work on our devices, a more expansive vision of the possibilities
inherent in embodied fabrication, and ends with our concluding thoughts.



% The work presented here draws on the stages of childhood developmental theories
% and conception of space developed by Piaget and refined by Papert, notions of
% cognitive development and embodied mathematics discussed by Lakoff and Nu\~nez,
% the democratization of digital fabrication technologies discussed by Gershenfeld
% and Lipson, and the previous adaptation of these achievements into computer
% science.
























%old intro
% Digital fabrication technologies are increasingly finding their way into
% educational spaces of all shapes and sizes. These new technologies 
% (3D printers, laser cutters, etc.) afford opportunities for exploring these new
% ways of `making' and how they may change the way we learn, explore, and play.
% Although there is much excitement surrounding the `maker movement' - and 3D
% printing in particular - there has been little examination of how to introduce
% a younger audience to 3D printing in an empowering way.
% This proposal argues that tangible interfaces - as opposed to 2D screen-based
% media - can be designed not only to support spatial reasoning and mathematical
% intuitions in children by engaging them in exploratory modeling and play, but
% that these interfaces can act as a democratizing force by enabling children to
% create physical objects with digital fabrication devices.
% The proposed work presents a series of novel tangible input devices for
% enhancing mathematical and spatial reasoning in kids with a focus on generating
% output for 3D printing. We discuss related work, the status of the proposed
% work, additional improvements to be made, a timeline for completion,
% and a discussion of risks, limitations, and outcomes inherent in the proposal.
% 
% 
% %from proposal
% A number of computer scientists, technologists, and educators have declared that
% the era of personal fabrication is upon
% us\cite{anderson2012makers}\cite{Gershenfeld:2007:FCR:1211574}. New devices
% aimed at increasing the ability of the individual to physically manufacture
% their own ideas are being released at breakneck speed. The cultural and
% technological shifts caused by this change are taking many forms, yet few
% technologies associated with the `maker' movement have received as much
% attention as 3D printing - the ability (by various means) to digitally design
% and then print out physical 3-dimensional objects. Media outlets from
% Forbes\cite{forbes} to The Economist\cite{economist} have extolled the
% disruptive and democratizing possibilities that 3D printing offers - at least as
% it affects the traditional manufacturing supply chain. Less examined has been
% how to introduce novices, specifically pre-teens and early adolescents, to 3D
% printing - and perhaps more importantly - discussing what (and how) they might
% learn by being exposed to it.
% While the variety of desktop 3D printers continues to increase and the cost of
% adding a `fab lab' of digitally-based manufacturing tools in the home or
% classroom steadily declines, the types of interfaces by which children can
% easily and intuitively design and explore the capabilities of 3D printers still
% remains a barren landscape consisting primarily of software-only solutions. It
% is this landscape that we are interested in seeding, following the best
% practices in computational and cognitive science with particular attention to
% children-centered design.




% \section{A Brief Overview of this Thesis}
% 
% \section{Motivations}

\chapter{Prototype Systems}
\label{prototypes}

Over the past several years we have been exploring various means of creating a
child-friendly tangible user interface that would serve as an input device for
exploring 3D modeling and digital fabrication. To this end, we have created
three prototypes: the UCube, an initial proof-of-concept device, SnapCAD, a more
expressive and study iteration of the UCube, and PopCAD a paper-based interface
addressing several of the concerns raised by SnapCAD. These systems all
communicate with versions of a companion software program running on  desktop
computer. This chapter describes these systems, the motivations behind their
design, and the technical work involved in their creation.

\section{UCube}

The UCube represents our first attempt to create a cooperative system of
hardware and software that encapsulated and combined our beliefs about embodied
cognition and the importance of accessible digital fabrication. The idea for the
UCube originally came from the attempt to create a ``3D Geoboard''.
\autoref{fig:geoboard} shows a rudimentary 2D geoboard consisting of a 3x3 grid
of nails stuck into a wooden block. Simple geometries, such as the triangle shown
in the referenced image, can be made by stretching rubber bands around some
number of ``pegs''. The geoboard invites a kind of tangible, exploratory, and
embodied play that (as we discuss in Chapter 3) promotes children's learning in
powerful ways. The goal, then, was to capture the ``gestalt'' of the traditional
2-dimensional geoboard and extend it - into 3-dimensions, and with a
computationally-enhanced interface that could translate physical modeling on a
device into a software program that could display the input from the geoboard in
a meaningful way.

\begin{figure}[ht]
\begin{center}$
\begin{array}{cc}
\includegraphics[width=.5\linewidth]{images/Geoboard}
\end{array}$
\end{center}
\caption{A simple 3x3 geoboard, with a rubber band stretched around several
pegs, forming a triangle.}
\label{fig:geoboard}
\end{figure}

The UCube (as seen on the left in \autoref{fig:cubev1}) was the initial result
of this goal. The physical interface consists of a set of vertical ``towers''
that are placed (and optionally re-placed) onto a board, acting somewhat like
the nails in the 2D geoboard. These towers are moved around a grid of 4x4 evenly
spaced nodes or sockets into which the towers are placed. The towers themselves
contain four switches placed vertically along the tower, creating a potential
for 64 (4x4x4) distinct points. Thus, when a tower is placed in a specific node
on the board and a switch is flipped on, a particular (x,y,z) coordinate in
three-dimensional space is activated and sent through a microcontroller to a
piece of software on the computer. An abstracted illustration of the hardware
system is seen on the right in \autoref{fig:cubev1}.

In turn, the UCube software takes the incoming coordinate data from the
microcontroller and translates it into a real-time visualization on screen. The
graphical user interface centers around a ``ghosted'' grid of all the potential
points, with the active points being highlighted. In the first version of the
software, the interface also provides a set of operations that can be performed
on the set of active points in addition to normal scene manipulations like zoom
and rotate. These functions include: taking the convex hull of the point set (as
imagined in \autoref{fig:cubev1}), creating a sequential path or knot through
the active points, exporting the convex hull or knot to .STL format for 3D printing,
drawing a (non-printable) spline through the active points, saving and loading a
shape, and editing the vertices of a convex hull via a click-and-drag interface
(a more complete review of the software occurs later on in this chapter).


\begin{figure}[ht]
\begin{center}$
\begin{array}{cc}
\includegraphics[width=.35\linewidth]{images/UCube-2}&
\includegraphics[width=.48\linewidth]{images/ucube_diagram}
\end{array}$
\end{center}
\caption{Left: The UCube device, with four towers and eight lit switches,
representing the eight vertices of a cube. Right: a schematic illustration of
the UCube hardware.}
\label{fig:cubev1}
\end{figure}

% We performed two separate studies using the first UCube interface with
% middle school children aged 11-14. The first (informal) study, detailed in
% \cite{Leduc-Mills:2011:UCD:1999030.1999039} had fourteen participants,
% consisting of five girls and nine boys, who were divided into six groups (five
% groups of two, one group of four). Participants were asked to model a sequence
% of five shapes of increasing complexity using the UCube along with the companion
% software. The target shapes were displayed on one half of a computer screen,
% while the UCube software showing the live model was displayed on the other half.
% The shapes were as follows:  a straight vertical line, a diagonal line, a cube,
% a triangular prism, and finally an irregular polyhedral object. No shape
% required more than four towers to complete, and shapes were always presented in
% the same order. Of the six groups who participated, four groups successfully
% modeled all five shapes, one group ran out of time after three shapes, and one
% group finished one shape. Sessions lasted between 17 and 30 minutes.


\begin{figure}[ht]
\begin{center}$
\begin{array}{cc}
\includegraphics[width=.45\linewidth]{images/ucube1_software} &
\includegraphics[width=.45\linewidth]{images/ucube1_user}
\end{array}$
\end{center}
\caption{Left: a screenshot of the UCube v1 software, showing the triangular
prism generated by performing the convex hull function on a set of 6 input
points. Right: A photograph of a middle-school student using the UCube. Here, the
student holds a tower in the platform and points simultaneously to the screen
representation of the selected point on the desktop computer.}
\label{fig:cubev2}
\end{figure}


% The second user study, from \cite{Leduc-Mills:2012:SSV:2307096.2307176},
% consisted of ten participants, eight boys and two girls, each of whom
% participated individually in two separate exercises. The first exercise was a
% modeling task, whereby the participant was handed a series of 3D-printed shapes
% and asked to recreate them on the UCube interface. The five physical models
% presented were: a cube, a tetrahedron, a diamond, a �house� (a cube with a
% pyramid on top), and a complex irregular polyhedron. The results were promising:
% overall, 21 of 50 shapes were completed from memory, 12 of 50 were completed
% while holding the shape, and a further 8 of 50 were completed with the aid of
% the UCube software, for a total of 41 out of 50 shapes modeled successfully
% (82\%). Of the nine missed shapes, seven were of the same shape, the complex
% polyhedron. The remaining two misses were from the same participant, who ran out
% of time before completion.
% 
% The second task was a matching task whereby participants were instructed to face
% away from the UCube while the facilitator modeled a set of lights on the UCube
% corresponding to one shape among a set of nine physical models laid out on the
% table next to the UCube. Once the lights on the UCube were set up, the
% participant was instructed to turn around, and indicate which physical object
% they thought the set of lights on the UCube corresponded to. Of the nine shapes,
% the participants were asked to match five of them (a cube, a triangular prism, a
% parallelogram, an elongated hexagon, and a trapezoid). Thus, only the cube was
% presented in both the matching and modeling exercises. Out of 50 matching tasks
% (5 per participant), 0 tasks resulted in the incorrect match being selected, and
% most matches were made in under 20 seconds, an encouraging result that points to
% the ability of children (of this age) to recognize convex hulls from a set of
% illuminated points.

% from IDC 2011 paper
As a first step in discussing the UCube's role in spatial design�and in
discussing the broader issue of children's three-dimensional design�this section
is devoted to a more thorough description of the UCube and its operation.
To begin with an overview, then: the UCube system is the combination of two
elements: the physical input device of ``towers'' placed on a board, and the
companion display software. These two systems work together to take the embodied
actions of the user and display corresponding points and shapes on the computer.
A sense of the scale of the device can be inferred from \autoref{fig:cubev2},
which shows a photograph of a middle-school student holding a newly-placed tower
in the UCube platform while pointing simultaneously at the desktop computer
screen beside it.
This photograph�which we will also return to in the discussion of pilot testing
in a later section�reflects the essential nature of interaction with the device:
points are designated in a spatial region provided by the platform, and then
represented in real time on the computer screen. Thus, the UCube promotes an
attention to the correspondence between the selected spatial points above the
platform and the (more abstract) representation on the computer screen.

\subsection{Hardware} 
The physical system for our first UCube prototype, as outlined earlier, consists
of a platform with a four-by-four grid of potential sites, each of which can
hold one tower with four switches, thus describing a 4x4x4 array of 64 potential
points.
The platform structure consists of three different horizontal ``layers''. The
top (or upper surface) layer has a four-by-four grid of circular holes, into
which the towers fit snugly. This layer of 1/4'' thick laser-cut clear acrylic
acts as a brace to hold the towers upright, and ensures that they are resistant
to being knocked over. The next layer down holds the headers, which allow the
towers to ``plug in'' and connect to the rest of the circuit. Wires from the
headers go down to the bottom layer, which holds the breadboarded circuit and
Arduino Mega microcontroller. The towers are made of transparent acrylic, the
side paneling of basswood. The towers were laser-cut in order to house the four
switches and corresponding circuitry elements. The switches are LED-backlit when
active, making it more apparent which points are active as well as giving a more
accessible ``gestalt'' of the shape being modeled. It also allows for some
potentially interesting applications in dimly-lit circumstances, such as
modeling constellations in a classroom or planetarium: in these situations, the
lights of the selected spatial points stand out especially vividly.

Each tower connects to the platform through a six-pin header (one pin each for
power, ground, and four switches). The switch connections are then routed
through a breadboard containing current limiting resistors for the LED switches
to pins on a microcontroller (an Arduino Mega\cite{ArduinoMega}).
The Arduino is then able to communicate (via asynchronous serial communication)
the active switches (and corresponding coordinates) to the computer through a
USB cable. \autoref{fig:cubev1}(right) depicts a schematic diagram of the UCube
hardware.

% \subsection{Software}
% 
% The UCube makes use of the Processing\cite{Processing} framework to read in
% the active coordinates from the Arduino microcontroller connected to the
% platform; the software then displays these as larger red points on a grid of
% grey dots. Users can rotate the grid along any axis by clicking and dragging
% with the mouse. In our current early prototype, there are only two buttons on
% the user interface: (i) an �export� button, responsible for taking the current
% set of active points and exporting them into "STL" file format (suitable for 3D
% printers), and (ii) a �mode� button which toggles between showing just the red
% dots as points and filling in an area (defined by a convex hull algorithm) to
% give a sense of shape.
% The software interface is intentionally minimal in order to encourage the user
% to focus on the physical interaction. We felt it was crucial not to fall into
% the trap of making another software tool for experts, so the main purpose of the
% software is to act as an aid�a means to cognitively clarify and confirm the
% user's intentions. Although it is likely that we will extend the software
% somewhat in future iterations, our goal is to support the physical experience of
% specifying a three-dimensional object, and not to add functionality beyond what
% is necessary or helpful to that end.

% \subsection{A Sample UCube Scenario}
% As a sample scenario, imagine that we wish to create a triangular prism solid
% employing the UCube. We can begin this process by selecting three points to form
% a triangle, as shown in Figure 6; then, by placing two more towers and creating
% the same triangular shape "shifted over" by two units (Figure 7) we create the
% entire prism. Naturally, there might be many alternative pathways to forming the
% same eventual shape: for example, we might begin by placing four (or more)
% towers in the platform, and then experiment or fiddle with the chosen lights to
% approach the eventual goal of creating our prism. Alternatively, we might begin
% without any towers in the device at all: by placing our hands or fingers above
% the device, roughly indicating where the prism should be, we might then use our
% imagined locations as "guides", helping us to place the necessary towers in the
% platform and select the correct lights for the vertices of the prism.
% In any event, having designed the prism using the UCube platform, and having
% checked that it looks like the correct shape on the computer screen, the final
% step is to export the shape into a format suitable for 3D printer output. The
% UCube software, as noted earlier, includes a feature for doing just this; and
% finally, we print out the prism, as shown in Figure 8.
% Figure 6. The first step in constructing our triangular prism: here, we create a
% planar triangular shape toward the left side of the platform, and can see the
% resulting shape on the computer screen shown at right.
% Figure 7. Completing the triangular prism. Here, we have added a second
% ("shifted") version of our original triangle to produce the six vertices needed
% to form the prism.

\subsection{Limitations}
It will probably not have escaped the reader's notice that the UCube, as a
three-dimensional modeling device, has significant limitations. To take the most
glaring of these: the user can only model those shapes whose vertices are among
the sixty-four locations accessible from the device. Moreover, those available
locations are evenly spaced in the form of a three-dimensional grid, or lattice;
thus, there are numerous simple-but-interesting shapes (such as the regular
dodecahedron, composed of regular pentagonal faces) that cannot be designed in
the current version of the UCube. Likewise, shapes with curved surfaces (such as
a cylinder), demanding at the very least a high resolution of accessible points,
could not be modeled in the current UCube. We will return to these issues in the
final section of the paper, in the discussion of ongoing and future work.



\section{SnapCAD}

\begin{figure}[ht] \begin{center}$
\begin{array}{cc}
\includegraphics[width=.45\linewidth]{images/BeatriceFinal-2} & 
\includegraphics[width=.45\linewidth]{images/BeatriceFinal-14}
\end{array}$
\end{center}
\caption{
Left: the SnapCAD interface, showing the hardware configuration corresponding to
the picture below in \autoref{fig:ucubev22}. Right: a detail of the SnapCAD
hardware - the PCB tower is housed in a 3D-printed shell, which plugs into a
shift-register board. The LED boards snap on to the towers via magnetic snaps.}
\label{fig:ucubev21}
\end{figure}

Based on the feedback from these two user studies, a second, more powerful
instantiation of the ideas from the UCube has been created. SnapCAD (formerly
known as UCube v2) consists of a total input space of 7x7x7 points, forming 343
distinct coordinates. In our user studies with UCube v1, we noticed that users
often encountered initial difficulties when required to `find a middle' in the
shape they were attempting to model, given an even number of total grid spaces.
For example, to model a pyramid on on a 4x4x4 grid, one needs to construct a 3x3
subset of the 4x4 grid, using the middle point within the 3x3 set as the top of
the pyramid. This influenced our decision to create an odd-numbered layout,
creating a more `natural' middle point in the hardware. The greater number of
inputs vastly increases the expressive potential of SnapCAD (compared to the
UCube) while still maintaining a manageable interface.
Working on the scale of multiple hundreds of inputs necessitated the design of
custom circuit boards to relay information effectively to the microcontroller.
This change in scale also meant rewriting most of the modeling software to
effectively handle the greater expressiveness of the physical system.



\begin{figure}[ht] \begin{center}$
\begin{array}{cc}
\includegraphics[width=.45\linewidth]{images/twoHulls} &
\includegraphics[width=.45\linewidth]{images/mst}
\end{array}$
\end{center}
\caption{Left: The SnapCAD software showing two convex hulls of different
colors. Right: the SnapCAD software showing a minimal spanning tree model.}
\label{fig:ucubev22}
\end{figure}


The use of conductive, magnetic snaps along towers constructed of custom-printed
circuit board allow for more than one color of illumination, as different
colored LED boards can be snapped onto any socket on the tower.
This not only results in the ability to represent multiple shapes at once, but
for the SnapCAD to become a platform for all manner of multi-player interactions
(e.g. games, puzzles, shape matching contests), with each `player' assigned a
unique color. To this end, we have created a simple `3D Tic-Tac-Toe'
implementation on the SnapCAD. Additional changes to the software include
supporting multiple but separate convex hulls of different colors, the ability
to create and export shapes created from the minimal spanning tree of a set of
input points, and the ability to adjust the width of the segments in the
knot/path and minimal spanning tree modes.
The click-and-drag editing mode now includes the knot/sequential path and
minimal spanning tree modes as well as the convex hull mode. We also adjusted
the knot-forming algorithm to handle paths that cross or self-intersect, as well
as providing a `close knot' button to complete a circuit in a shape, allowing
for even more kinds of 3D-printable objects. While significant work has been
done to bring the UCube and SnapCAD to their current states, we believe not only
that there is room for additional improvements to be made, but that, as opposed
to focusing on a incremental but essentially similar interface as the subject of
a thesis, it is far more intellectually interesting to focus on a class of
objects that demonstrate multiple incarnations of a set of ideas.

% \section{Proposed Work: Technical Additions}
% The proposed work is in two sections: technical additions and
% evaluation. This section deals with the technical additions to the proposed
% devices: SnapCAD and PopCAD.
% 
% \subsection{SnapCAD}
% 
% With 343 potential points, a click-and-drag editing tool, and three separate
% modeling modes (convex hull, knot/path, and minimal spanning tree) the SnapCAD
% is capable of generating countless 3-dimensional forms.
% Although the potential for additional modeling tools is certainly a possibility
% (we have yet to experiment with curved surfaces, for instance) we believe that
% the multi-player `platform' aspects of the SnapCAD system are the most ripe for
% development. We already have two colors of LED boards, the ability to display
% two colors of convex hulls, and a 3D implementation of a two-player tic-tac-toe
% game. Displaying multiple colors of the path/knot and minimal spanning tree
% modes should be fairly straightforward to implement.
% We would also like to expand and change the colors currently being used - we
% currently use red and green LEDs, which would be problematic for anyone with
% red/green color blindness. We propose using three colors: red, blue, and purple.
% This not only allows for up to three-player interactions, but could help to
% solve a deeper problem: representing in hardware a node occupied by two players.
% Using an idiom where solely-occupied nodes are either blue or red, and a jointly
% occupied node is purple, we can then expand the types of games, puzzles, or
% modeling activities the SnapCAD system can support. Once these improvements are
% made we can expand the activities supported on SnapCAD. Developments include
% two-to-three player games like tic-tac-toe as well as games built
% off of the modeling capabilities of the SnapCAD (e.g. match the model generated
% by the computer, model a sequential path through a generated maze, place points
% on or interior to the convex hull until there are no more to be found). Colors
% can also be used for certain as-yet unexplored modeling operations (e.g., the
% set union, intersection, or difference). While some of these operations may
% prove difficult or even impossible, these are all avenues worth exploring, as
% they all point towards the extensibility and potential expressiveness of SnapCAD
% as a platform for future development.

\section{PopCAD}

Our motivations for creating alternative interfaces to the UCube and SnapCAD
stem from the desire to explore this intellectual space more generally; it is
far more interesting to discuss a \emph{class} of tangible interfaces for
scaffolding digital fabrication than it is to discuss a singular device. To this
end, we looked at some of the weaknesses of SnapCAD and towards technologies we
had yet to explore. While SnapCAD can admirably perform a number of modeling
tasks, it was always envisioned as one device amongst an `ecosystem' of next
generation fabrication tools. It has strengths, but obvious weaknesses as well;
in particular, the SnapCAD hardware was expensive to produce, and so would be a
difficult proposition for some schools or fab labs; it is also rather unwieldy
and unportable - it moderately heavy, fairly large, and has many separate pieces
that could break or go missing. Thus, an interface with cheaper and more
portable materials was desirable.

To address these issues we chose to build a pop-up book combining traditional
paper-crafts and paper-friendly electronics such as copper tape.
In recent years, revolutionary work has been done in combining electronics and
paper
crafting\cite{Qi:2010:EPE:1709886.1709909}\cite{Mellis:2013:MMC:2460625.2460638},
leading to new techniques and new uses for traditional materials. Paper is
inexpensive (especially when compared to circuit boards), light, and easily
portable, making it an ideal material choice for a device that would not suffer
the same limitations present in the SnapCAD. Although we often think of `paper'
as a rather static material, there are in fact many variations in the size,
weight, color, transparency, and composition of contemporary paper products. For
the initial prototype, we used a simple construction paper as it provided a
balance between strength and flexibility as well as having a consistency
well-suited to laser etching and cutting.
The pop-up book (named PopCAD) has a 3x3x3 array of 27 points which are evenly
spaced 3 inches apart on a 12'' x 18'' paper surface.
The book folds on a single center crease making the closed footprint of the book
roughly 12'' x 9''.

\begin{figure}[ht] \begin{center}$
\begin{array}{cc}
\includegraphics[width=.45\linewidth]{images/popup1} &
\includegraphics[width=.45\linewidth]{images/popup2}
\end{array}$
\end{center}
\caption{Two views of the pop-up book prototype, showing the paper towers and 
LEDs in both open and closed states.}
\label{fig:popup}
\end{figure}

Each tower has a copper tape circuit consisting of three LEDs on the front face
and three corresponding capacitive touch sensors on the left face. The copper
tape acts as a paper-friendly conductive material to connect the electronic
components together much like traditional wire. The LEDs are soldered onto the
copper tape for greater stability. The capacitive sensors are simply a piece of
copper tape which is connected to a pin on a microcontroller (in the first
version, this is an Arduino Mega Pro). By bringing the internal pull-up resistor
connected to the pin `LOW' (to ground) and then timing how long it takes to get
back to a `HIGH' state we can tell if the connection is being influenced by a
capacitive force. For example, if there is no interference on the circuit, the
timer will normally only get to `1' before the resistor is back to a HIGH state;
if a finger is placed on the copper tape, the reading will be
much higher (typically around `17'). Based on this change, we can detect which
switch was touched and toggle the associated LED on or off. The hollow interior
of each paper tower is used to solder thin 30-gauge wire to the three LEDs, the
three switches, and ground. These seven wires are soldered to a row of headers
that stick through the bottom of the first layer of the pop-up book. Wires are
then run along the backside of the top layer of paper from these headers to the
microcontroller. The entire circuit in then encased in a cloth-covered cardboard
binder that acts as a book cover as well as a means to protect and hide the
electronics.

The software originally written for the UCube and SnapCAD was adapted to work
with the pop-up book, making it capable of similar types of algorithmic modeling
and stereolithography output for 3D printing. As the grid is 3x3x3, it also
makes sense to adapt some of the game-playing aspects of the larger devices
(e.g., it would still be possible to play 3D tic-tac-toe). In addition to adding
this functionality, there are several improvements and finishing touches to be
made on the book itself. Additionally, the current hardware setup for the pop-up
book does not allow for the LED's to be snapped on or off, making certain
multi-player or multi-shape operations impossible. Weather or not this
functionality is crucial to the pop-up book will determine if changes need to be
made.  

Given the different medium of the pop-up book (paper as opposed to circuit
boards), it is worth exploring the possibilities afforded by a cheaper, more
flexible material. For instance, the flexibility of paper might provide the
means for new types of modeling actions. It is plausible to imagine paper tabs
or other mechanisms that perturb the LEDs off the integer lattice, or alter the
overall topology in such a way that new shapes are possible (e.g. by deforming
an equidistant grid into a spherical shape). There may be additional sensors or
hardware that could be embedded into the book to provide new functionality
(rotation, proximity, pressure). Additionally, due the inexpensive and portable
nature of the pop-up book, it is worth exploring the sorts of interactions that
could occur between several pop-up books (e.g., extending the input field to
include two or more grids, networked interactions like cooperative modeling
tasks, or competitive games like 3D-battleship). By using paper as a material to
think with, we may find further possibilities as development continues.

\section{Software}
Put stuff about software development here. Details. Screenshots.
\chapter{Related Work}
\label{related}

% from proposal
The belief that tangible objects\footnote{It is worth noting the difference in
this work between `tangible objects' of the sort that a child might play with
(e.g. Lego) and `tangible user interfaces' (TUIs) that a child might interact
with - typically a peripheral device (apart from the keyboard and mouse) that
communicates physical interactions to a computer.} play an important role in
children's education is relatively recent. Friedrich Froebel's use of 20 wooden
forms he dubbed `gifts' in the first Kindergarten was in 1837\cite{froebel}. It
took until 1907 before an extension of Froebel's ideas and a focus on physical,
manipulative objects and tasks was implemented by Maria Montessori in the first
Casa Dei Bambini\cite{montessori}.
The interest in children's learning incorporating the use of manipulatives
progressed steadily, most notably by Jean Piaget and his work on `genetic
epistemology'. Piaget wrote extensively on the stages of development during
which certain kinds of knowledge emerged\cite{Piaget}, including
logical-mathematical knowledge related to the kind we wish to foster.
Additionally, by using our devices as an assessment vehicle for children's
spatial reasoning, one can position our work as part of a tradition (dating back
at least to Piaget \cite{piaget1967child}) in understanding spatial thinking and
its development (cf. also \cite{newcombe2003making} for a more recent treatment
of the subject). While Piaget's specific theories have been strongly
challenged\cite{Esther}\cite{Repacholi}, his influence was (and is!) extremely
important. Seymour Papert, one of Piaget's intellectual descendants, published
Mindstorms\cite{mindstorms} in 1980 and with it introduced his own ideas about
constructivism. Combined with the advent of the physical Logo turtle, Papert
brought many constructivist ideas into the modern age and opened the door for a
technical and cognitive exploration of how computation and interactive objects
could be combined to examine the link between tangibles and children's learning.

% It should also be noted, along these lines, that our early pilot test experience
% suggested a use for our work as an assessment vehicle for children's spatial
% cognition. We have, for instance, given children a pattern of lights
% and ask them to match that pattern to one of a set of physical or pictorial
% solid representations as well as ask children to recreate a variety of
% physical solids (such as a plastic prism or tetrahedron) by selecting the
% appropriate set of lights, and note their development and difficulties in doing
% so. By using the UCube as an experimental device in this fashion, one can
% position this work as part of a tradition (dating back at least to Piaget
% \cite{piaget1967child}) in understanding spatial thinking and its development
% (cf. also \cite{newcombe2003making} for a more recent treatment of the subject).

While a rich and diverse lineage of tangible and embedded user interfaces has
progressed since (and partially because of) Papert, the genealogy of the
proposed work derives from an interest not only in constructivist-like
activities, but in theories about how interaction with physical objects may be
beneficial to learning. In cognitive science, the area of embodied cognition
examines the ways in which our interactions with the physical world shape our
cognitive experiences from a body-centric point of view. More specifically,
embodied cognition holds that our cognitive processes are `deeply rooted in the
body's interactions with the world'\cite{Wilson}. This is in stark contrast to
decades of research in cognitive science wherein the mind was viewed as a sort
of central but detached information processing unit where motor-sensory
functions were more-or-less secondary inputs and outputs to a main
system\cite{clark1998being}.

Although there are several different tenets of this body-centric view, the
primary conclusion relevant to our proposal is that interactions with physical
objects can shape, clarify, and reinforce our cognitive processes in scores of
disparate areas. Of keen interest for this work in particular is a domain
referred to as embodied mathematics.
Lakoff and Nu\~nez\cite{lakoff} give a fascinating account of the origins of
mathematics from an embodied point of view. They propose that humans, by virtue
of their interactions with the physical world, inevitably form certain
intuitions of a mathematical nature. Recognizing small numbers of objects (e.g.
the pre-verbal ability to do arithmetic with less than five objects),
estimation, and simple comparisons are a few of the examples given
in\cite{lakoff}. From these basics, they argue that four kinds of physical
operations (object collection, object construction, using a measuring stick, and
movement along a path) form the basis of simple arithmetic. Although the book
postulates about concepts as ungrounded and seemingly abstract as infinity, for
our work it is enough to suggest that the interactions present in our designs
follow from these four operations and may in fact contribute to the
solidification of more complex mathematical ideas in 3D modeling and digital
fabrication (e.g. forming correct mental models of 3-dimensional objects). Such
notions of embodied mathematics have�even before the Lakoff/Nu\~nez text�played
a role in discussions of the development or instruction of mathematical ideas.
The link between physical experience and mathematical growth was a strong
element, for instance, in Montessori's work (see, e.g.,
\cite{hainstock1978essential}); much of the motivation behind traditional
mathematical ``manipulatives'' such as number rods and balance beams can also be
traced to this intellectual tradition. More recently, theoretical discussions of
embodied cognition have given rise to fine-grained observations of the
connections between bodily activity and mathematical learning:
Goldin-Meadow\cite{goldin2005hearing}, for instance, describes a fascinating
line of research in which children's nonverbal gestures appear to both reflect
and, in some cases, anticipate their verbal understanding of concepts such as
conservation and ``inverse operations''. In other work, Ehrlich, Levine and
Goldin-Meadow show that through an analysis of hand gestures, one is not only
able to predict a subject's `readiness' to learn mathematical
concepts\cite{goldin} but that the kinds of gestures children make (those
relating to movement, for example) are correlated with spatial reasoning
ability\cite{ehrlich2006importance} and performance on mental transformation
tasks.
 
Pedagogical research in embodied mathematics has, moreover, proceeded
hand-in-hand with the development of desktop, embedded, or portable
technological artifacts to support the link between bodily actions and
mathematical conceptualization. Papert's discussions of the Logo computer
language \cite{mindstorms} reveal this connection early in the history of
children's computing: Papert discussed, for example, the way in which the
program for a Logo circle resonated with children's bodily understanding of
moving in a circular path. More recently, Nemirovsky et al.
\cite{nemirovsky1998body} describe the use of a computer-based motion detector
system to assist children in the development of intuitions behind graphing;
Howison et al. \cite{howison2011mathematical} used a device based on a
Nintendo Wii remote to assess children's understanding of ratio (the children
attempt to move their arms in a manner illustrating a target ratio); Bakker et al.
\cite{bakker2011moso} created a collection of handheld objects (``MoSo
Tangibles'') with embedded sensors to help children learn about musical ideas
via hand motions such as waving, squeezing (pressing hands together), and
shaking up and down, among others; Mickelson and Ju \cite{mickelson2011math} use
sophisticated video and projection equipment as the basis of activities through
which children can learn about mathematical ideas (e.g., symmetry, rotation
angles) via large-scale physical movements.
 
In their section on `Thinking Through Doing', Klemmer et
al.\cite{Klemmer:2006:BMF:1142405.1142429} give a particularly poignant summary
of why we ought to consider the body as instrumental in any human-computer
interaction design, stepping through many of the concepts outlined above. In
fact, the marriage of ideas derived from Papert's work with the conclusions of
embodied cognition are not new, and appear to substantiate our motivations to
produce tangible, manipulative interfaces as opposed to purely 2-dimensional
screen-based work. In the mid-to-late 1990's, research examining the ways in
which physical objects might be infused with computational ability started to
coalesce around several themes\cite{Eisenberg:1996:RMV:257089.257230}. Resnick's
work with ``digital
manipulatives"\cite{Resnick:1998:DMN:274644.274684}\cite{Zuckerman:2005:ETI:1054972.1055093}
specifically references the contributions of Froebel and Montessori in the
design of a series of ``programmable bricks'' with computational ability whose aim
is to make certain specific concepts (e.g. systems-level thinking) more salient
for the user. Ishii's work on breaking down the divide between physical and
virtual worlds into `tangible
bits'\cite{Ishii:1997:TBT:258549.258715}\cite{Ishii:2008:TBB:1347390.1347392}
has subsequently set the stage for a new family of tangible interface designs
that support the kind of embodied interactions that our work seeks to produce.
By constructing environments and artifacts that focus on the possible physical
representations of computational components, these works (among others) created
the philosophical space to delve into how tangible objects might affect users at
a cognitive level. Our proposal is a confluence of both tangible and cognitive
design; as Resnick states, `We are interested in Things That Think only if they
also serve as Things To Think With'\cite{Resnick:1998:DMN:274644.274684}.

Having shown several PopCAD prototypes in Chapter 2 representative of a
``renaissance'' in papercrafting by infusing it with electronics, it is worth
situating that work in relation to that of other researchers in this (still
embryonic) field. The blending of traditional papercrafts with emerging
technology is in fact still a relatively novel technique, but there is a
remarkable community of researchers beginning to explore this area. For us, a
special debt is owed to Leah Buechley's High-Low Technology group at the MIT
Media Lab; that group first (to our knowledge) introduced conductive ink and
copper tape into paper-based projects. Early (c. 2008) use of conductive ink
with microcontrollers on a paper substrate can be found in
\cite{buechley2009paints} and \cite{Eisenberg:2009:CPR:1551788.1551790} with the
development of paper-based Arduino processors and simple electronic components
(e.g. LEDs, toy motors, switches) that could be placed onto conductive paint to
form an electronic connection. This work culminated with a paper application
usually reserved for home remodeling: a ``living wallpaper''\cite{Buechley2010}
where passers-by could trigger light, movement, and sound by interacting with
different parts of the surface (see Figure \ref{fig:paperCrafts}).

\begin{figure}[ht]
\begin{center}$
\begin{array}{cc}
\includegraphics[width=.45\linewidth]{images/elect_popables}&
\includegraphics[width=.45\linewidth]{images/living_wall}
\end{array}$
\end{center}
\caption{Examples of paper-based electronics: Electric Popables (left) is a
pop-up book infused with a variety of paper-friendly electronics. The Living
Wall (right) is a complete interactive environment embedded in wallpaper,
reacting with light, sound, and movement.}
\label{fig:paperCrafts}
\end{figure}


These early efforts in turn spawned developments that further refined the
expressive potential of paper-based electronics, infusing traditional
papercrafts with new elements and abilities. An electronic pop-up book by Qi and
Buechley\cite{Qi:2010:EPE:1709886.1709909} re-imagined the traditional pop-up by
infusing each page with paper-friendly, interactive circuitry (e.g. by using a
copper tape circuit to power LEDs in a pop-up cityscape), and from which PopCAD
certainly owes some debt. Other projects in this vein include techniques to
animate origami structures through shape-memory alloy (SMA)\cite{Qi2012}, using
SMAs in the design and fabrication of printable paper-based devices (e.g.
speakers and lamps)\cite{Saul2010a}, storytelling and craft-making through
electronically-enhanced storybooks and workshops
\cite{Jacoby2013a}\cite{Buechley2012}\cite{Sylla2012} and the use of small
microcontrollers incorporated into programmable paper-based
sculptures\cite{Mellis:2013:MMC:2460625.2460638}.

These efforts have focused on the creation of compelling(either electronically
or digitally enhanced) papercrafts. As noted in the introduction, there are
numerous technological developments that, in combination, serve to accelerate
the development of paper mechatronics. For instance, Kawahara et
al.\cite{Kawahara2013} describe how inkjet-ready conductive ink can allow
circuits to be printed easily and directly onto paper; and Koizumi et
al.\cite{Koizumi2010a} present a toolkit for wireless control of movable paper
toys, Zhu et al.\cite{Zhu2011a} describe a method for wireless power transfer
for paper computing, and Coelho et al.\cite{Coelho2009} have achieved the direct
embedding of conductive components during the papermaking process.

Of particular interest for the current work are explorations focusing on 3D
modeling and perception with tangible interfaces. Prime examples include
software that allows for 3D shapes to be flattened into paper-printable,
origami-esque polyhedra\cite{Eisenberg:1997:HUT:238218.238312}, a construction
kit with kinetic memory so as to record and playback certain user-generated
manipulations\cite{Raffle:2004:TCA:985692.985774}, as well as several variations
of ``smart-cube'' interfaces
\cite{Watanabe:2004:SAI:1037851.1037874}\cite{Schweikardt:2006:RRC:1180995.1181010}
that encourage spatial and logical reasoning in order to make use of the
computational aspects of the cubes. While diverse in their implementation, these
kits point to ways in which interface design can tease out the kind of
3-dimensional problem-solving and exploration present in the proposed work.

\begin{figure}[ht]
\begin{center}$
\begin{array}{cc}
\includegraphics[width=.54\linewidth]{images/activecube}&
\includegraphics[width=.37\linewidth]{images/roblocks}
\end{array}$
\end{center}
\caption{Left: The ActiveCube system. Right: The Roblocks system.}
\label{fig:activecube}
\end{figure}

Related contributions focus more on the cognitive processes involved when
exploring embodied interfaces with children. Research on supporting creative
problem solving with children\cite{Bevans:2011:SCC:1979742.1979838}, arguing for
a kindergarten-influenced approach to creative
thinking\cite{Resnick:2007:IRN:1254960.1254961}, embodied approaches to
analyzing children's interactions with smart
objects\cite{Antle:2009:THE:1520340.1520612}, as well as the embodied design of
interfaces for introducing mathematical concepts to
kids\cite{Abrahamson:2011:TED:1999030.1999031} have shown a great degree of
correlation between physical interaction and learning in children.


\begin{figure}[ht]
\begin{center}$
\begin{array}{cc}
\includegraphics[width=.40\linewidth]{images/interactiveLaser}&
\includegraphics[width=.54\linewidth]{images/interactiveFab}
\end{array}$
\end{center}
\caption{Examples of interactive fabrication interfaces: Constructable (left)
allows users to control a laser cutter with a set of physical tools as opposed
to a pre-defined design file. Shaper (right), and interactive fabrication tool
using expanding polyurethane foam.}
\label{fig:interactiveFab}
\end{figure}



\begin{figure}[ht]
\begin{center}$
\begin{array}{cc}
\includegraphics[width=.45\linewidth]{images/kidcad}&
\includegraphics[width=.45\linewidth]{images/easigami}
\end{array}$
\end{center}
\caption{Left: The KidCAD interface showing a model Zebra and its 2.5D
impression on screen.
Right: The Easigami system, showing a series of connected polygonal faces with
smart-hinges and embedded electronics.}
\label{fig:kidcad}
\end{figure}

Yet so far, there have been few attempts to design embodied interfaces for
children that specifically address the growing presence and availability of
digital fabrication tools.
KidCAD\cite{Follmer:2012:KDR:2207676.2208403}, a deformable pad that captures
the 2.5D geometry of depressions made on the underside of the surface, was a
very promising idea in that it allowed very young children to take small objects
from their surroundings (or their hands) and `stamp' them into the pad - an
intuitive and satisfying experience. Unfortunately, the authors intentions to be
able to output the geometry to 3D printers has not yet manifested.
Easigami\cite{Huang:2012:EVC:2148131.2148143} is a set of interchangeable and
interlocking polyhedral faces with smart `hinges' that can reproduce the
morphology of a set of connected faces while connected to a computer. In
contrast, Easigami \emph{is} able to export this morphology to a
stereolithography file ready for 3D printing. There are several other interfaces
that deal with `interactive fabrication'\cite{Willis:2010:IFN:1935701.1935716};
devices that manipulate materials interactively based on various input from a
user, such as controlling a laser cutter with a laser pointer (instead of
through a CAD program)\cite{Mueller:2012:ICI:2380116.2380191}, or a wearable
device that takes in a CAD file and provides haptic feedback to make the
physical creation of the device by hand easier, even for a
non-fabricator\cite{Zoran:2013:FFD:2470654.2481361}.
These projects, as well as several others that deal specifically with digital
fabrication for laser
cutting\cite{Johnson:2012:SMS:2212776.2212390}\cite{Willis:2010:SSB:1709886.1709890},
are examples of the subset of tangible interfaces to which this work belongs -
namely, those concerned with providing a means to engage with digital
fabrication technologies in a more intuitive, embodied fashion. However, with
the exception of KidCAD and Easigami these designs are not made with children in
mind, nor do they cover the range of possibilities for child-friendly input
devices that focus on 3D-printing. Thus, we argue that there is room in this
area for the work described in the thesis, as well as a lineage that suggests
meaningful results may follow from continuing to explore the incorporation of
tangible interfaces with embodied design.

Specifically, we see our devices as part of a larger, burgeoning ``technological
ecosystem'' around the activity of three-dimensional printing. The introduction
chapter to this work noted several prominent researchers who argue for the
democratization of this technology, and for its applications to education.
Indeed, exciting early work has been done in applying 3D printing to education
in fields such as architecture \cite{breen2003tangible}, solid geometry
\cite{hart2008procedural}, and mechanical design \cite{lipson20053}.
Our devices are specifically designed so that they can be employed by younger
students � younger, for instance, than the typical (undergraduate-age)
architecture student - and certainly less skilled or experienced with
traditional 3D modeling software. The devices were created to enable children to
specify and identify three-dimensional shapes by hand motions (instead of, by
contrast, using symbolic commands directed at a two-dimensional screen display).
At the same time, they are not simply devices for mathematical instruction,
nor even a general tool for mathematical design - but as a suite of
experiential, embodied interfaces for engaging youth in a variety of spatial
design activities aimed not only at learning but at democratizing authorship
for 3D printing as well.


% It should also be noted, along these lines, that our early pilot test experience
% suggests a potentially fruitful use for the UCube as an assessment device for
% children's spatial cognition. (The young subject who suggested that it could be
% made into a "puzzle game" is anticipating our thoughts here!) A researcher
% could, for instance, give children a pattern of lights and ask them to match
% that pattern to one of a set of physical or pictorial solid representations; or
% one might ask children to recreate a variety of physical solids (such as a
% plastic prism or tetrahedron) by selecting the appropriate set of lights, and
% note their development and difficulties in doing so. By using the UCube as an
% experimental device in this fashion, one can position this work as part of a
% tradition (dating back at least to Piaget \cite{piaget1967child}) in
% understanding spatial thinking and its development (cf. also
% \cite{newcombe2003making} for a more recent treatment of the subject).

% It should also be noted, along these lines, that our early pilot test experience
% suggests a potentially fruitful use for the UCube as an assessment device for
% children's spatial cognition. (The young subject who suggested that it could be
% made into a "puzzle game" is anticipating our thoughts here!) A researcher
% could, for instance, give children a pattern of lights and ask them to match
% that pattern to one of a set of physical or pictorial solid representations; or
% one might ask children to recreate a variety of physical solids (such as a
% plastic prism or tetrahedron) by selecting the appropriate set of lights, and
% note their development and difficulties in doing so. By using the UCube as an
% experimental device in this fashion, one can position this work as part of a
% tradition (dating back at least to Piaget \cite{piaget1967child}) in
% understanding spatial thinking and its development (cf. also
% \cite{newcombe2003making} for a more recent treatment of the subject).



% from IDC 2011
% There are several strands of research that have strongly influenced the design
% (and motivation) for the UCube. Perhaps the most fundamental of these is in the
% area of "embodied mathematics"� that is, the notion that mathematical thinking
% and learning are affected by, and perhaps grounded in, metaphors derived from
% bodily experience. The most thorough and discursive (though largely theoretical)
% discussion of these ideas is in the foundational text by Lakoff and Nu�ez
% \cite{lakoff}:
% the authors discuss physically-derived metaphors that underlie such essential
% mathematical ideas as numbers, operations, and sets. Such notions of embodied
% mathematics have�even before the Lakoff/Nu�ez text�played a role in discussions
% of the development or instruction of mathematical ideas. The link between
% physical experience and mathematical growth was a strong element, for instance,
% in Montessori's work (see, e.g., \cite{hainstock1978essential}); much of the
% motivation behind traditional mathematical ``manipulatives'' such as number rods
% and balance beams can also be traced to this intellectual tradition. More
% recently, theoretical discussions of embodied cognition have given rise to
% fine-grained observations of the connections between bodily activity and
% mathematical learning: Goldin-Meadow\cite{goldin2005hearing}, for instance,
% describes a fascinating line of research in which children's nonverbal gestures
% appear to both reflect and, in some cases, anticipate their verbal understanding
% of concepts such as conservation and ``inverse operations''.


% Pedagogical research in embodied mathematics has, moreover, proceeded
% hand-in-hand with the development of desktop, embedded, or portable
% technological artifacts to support the link between bodily actions and
% mathematical conceptualization. Papert's discussions of the Logo computer
% language \cite{mindstorms} reveal this connection early in the history of
% children's computing: Papert discussed, for example, the way in which the
% program for a Logo circle resonated with children's bodily understanding of
% moving in a circular path. More recently, Nemirovsky et al.
% \cite{nemirovsky1998body} describe the use of a computer-based motion detector
% system to assist children in the development of intuitions behind graphing;
% Howison et al. \cite{howison2011mathematical} used a device based on a Wii
% remote to assess children's understanding of ratio (the children attempt to move
% their arms in a manner illustrating a target ratio); Bakker et al.
% \cite{bakker2011moso} created a collection of handheld objects (``MoSo
% Tangibles'') with embedded sensors to help children learn about musical ideas
% via hand motions such as waving, squeezing (pressing hands together), and
% shaking up and down, among others; Mickelson and Ju \cite{mickelson2011math} use
% sophisticated video and projection equipment as the basis of activities through
% which children can learn about mathematical ideas (e.g., symmetry, rotation
% angles) via large- scale physical movements.
% The development of the UCube follows within this tradition, in that the device
% was created to enable children to specify and identify three-dimensional shapes
% by hand motions (instead of, by contrast, using symbolic commands directed at a
% two-dimensional screen display). At the same time, the UCube is not simply a
% device for mathematical instruction, but is more generally a tool for
% mathematical design. As noted at the outset of this paper, the intent of the
% UCube is to enable youngsters not only to learn about but also to build
% mathematical shapes.
% Specifically, we see the device as part of a larger, burgeoning ``technological
% ecosystem'' around the activity of three-dimensional printing. The first section
% of this paper noted several prominent researchers who argue for the
% democratization of this technology, and for its applications to education.
% Indeed, exciting early work has been done in applying 3D printing to education
% in fields such as architecture \cite{breen2003tangible}, solid geometry
% \cite{hart2008procedural}, and mechanical design \cite{lipson20053}. The UCube
% is designed so that it can be employed by younger students�younger, for
% instance, than the typical (undergraduate-age) architecture student. At the same
% time, we see no reason at all why the device could not be used by adult or
% professional-level students�particularly if (as we anticipate) the device and
% software are made more expressive or powerful in future iterations.
% It should also be noted, along these lines, that our early pilot test experience
% suggests a potentially fruitful use for the UCube as an assessment device for
% children's spatial cognition. (The young subject who suggested that it could be
% made into a "puzzle game" is anticipating our thoughts here!) A researcher
% could, for instance, give children a pattern of lights and ask them to match
% that pattern to one of a set of physical or pictorial solid representations; or
% one might ask children to recreate a variety of physical solids (such as a
% plastic prism or tetrahedron) by selecting the appropriate set of lights, and
% note their development and difficulties in doing so. By using the UCube as an
% experimental device in this fashion, one can position this work as part of a
% tradition (dating back at least to Piaget \cite{piaget1967child}) in
% understanding spatial thinking and its development (cf. also
% \cite{newcombe2003making} for a more recent treatment of the subject).


%From IDC 2014 on paper-based electronics
% Having shown several representative prototypes of our own work in paper
% mechatronics, it is now worth situating that work in relation to that of other
% researchers in this (still embryonic) field. The blending of traditional
% papercrafts with emerging technology is in fact still a relatively novel
% technique, but there is a remarkable community of researchers beginning to
% explore this area. For us, a special debt is owed to Leah Buechley's High-Low
% Technology group at the MIT Media Lab; that group first (to our knowledge)
% introduced conductive ink and copper tape into paper-based projects. Early (c.
% 2008) use of conductive ink with microcontrollers on a paper substrate can be
% found in \cite{buechley2009paints} and \cite{Eisenberg:2009:CPR:1551788.1551790}
% with the development of paper-based Arduino processors and simple electronic
% components (e.g. LEDs, toy motors, switches) that could be placed onto
% conductive paint to form an electronic connection. This work culminated with a
% paper application usually reserved for home remodeling: a ``living
% wallpaper''\cite{Buechley2010} where passers-by could trigger light, movement,
% and sound by interacting with different parts of the surface.
% 
% 
% These early efforts in turn spawned developments that further refined the
% expressive potential of paper-based electronics, infusing traditional
% papercrafts with new elements and abilities. An electronic pop-up book by Qi and
% Buechley\cite{Qi:2010:EPE:1709886.1709909} re-imagined the traditional pop-up by
% infusing each page with paper-friendly, interactive circuitry (e.g. by using
% a copper tape circuit to power LEDs in a pop-up cityscape). Other projects in
% this vein include techniques to animate origami structures through shape-memory alloy
% (SMA)\cite{Qi2012}, using SMAs in the design and fabrication of
% printable paper-based devices (e.g. speakers and lamps)\cite{Saul2010a},
% storytelling and craft-making through electronically-enhanced storybooks and
% workshops \cite{Jacoby2013a}\cite{Buechley2012}\cite{Sylla2012} and the use
% of small microcontrollers incorporated into programmable
% paper-based sculptures\cite{Mellis:2013:MMC:2460625.2460638}.
% 
% These efforts have focused on the creation of compelling(either electronically
% or digitally enhanced) papercrafts. As noted in the introduction, there are
% numerous technological developments that, in combination, serve to accelerate
% the development of paper mechatronics. For instance, Mueller et
% al.\cite{Mueller:2012:ICI:2380116.2380191} describe the use of a laser cutter to
% produce origami figures; Kawahara et al.\cite{Kawahara2013} describe how
% inkjet-ready conductive ink can allow circuits to be printed easily and directly
% onto paper; and Koizumi et al.\cite{Koizumi2010a} present a toolkit for wireless
% control of movable paper toys, Zhu et al.\cite{Zhu2011a} describe a method for
% wireless power transfer for paper computing, and Coelho et al.\cite{Coelho2009}
% have achieved the direct embedding of conductive components during the
% papermaking process.

% These last efforts are powerful examples of expanding techniques--they signal
% the emergence of a new territory within which to explore paper-based electronics.
% Our own prototypes are intended to continue this communal development of
% techniques and examples, but there are several factors that distinguish our work
% from that of other efforts. First, several of our own prototypes (e.g., PopCAD,
% the bicycle rider, and the cherry blossom painting) may be seen as incorporating
% paper elements as portions of larger, composite systems. PopCAD is a paper-based
% input device; one might think of it as one early foothold in an unexplored
% landscape of paper input devices for children's activities. The bicycle rider is
% an artifact that combines a desktop computer screen with a paper model; again,
% one could think of it as an exemplar of blending papercrafts with (e.g.)
% high-resolution graphics, or one could imagine websites designed to work as
% ``background graphics'' for electronically-controlled paper constructions. The
% cherry blossom painting is (in a sense) the ``flip side'' of PopCAD; whereas
% PopCAD is a paper-based input device, the cherry blossom painting is a
% paper-based display for output. And once more, one could take the example still
% further: paper mechanisms or models could be moved or controlled as components
% of extended output displays that combine physical and screen-based elements.

% More generally, we see our prototypes as (still-early) pointers toward a new
% genre of activities for children. In the final section, we turn our attention to
% the mechatronic future of children's papercrafts.







\chapter{Evaluation}
\label{evaluation}

This chapter is devoted to the description and discussion of three separate user
studies with the devices introduced in Chapter 2. Two studies were performed
with the original UCube device (one more informal than the other), while a longer,
more detailed study involved both the SnapCAD and PopCAD systems. We present the
procedure, results, and basic observations of each study in this chapter, and
discuss the results more thoroughly in the next chapter.

\section{UCube Pilot}

Early in 2011, shortly after the UCube prototype was complete, we conducted an
initial (and informal) pilot study with the UCube. Our participants were a group
of 12-14 year old middle school children from a local middle school multimedia
class. We had fourteen participants (predominantly Caucasian), consisting of
five girls and nine boys, who were divided into six groups (five groups of two,
one group of four).

\subsection{Procedure}

Participants were asked to model a sequence of five shapes of increasing
complexity using the UCube along with the companion software. The target shapes
were displayed on one half of a computer screen, while the UCube software
showing the live model was displayed on the other half (as in
\ref{fig:ucube_test1}). The first shape that participants were asked to model
was a straight vertical line; after this, the requested shapes were a diagonal
line, a cube, a triangular prism, and finally an irregular polyhedral object. No
shape required more than four towers to complete, and shapes were always
presented in the same order.

\begin{figure}[!ht]
\begin{center}$
\begin{array}{cc}
\includegraphics[width=.8\linewidth]{images/ucubescreentest}
\end{array}$
\end{center}
\caption{A screenshot of the testing setup, with the live output from the UCube
on the right and the target shape on the left.}
\label{fig:ucube_test1}
\end{figure}

Participants were instructed to place the tower on the board (but not shown
how), and were told that the software model could be rotated and filled in using
the keyboard and mouse, should that help them complete the task. The
participants were not given any hints as to how to complete the shapes and were
not told when they had the correct configuration (they had to indicate their
belief that the model was done). Participants were also instructed to ``think
aloud'' about their actions. The main purpose of the pilot study was to get an
initial impression of how the UCube would act as an accessible 3D modeling
tool - how well it could help ``3D novices'' overcome the ``2D bottleneck''.

\subsection{Results and Observations} 
Of the six groups who participated, four groups successfully modeled all five
shapes, one group ran out of time after three shapes, and one group finished one
shape, for a total of 24 of 30 possible shapes, or 80\%. Sessions lasted between
17 and 30 minutes. A variety of problem-solving strategies were observed during
testing, as the participants tended to treat the exercise as a sort of puzzle to
be solved. Simple methods equivalent to ``try and see'' were common, and seemed
to serve as a base point from which to draw conclusions about the relationship
between the 3D model and 2D on-screen representation (e.g. ``No, not there, up
one''). More sophisticated strategies were also observed: ``deconstructing''
more complex shapes into smaller, easier-to-model shapes (e.g. thinking of one
side of a cube as a square) was observed from several groups. Another popular
technique was to systematically match the on-screen perspective from the live
model with the shape they were attempting to model (e.g. ``Okay, first let's do
the top view, and then go from the side''). By orienting the two models
similarly, participants were able to make more accurate modeling decisions as
well as check their model against the on-screen shape. Counting distance in
terms of spaces on the board, between switches, or between dots on the screen
was also a very common technique of reasoning about and describing position. For
example, by counting that two vertices of a shape were separated by ``two dots
over and one down'' on the screen, subjects were able to count the distance out
on the physical UCube board. A few of the more mathematically-advanced
participants used terms such as ``axis'' and ``origin'' to orient themselves and
describe various positions on the board to their partners.
Another revealing observation in the pilot study was that, in the few instances
of mechanical failure (certain switches not lighting up, towers not plugging in
properly, or points not showing up on screen) the participants were still able
(with a high degree of certainty) to complete the assigned tasks. This appears
to indicate that, as opposed to arbitrarily moving the towers around until the
two sides of the computer screen looked the same, participants had formed a more
substantial mental model of the relationship between the UCube interface and the
2D representations on the screen. That opens the possibility that by performing
the embodied interactions necessary to operate the UCube, participants had
actually strengthened their understanding of how 3-dimensional space is
typically represented on a 2D screen. Although a small, informal study on its
own, this finding would strengthen the argument for using the UCube in an
educational setting to improve understanding of 3D space, as well as providing a
gateway for youngsters to move on to more complex modeling software.
While the variety of problem-solving techniques we witnessed is a testament to
the participants' ingenuity, it is also indicative of the fact that parts of the
UCube are not immediately intuitive. While none of the participants had trouble
understanding how to place the towers on the platform, the positions of the
towers and switches had to be reasoned out explicitly. It was common for groups
to clear the board of any poles when starting a new shape, even in cases where
an overlap of points or tower positions existed. Although most groups completed
all the shapes (or ran out of time), there were some expressions along the way
of the difficulty of the task (e.g. ``This is hard'', or ``This is like a
puzzle''). This indicates that design changes can be made in future iterations
to help clarify the correspondence between positions on the UCube platform and
the on-screen representation; for example, labeling the both the physical and
software grid with a simple alphanumeric system.
Despite these drawbacks as well as the inherent limitations of the UCube design,
these early results indicate a promising ability of youngsters to effectively
engage with the UCube interface. In fact, despite various levels of success in
completing the assigned tasks, the vast majority of participants exhibited a
high level of engagement with the UCube. For example, although the group that
completed only one shape seemed unmotivated to attempt to model the other
shapes, they continued to play with the interface and observe the results, even
stating ``this is fun'' and ``I like the switches''. Participants also saw
potential uses for the UCube outside of the specific exercise we assigned.
Comments (unsolicited) included, ``you should use this to teach geometry'' and
``you could make this a puzzle game''. At the very least, these early results
indicate that the majority of participants were able to take a 2-dimensional
representation on the screen and model its 3-dimensional equivalent using the
UCube, a very encouraging result in our eyes, prompting refinement of the UCube
software and hardware as well as further user study, as we explain below.

\section{Further UCube Study}

Early in 2012, we conducted an IRB approved follow-up user study of the UCube
with a group of 11-13 year olds. The group consisted of ten participants, eight
boys and two girls, from a local middle school multimedia class. Each
participant was individually led through two separate exercises (outlined below)
using the UCube.

\subsection{Procedure: Modeling}
Participants were handed a 3D-printed shape (modeled and printed from the UCube)
and were instructed to attempt to model the shape using the UCube. The
participant was initially allowed to hold the shape for approximately 10
seconds, after which they would hand the shape back to the facilitator and
attempt to model the shape from memory. Participants were instructed that they
may ask to hold the shape again, at which point they were allowed to hold it
throughout the duration of the modeling task. Additionally, users were
instructed that they had the option to skip a shape and return to it at a later
point in the exercise.
The five physical shapes presented were: a cube, a tetrahedron, a diamond, a
``house'' (a cube with a pyramid on top), and a complex irregular polyhedron.
The models were presented to the user starting with the cube (as this was deemed
to be the most basic shape with regard to modeling complexity). To avoid an
ordering bias, we randomized the presentation sequence of the next four shapes
using an online random order generator. If, after skipping a shape and returning
to it, the participant was still having difficulty, we offered them the
opportunity to attempt modeling the shape with the help of the UCube software,
the effects of which are discussed in the results section. Participants were
given a total of 25 minutes for the modeling exercise. We recorded, but did not
limit the modeling time per shape, only the total time for all five shapes.

\subsection{Procedure: Matching}
Participants were instructed to face away from the UCube while the facilitator
modeled a set of lights on the UCube corresponding to one shape among a set of
physical models laid out on the table next to the UCube.
Once the lights on the UCube were set up, the participant was instructed to turn
around, and indicate which physical object they thought the set of lights on the
UCube corresponded to.
There were nine physical models presented on the table, and consisted of a cube,
a tetrahedron, the ``house'' shape, a diamond, a triangular prism, an elongated
hexagon, a parallelogram, a trapezoid, and an irregular polyhedron (see
\ref{fig:ucube_shapes} for a picture of all the models). The shapes were always
presented on the table in the same order and orientation to avoid discrepancies
in perception or association.
Of the nine shapes, the participants were asked to match five of them (the cube,
the triangular prism, the parallelogram, the elongated hexagon, and the
trapezoid). Thus, only the cube was presented in both the matching and modeling
exercises. As with the modeling exercise, the cube was presented first, with the
remaining four shapes presented in a computer-generated randomized order.
Participants were given a total of ten minutes for the matching exercise,
corresponding to two minutes per shape, and were instructed to think aloud
during the process.

\begin{figure}[!ht]
\begin{center}$
\begin{array}{cc}
\includegraphics[width=.8\linewidth]{images/ucube_shapes}
\end{array}$
\end{center}
\caption{The nine models used during the user study:
a diamond, trapezoid, parallelogram, cube, elongated hexagon, irregular 
polyhedron, triangular prism, tetrahedron, house.}
\label{fig:ucube_shapes}
\end{figure}


\subsection{Results} 
While many established forms of 3D modeling systems can be confounding and
operationally too complex for a child to navigate, the UCube was positively
received and system instruction was accomplished with just a minor introduction
and demonstration (system instruction and demonstration lasted approximately 2-3
minutes). We found this first instance of system comprehension to offer some
validation that the UCube worked well as a user-friendly 3D modeling device.
This section will detail the outcome of both the modeling and matching tasks
performed.

\subsubsection{Exercise 1: Modeling} 
Modeling occurred under three conditions: recreate the object from memory,
construction of the object while it was in the participant�s possession, and
modeling the shape with the help of the UCube software. Overall, 21 of 50 shapes
were completed from memory, 12 of 50 were completed while holding the shape, and
a further 8 of 50 were completed with the aid of the UCube software, for a total
of 41 out of 50 shapes modeled successfully (82\%). Of the nine missed shapes,
seven were of the same shape, the complex polyhedron. The remaining two misses
were from the same participant, who ran out of time before completion.
Of the ten participants, eight were able to recreate the cube from memory,
whereas only four were able to recreate the diamond and the tetrahedron from
memory. Half of the participants constructed the house from memory, and no
participants were able to complete the irregular polyhedron from memory.
However, once shown the software the majority of the participants found the
modeling task significantly easier to perform. The irregular polyhedron was by
far the hardest shape and was only able to be completed by three of the ten
participants either after continued possession of the shape or using the
software.

\begin{figure}[!ht]
\begin{center}$
\begin{array}{cc}
\includegraphics[width=.47\linewidth, height=1.75in]{images/modeling1}&
\includegraphics[width=.47\linewidth, height=1.75in]{images/modeling2}
\end{array}$
\end{center}
\caption{Results of the modeling task, showing total modeling time spent per
participant (left) and average modeling time spent per shape across
participants (right).}
\label{fig:modeling}
\end{figure}


The graphs in Figure \ref{fig:modeling} represent the total completion times per
participant (on the left) and average time per shape (right). Two exceptional
completion times were observed, where participants finished modeling all the
shapes in under 10 minutes. However, the majority of participants finished the
task in the 19-25 minute range. Only one of the participants ran out of time.
Once participants had been introduced to the software, 9 of 10 of participants
were able to complete all but the irregular polyhedron. It is interesting to
note that of the 10 participants, the child that had the most difficult time
modeling, the lowest shape completion rate, and the longest completion time
during the matching exercise was the youngest participant.


\subsubsection{Exercise 2: Matching} 
Out of 50 matching tasks (five per participant), all but three tasks were
completed in 20 seconds or less. Figure \ref{fig:matching} displays the total
time spent on the matching task per participant (left) and the average completion
times for each shape (right). No participant selected the wrong shape (a few
preliminary ``mis-selections'' were made that the participants quickly
corrected), and all participants completed the task in well under the allotted
10 minutes. The lack of errors in the matching task is highly encouraging as a
basis from which to reason about youngsters' abilities to perceive and reason
about convex hulls as a set of lit vertices in space, meaning that this kind of
3D modeling interface might be applied to other domains (e.g., as a cognitive
assessment tool, a puzzle game, etc.) with some optimism.

\begin{figure}[!ht]
\begin{center}$
\begin{array}{cc}
\includegraphics[width=.47\linewidth, height=1.75in]{images/matching1}&
\includegraphics[width=.47\linewidth, height=1.75in]{images/matching2}
\end{array}$
\end{center}
\caption{Results of the matching task, showing total time spent per
participant (left) and average time spent per shape across
participants (right).}
\label{fig:matching}
\end{figure}


\subsection{Observations} 
Modeling trends as well as distinct modeling behaviors were documented in the
process. Common observations included building from the ground up (lowest
vertices first), building in the orientation that the object had been presented
in, not clearing the poles/lights from the UCube before starting to model a new
shape, and modeling a shape by breaking it up into discrete parts (e.g. a
participant building a house would commonly build a cube first and then add on a
vertex to the top; a participant constructing the diamond might combine two
opposite facing triangles).

\begin{figure}[!ht] \begin{center}$
\begin{array}{cc}
\includegraphics[width=.4\linewidth]{images/idc3}&
\includegraphics[width=.5\linewidth]{images/ucube1_user}
\end{array}$
\end{center}
\caption{Left: A participant using a strategy of
placing the physical model on top of the UCube while using both hands
simultaneously to manipulate the towers. Right: A user pointing at the software
representation of the shape with one hand, while manipulating the UCube
interface with the other hand.}
\label{fig:user_placedModel}
\end{figure}

Unique behaviors were exhibited in the modeling process as well, reflecting a
type of user specific construction-based problem solving. One participant used
their arm to connect the red lights of the UCube for shape definition. A few
participants oriented the object differently than how it had been
presented�typically this occurred for the modeling of those objects with a
pyramidal apex (tetrahedron, house, diamond). Apex formation was perhaps one of
the most difficult concepts for most participants to grasp, as it required them
to strategically align the base on a 3x3 grid so there was a middle plug for
them to create the apex. If participants were fixated on designing from a 4x4
grid then there was no center plug for them to create a midpoint. Some
participants ended up building an oblong polyhedron as opposed to a cube, or an
oblique polyhedron as opposed to an equilateral tetrahedron. Other observed
behaviors included a participant who modeled shapes by turning on lights for an
entire shape edge, as opposed to just the corners and a participant who built
shapes that were floating, as opposed to resting on the base of the UCube.
There were also some notable behaviors regarding physical and gestural actions
of the participants. Many participants modeled with both hands simultaneously,
placing towers and flipping switches without a clear preference for a dominant
hand. Participants would often gesture with their arms following an arc in
parallel with a face of the object they were currently modeling. This ``tracing''
behavior was also noticed when participants were holding a physical model and
tracing a side of the object with their fingertip, often while rotating the
object with the other hand. Finally, during object possession phase three
participants actually placed the 3D object on top of the UCube in the modeling
space while they reasoned out the construction (see
\ref{fig:user_placedModel} for an example). These gestural and ``embodied''
interactions with the UCube, combined with a high degree of modeling success
spurred us not only to create a more robust and expressive system (called -
SnapCAD - as detailed in Chapter 2), but to attempt to tease out the
relationships between modeling on these kinds of devices and the gestures and
speech produced when subjects were explaining their strategy in using the
devices. This eventually led to a comparative study using two new devices, two
new modeling modes, and introducing metrics to analyze some of the ``embodied''
aspects hinted at above.



\section{SnapCAD and PopCAD}

%- see
%http://silccenter.org/index.php/testsainstruments#MRT for the instruments, see
%http://www.spatialintelligence.org/publications_pdfs/Ehrlich\%20Levine\%20\%20Goldin-Meadow\%20\%282006\%29.pdf

Starting in early 2014 we conducted a study using both the SnapCAD and PopCAD
devices with a group of 11-18 year olds at a local drop-in enrichment program
that focuses on children from under-served and low socioeconomic communities.
Twenty participants enrolled in the study, consisting of 12 boys and 8 girls (no
one responded with other, although it was an option). We collected some basic
demographic information, including age, race, grade level, 3D modeling
experience, 3D printing experience, computer ownership and use, interest in
engineering, and how difficult they thought classes in school were. Parental
consent was obtained (and child assent given) for each subject in the study.

To present a snapshot of the demographic findings, then: the participants were
primarily of Latino or Hispanic descent, but also included those of
African-American, American-Indian, Asian, and Caucasian descent. Grade levels
ranged from 6th-12th, with an overall average of 7.9 (8.33 for boys, 7.75 for
the girls). Average age was 14 years, 1 month, 20 days (14 years, 6 months for
boys, 13 years, 7 months for girls). 18 of 20 participants had a computer at
home. Describing their comfort level using a computer on a scale from 1 to 10
(10 being most comfortable), the participants averaged 7.9 (8 for boys, 7.75 for
girls), with no scores below a 5. Of the participants who had a computer at home
(all but two of the subjects), two reported using it only a few times a year,
five used it a few times per month, four used it a few times per week, and five
reported using the computer everyday. Only three of the participants had any
experience with 3D modeling software. Interestingly, only two of the
participants had never heard of 3D printing before enrolling in the study, but
none of them had ever designed or printed anything using a 3D printer - further
underscoring the lack of available tools for novice designers. When asked about
their interest in engineering, only seven children (all boys) stated they were
definitely interested. However, only two children (both girls) stated that they
were definitely not. The rest (11 kids) stated that they were either ``maybe''
interested, or ``not sure''. When asked how difficult they felt school classes
were, six responded ``easy for me'', 10 said `somewhat easy for me', and four
responded ``somewhat hard for me'' (no one responded ``hard for me'').

\subsection{Procedure}

The study ran for seven weeks total, comprising several stages, the first being
a pre-assessment of spatial reasoning skills. The spatial reasoning assessment was
done using the ``Children's Mental Transformation Task'' developed by Susan
Levine (\cite{ehrlich2006importance} pp.1260-1261). In the task, participants
are shown two pieces of paper, side-by-side. One piece shows a 2D geometric shape,
split apart and rotated in one of several different ways. All shapes were
symmetrical either horizontally or vertically (or both), and thus split along
either a vertical or horizontal line of symmetry. Shapes were translated in one of four
different ways: (a) translated perpendicular to the line of symmetry (direct
translation), (b) translated and then moved diagonally apart (diagonal
translation), (c) rotated 45 degrees outward from the line of symmetry (direct
rotation), or (d) rotated and then moved diagonally apart (diagonal rotation).
The other piece of paper contained the geometric shape, recombined correctly,
along with three incorrect choices. In the study we conducted, participants were
given two sets of 10 shapes, one set as a pre-assessment before doing any
modeling, and another (completely different) set of 10 after completing the
entire study, as a post-assessment. Figure \ref{spatialTest} shows an example
instrument, with the four possible translations.

\begin{figure}[!ht]
\begin{center}
\includegraphics[width=.5\linewidth]{images/SpatialTestArray}
\end{center}
\caption{An example problem from the spatial reasoning exercise. The figure at
the top shows the choice array of four shapes, where the lower right figure is
the correct option. Examples (a) through (d) show the four different types of
translations found in the exercises - direct translation, diagonal translation,
direct rotation, and diagonal rotation.}
\label{spatialTest}
\end{figure}


After the pre-assessment, participants were split into two groups of 10 students
each - the selection alternated evenly based solely on order of participation -
with group A modeling first on the PopCAD and group B modeling first on the
SnapCAD (each device is described in Chapter 2). Each session begins with a
brief ($\approx$ one minute) introduction to the device, during which the
participant is told how to operate the device, but not what any of the software
buttons do, and given free time to become comfortable with the interface.
Participants were encouraged to explore both the interface, and the buttons in
the software that control the three primary modeling modes (convex hull, path,
minimal spanning tree).

Once the subject indicates that they are ready to move on (capped at 10
minutes), we move into a series of three modeling exercises that explore each of
the aforementioned modes. The basic operation and a brief explanation of each
mode were given to the participants as an introduction to each mode. Four
3D-printed models representative of each mode were presented to the user in an
order judged to be from least complex to most complex (and thus was the same for
each user), for a total of 12 modeling tasks across the three modes. 24 models
were used - one set of 12 was used across every user's first session
(independent of device), with a remaining 12 models used in every user's second
session. Figure \ref{3dModel} shows the two sets of models side-by-side.

\begin{figure}[!ht]
\begin{center}$
\begin{array}{cc}
\includegraphics[width=.47\linewidth]{images/round1shapes}&
\includegraphics[width=.47\linewidth]{images/round2shapes}
\end{array}$
\end{center}
\caption{The two groups of 12 3D printed models used in the first session
(left) and second session (right). Each row is a different modeling mode (back
= convex hull, middle = path, and front = minimal spanning tree). The shapes
were presented in order from left to right as pictured above.}
\label{3dModel}
\end{figure}


The tasks that follow are the same for each device:

Tasks 1-3: Convex Hull Modeling, Path Modeling, Minimal Spanning Tree Modeling

Before each set of modeling tasks, the participant was given a brief demo of
how each modeling mode interprets the points from the device.  The user was
then presented with a series of four (4) plastic, 3D-printed models that were
modeled on the device using the current modeling mode. For each of these shapes,
the participant attempted to recreate the shape using the modeling abilities
of the device and the software. The user was be instructed to indicate when
they believe they have successfully recreated the shape, as well as to think
aloud about their modeling process. The time to completion (of lack thereof),
completion code, observational notes, and video were recorded. If the user
indicates modeling success, they shall be asked to explain their modeling
strategy for the purpose of logging gesture and speech data.

Task 4: Freehand Modeling

After the modeling tasks are complete, participants were invited to
``freestyle'' model an object of their choosing, using any of the three modeling
modes. By asking participants to think aloud about their intentions and thinking
processes during this exercise, we aimed to gain a deeper understanding of the
strengths and weaknesses of the system, as well as the thought processes and
engagement of the users in attempting to model a specific model of their own
choosing. These saved models were analyzed, based on which mode was used to
create them, complexity (based on number of points, faces, segments, and
symmetry), and whether the shape was ``exploratory'' or ``intentional'' (i.e.
was the end artifact a result of sort of happy accident, or the result of
intentional process to create a specific model).

For the first three modeling tasks (but not the freestyle modeling), time to
completion (or request to move on) was recorded, along with an outcome code. The
outcome was coded according to a set of conditions detailed below in table
\ref{modelingError}, and was developed upon analysis of the recorded video, in
an attempt to fit the sorts of repeated behaviors that were in fact observed.

\begin{table}[!ht]
\small
    \caption[Coding rubric used in analyzing modeling exercise outcomes.]{
	The coding used in analyzing the modeling exercise outcomes, based on
	observations from video taken during the study.}
    \begin{center}
    \begin{tabular}{| p{3.5cm} | p{1.0cm} | p{7.2cm} | } \hline
	$Category$ & $Code$ & $Definition$   \\ \hline
	Correct & C & A complete and correct modeling of the shape \\ \hline
	Error in recognition & E1 & The correct shape was modeled, but the user did not
	identify it \\ \hline 
	Error in belief & E2 & A belief that the modeled shape has been modeled
	correctly, when it has not \\ \hline 
	Error in implementation & E3 & User knew shape was incorrect, and gave a
	correct explanation \\ \hline 
	Error in strategy & E4 & Knew shape was incorrect, and did not know why or gave
	an incorrect explanation as to why \\ \hline 
	Error in proportion & EP & The general shape is correct, but the proportions in
	one or more dimensions is off (e.g. too tall, not wide enough, etc.) \\ \hline
	Incomplete & I & Participant ran out of time, gave up, or asked to move on \\
	\hline
	\end{tabular}
   \\ \rule{0mm}{5mm}
\end{center}
\label{modelingError}
\end{table}

Participants were asked to ``think aloud'' about their process, difficulties,
modeling choices, etc. In the case that the user believed they had correctly
modeled the shape (cases C and E2 in table \ref{modelingError}) they were asked
to explain their modeling strategy.\footnote{Cases E1,E3,E4, and I did not
provide the grounds from which to ask about modeling strategy and so were not
recorded.} Their explanation was videotaped and analyzed based on the coding
strategies laid out in ``The Importance of Gesture in Children's Spatial
Reasoning"(\cite{ehrlich2006importance}, p.1264), laid out in table
\ref{GcodingStrategy} below. The rationale for performing this analysis in based
in part on work by Ehrlich, Levine, and Goldin-Meadow
\cite{ehrlich2006importance}\cite{levine1999early}\cite{goldin2005hearing},
which suggests that the frequency of gesture and relationships between speech
and gesture act as a window into the learning state and performance of the
subjects.

\begin{table}[!ht]
\small
    \caption[Coding rubric for speech and gesture during user explanation of
    modeling strategy]{ The various coding strategies used in the video
    analysis of subjects' modeling strategy explanations. Borrowed and adapted
    from \cite{ehrlich2006importance}.}
    \begin{center}
    \begin{tabular}{| p{1.5cm} | p{4.2cm} | p{4.2cm} | p{4.2cm} |} \hline
	$Category$ & $Definition$ &   $Speech$ $Examples$  & $Gesture$ $Examples$ \\
	\hline Movement & Any indication of movement & ``Just slide them together and then it
	looks like that'' & Miming movement with the hands\\ \hline 
	Perceptual Features & Focus on a particular feature of the model & ``Because
	there is a little bend in here and a point thing here'' & Pointing to a
	specific feature on the model \\ \hline 
	Perceptual Whole & Any indication of seeing the model as a whole & ``It looks
	like an arrow!'' & Gesture indicating inclusion of the whole shape \\ \hline
	Vague & An expression of strategy that the coder cannot decipher & ``Because I
	looked at that and I looked at the differences'' & Waving gestures above the
	computer device that do not indicate any specific strategy \\ \hline 
	Other & Any strategy not listed above & ``And here is like half of it.
	But so and two halves make a whole'' & Using the hand to form a straight line
	through the middle of the whole shape to represent the line of symmetry
	\\
	\hline
	\end{tabular}
   \\ \rule{0mm}{5mm}
\end{center}
\label{GcodingStrategy}
\end{table}


The second session was similar to the first, with the subject using the device
not used in session one (no subject used the same device twice), and with 12 new
models. Once modeling on the second device was completed, users took a
second spatial reasoning assessment of an additional ten questions to help gauge
if any meaningful difference in spatial reasoning skills has occurred throughout
the study.

A slightly modified version of the software was used for the user study,
eliminating several of the functions not being evaluated for the sake of
presenting a clear interface for the users. The multiple hull modes, spline,
load, and save functions (described in Chapter 2) were eliminated, and the rest
of the graphical user interface was reorganized and streamlined. We
combined the three different .STL export buttons into a single export button
that handled all three modes, changed the order of the remaining buttons and
made them larger, and made the X,Y, and Z axis markings larger.

\subsection{Results}

This section reports on the results from our study, relaying our findings across
both sessions, genders, modeling modes, and spatial reasoning scores in an
attempt to tease out what conclusions, if any, we might make about the strengths
and weaknesses of our devices as well as how interacting with our devices
affected user's spatial reasoning abilities, 3D modeling skills, or congruence
between speech and gesture in explaining the cognitive learning state of the
user.

\subsubsection{Modeling Results}

In this section we will focus on delivering the results from the modeling
exercises. Users went through two sessions, modeling 12 shapes each time (four
shapes each using convex hull, path, and minimal spanning tree modes) for a
total of 24 exercises. For each modeling task, a result code was recorded per
the rubric shown in table \ref{modelingError}. One user dropped out of the study
(user six) before completing round one, leaving us to report on 19 users for the
first modeling session, ten of whom started on the PopCAD and nine of whom
started with the SnapCAD. A further three users did not complete session two,
leaving 16 users, seven girls and nine boys, who were split evenly over the two
devices in the second session (eight each on PopCAD and SnapCAD).

% \begin{table}[!ht] 
% \small
%     \caption[Modeling Results Overview]{This table gives the subject's age,
%     gender, and number of correctly completed modeling tasks during each of the
%     two sessions.}
%     \begin{center}
%     \begin{tabular}{| c | c | c | c | c | c | c |} \hline
% 	$User$ & $Age$ & $Gender$ & $Device$ & $S1$ $Score$ & $Device$ & $S2$ 
% 	$Score$\\\hline 
% 	1 & 14 & M & Pop & 11 & Snap & 9 \\ \hline
% 	2 & 12 & F & Snap & 6 & Pop & 8 \\ \hline
% 	3 & 15 & M & Pop & 6 & Snap & 3 \\ \hline
% 	4 & 13 & M & Snap & 4 & Pop & 5 \\ \hline
% 	5 & 12 & M & Pop & 5 & Snap & 6 \\ \hline
% 	7 & 15 & M & Pop & 12 & Snap & 11 \\ \hline
% 	8 & 12 & F & Snap & 4 & Pop & 8 \\ \hline
% 	9 & 18 & M & Pop & 8 & Snap & 3 \\ \hline
% 	10 & 14 & F & Pop & 12 & Snap & 10 \\ \hline
% 	11 & 17 & M & Snap & 9 & Pop & 12 \\ \hline
% 	12 & 12 & M & Snap & 4 & Pop & 7 \\ \hline
% 	13 & 13 & M & Pop & 9 & -- & -- \\ \hline
% 	14 & 14 & M & Snap & 1 & Pop & 5 \\ \hline
% 	15 & 12 & M & Snap & 1 & -- & -- \\ \hline
% 	16 & 13 & M & Pop & 12 & -- & -- \\ \hline
% 	17 & 13 & F & Pop & 11 & Snap & 9 \\ \hline
% 	18 & 17 & F & Snap & 8 & Pop & 12 \\ \hline
% 	19 & 11 & F & Pop & 4 & Snap & 3 \\ \hline
% 	20 & 13 & F & Snap & 0 & Pop & 5 \\ \hline
% 	\end{tabular}
%    \\ \rule{0mm}{5mm}
% \end{center}
% \label{modelOverview}
% \end{table}

\begin{table}[!ht] 
%\small
    \caption[Modeling Results Overview]{An overview of the modeling task
    results, broken down into session number, gender, device, and modeling
    mode.}
    \begin{center}
    \begin{tabular}{| c | c | c | c | c | c | c | } \hline
	& $Session$ $1$ & \% & $Session$ $2$ & \% & $Total$ & \% \\\hline 
	$Overall$ $Correct$ & 127/228 & 55.7\% & 116/192 & 60.4\% & 243/420 & 57.9\% 
	\\
	\hline $Girls$ & 45/84 & 53.6\% & 55/84 & 65.5\% & 100/168 & 59.5\%  \\ \hline
	$Boys$ & 82/144 & 57.6\% & 61/108 & 56.5\% & 143/252  & 56.7\%   \\ \hline
	$PopCAD$ & 90/120 & 75\% & 62/96 & 64.6\% & 152/216 & 70.4\%  \\ \hline
	$SnapCAD$ & 37/108 & 34.3\% & 54/96 & 56.3\% & 91/204 & 44.6\%  \\ \hline
	$Convex$ $Hull$ & 40/76 & 52.6\% & 38/64 & 59.3\% & 78/140 & 55.7\%  \\ \hline
	$Path$ & 48/76 & 63.2\% & 44/64 & 68.8\% & 92/140 & 65.7\%  \\ \hline
	$Tree$ & 39/76 & 51.3\% & 34/64 & 53.1\% & 73/140 & 52.1\%  \\ \hline
	\end{tabular}
   \\ \rule{0mm}{5mm}
\end{center}
\label{modelOverview}
\end{table}

Out of the 228 modeling tasks in session one, the group successfully modeled
127, or roughly 56\%. Those users who started with SnapCAD performed 37 of 108
tasks, or 34\%, while those using the PopCAD device completed 90 of 120 tasks
correctly, for a success rate of 75\%. Girls completed 45 of 84 tasks (54\%),
while boys correctly completed 82 of 144 tasks (58\%). Individual scores ranged
from 0 to 12 (perfect), with an overall overage of 6.68 correct shapes per user.
Average correct shapes per user was 4.11 for SnapCAD and 9.00 for PopCAD.

In session two, 116 of 192 (60\%) tasks were performed correctly, with SnapCAD
modelers correctly representing 54 of 96 shapes (56\%) and PopCAD modelers
completing 62 of 96 shapes, or roughly 65\%. Girls completed 55 of 84 tasks
(65\%) while boys completed 61 of 108 tasks for 56\%. Individual scores ranged
from 3 to 12 (perfect), with an average of 7.25 correct shapes overall, while
the average correct shapes per user was 6.75 for SnapCAD and 7.75 for PopCAD.


% \begin{table}[!ht] 
% \small
%     \caption[Session one modeling results per shape]{ This table shows the
%     number of correct models generated from a given shape, broken down by
%     device, gender, and average modeling time spent on the shape. Session one
%     results only.}
%     \begin{center}
%     \begin{tabular}{| c | c | c | c | c | c | c |} \hline
% 	$Shape$ & $Total$ & $PopCAD$ & $SnapCAD$ & $Average$ & $PopCAD$
% 	& $SnapCAD$ \\
% 	 & $Number Correct$ & $Only$ & $Only$ & $Modeling Time$ & $Only$ & $Only$
% 	 \\\hline 
% 	 Convex Hull 1 & 8 & 7 & 1 & 6:33 & 5:25 & 7:49 \\ \hline 
% 	 Convex Hull 2 & 13 & 9 & 4 & 4:34 & 3:03 & 6:15 \\ \hline 
% 	 Convex Hull 3 & 8 & 7 & 1 & 5:29 & 4:33 & 6:32 \\ \hline 
% 	 Convex Hull 4 & 11 & 7 & 4 & 4:52 & 4:31 & 5:14 \\ \hline 
% 	 Path 1 & 17 & 10 & 7 & 2:08 & 1:22 & 3:00 \\ \hline 
% 	 Path 2 & 14 & 9 & 5 & 4:39 & 3:24 & 6:02 \\ \hline 
% 	 Path 3 & 8 & 6 & 2 & 6:21 & 4:52 & 8:00 \\ \hline 
% 	 Path 4 & 9 & 7 & 2 & 4:12 & 1:56 & 6:43 \\ \hline 
% 	 Tree 1 & 12 & 8 & 4 & 2:10 & 1:10 & 3:17 \\ \hline 
% 	 Tree 2 & 8 & 6 & 2 & 2:25 & 1:08 & 3:51 \\ \hline 
% 	 Tree 3 & 8 & 5 & 3 & 3:11 & 1:29 & 5:04 \\ \hline 
% 	 Tree 4 & 11 & 8 & 3 & 4:24 & 3:12 & 5:43 \\ \hline
% 	 \em{Total} & 127 & 90 & 37 & 4:15 & 3:00 & 5:38 \\ \hline
% 	\end{tabular}
%    \\ \rule{0mm}{5mm}
% \end{center}
% \label{codingStrategy}
% \end{table}

The two bar graphs in \ref{ModelingTimes} show the average modeling times broken
out over device and gender (on the top) and modeling mode (on the bottom).
Modeling times were recorded from the time the user was handed the shape until
they indicated either that (a) they believed the model to be complete, or (b)
they gave up, wished to move on, or thought they were as close as they were
going to get (though they knew their model to be incorrect). 

\begin{figure}[!ht]
\begin{center}$
\begin{array}{cc}
\includegraphics[width=.55\linewidth]{images/avgtimesmod} \\
\includegraphics[width=.55\linewidth]{images/avgtimesmodeltype}
\end{array}$
\end{center}
\caption{The average recorded modeling times for each session, broken out (on
top) by device and gender, and (on the bottom) by modeling mode. Error bars
show standard error ($SE$).}
\label{ModelingTimes}
\end{figure}

We can easily pick out a few trends from these two graphs: average modeling
session time went down significantly in the second session, regardless of device
or gender, although boys took less time in both sessions, and the PopCAD seemed
to take less time overall in each session than modeling on the SnapCAD (although
interestingly, the SnapCAD modelers in the second round improved on their times
from modeling on the PopCAD in the first round). When examining mode types, we
see a similar trend of significantly decreasing modeling times in the convex
hull and path modes, but curiously, not in the tree mode where times improved in
the second session by only a few seconds. While the minimal spanning tree mode
took subjects the least amount of time (of the three modes) in session one, the
improvement in both convex hull and path modeling times left the spanning tree
with slowest overall and average modeling times in session two.
Seeing as the minimal spanning tree mode posted the lowest percentage of correct
shapes in both rounds (and thus overall), we might expect the ranking we
observed in round two, where average modeling times corresponded with the
overall percentage of correct shapes. It seems plausible that mastery of the
tree mode is slower to arrive than either the convex hull or path modes, and
therefore one extra session produced more dramatic results in the other modes
(convex hull and path modeling both improved by almost 7\% in session two,
minimal spanning tree by less than 2\%). 

% At the end of both sessions, the subjects were given the opportunity to model a
% shape of their own design, using any of the modes presented during the modeling
% exercises. Each participant modeled after the first round, and then were given
% the opportunity to create a different shape after the second round that they
% wanted printed out instead (most subjects stuck with their original model). The
% prints from the 16 participants who finished the study are shown in Figure
% \ref{UserShapesAll}.
% 
% \begin{figure}[!ht]
% \begin{center}
% \includegraphics[width=.55\linewidth]{images/userShapesAll} 
% \end{center}
% \caption{A collection of the child-designed objects from the PopCAD/SnapCAD
% study.}
% \label{UserShapesAll}
% \end{figure}
% 
% Objects were created using all three modes, though only one participant chose
% the convex hull mode for their object; nine subjects used the path mode, while
% six used the minimal spanning tree mode. It should be noted that almost every
% participant explored all three modes on their own before settling on one they
% liked the best. Choices were sometimes based on strategy (e.g. only one mode was
% capable one making the shape they envisioned) while many users simply explored
% different modes and patterns until something struck their fancy.
% As evident in the picture, some users went with letters (usually the initial of
% their first name), others attempted to create a symbol they knew (e.g. one
% participant attempted the ``Tri-Force'' symbol from the Zelda video game
% franchise), and others (as hinted at in Figure \ref{UserShapesAll}) simply
% turned points on and off until they achieved an aesthetically pleasing object.
% 
% Of the 24 user-created models (19 from the first session, 5 more from the second
% session) we received an even number generated from each device (12 apiece). We
% were curious to see if, on the SnapCAD models, users took advantage of the
% greater expressive power offers by the larger input space (a $7^3$ grid compared
% to a $3^3$ grid). Of the 12 SnapCAD models, 9 of them would have been impossible
% to model using the PopCAD (without substantial use of the editing mode, at
% least). Of the five users who chose to model a new shape after the second round,
% four of them had used PopCAD in the first round and SnapCAD in the second round.
% Three of these four users modeled shapes they would not have been able to using
% PopCAD. However, based on the difficulty rubric, the average complexity of
% shapes modeled on the PopCAD was actually higher than that of SnapCAD (22.83 to
% 18.50). Interestingly though, the shapes with the three highest scores (all from
% PopCAD) were from users who chose to model different shapes after the second
% round in order to use the SnapCAD - but to create a ``simpler'' shape.
% 
% Interestingly, most of the ``intentional'' models - those models created from a
% firm mental model or notion of what the final shape should be - a preferred
% strategy was to use the path mode and a singular vertical plane (e.g. the first
% row of three towers on the PopCAD) to treat the device essentially as a
% 2-dimensional drawing tool. We can see (in the figure above) shapes like the
% star, or the letter ``s'' - while they print in 3D of course, the modeling
% necessary to create these shapes happens in 2D. Also of note is that the users
% overwhelmingly chose a vertical (as opposed to horizontal) plane in which to
% work. When we imagine ourselves drawing or writing, it is almost along a flat
% horizontal surface (except perhaps when writing on a whiteboard or painting on
% an easel), so why the preference for verticality? While we cannot know for
% certain, it is true that some amount of verticality is implied in the device -
% the towers themselves rise in vertical columns above the ``floor'' of the
% device. Additionally, the average time writing manually as opposed to typing on
% a computer has significantly decreased in recent years, so perhaps the vertical
% screen of a computer somehow relates. In any event, using the device as more of
% a 2D drawing instrument is a somewhat unexpected, but nonetheless welcome
% observation.
% 
% None of the participants
% refused to do the freehand modeling session, or quit in the middle of it - many
% subjects went over the allotted 10 minute modeling window exploring, testing
% ideas, rearranging point configurations, and generally being absorbed by the
% experience, which we found encouraging (and of course we allowed subjects to
% play as long as we could).

% Given the massive development (both cognitively and physically) that occurs
% between the age extremes in our subject population (11 to 18), it would be
% tempting (and even logical) to assume that the older subjects would perform much
% better on the modeling tasks than their younger counterparts. However, we found
% a very modest correlation ($r = .39, p < .15$) between age and the number of
% correctly modeled shapes, suggesting it may play less of a role then we would
% have suspected. It is possible that the statistics are slightly misleading here
% - the subject population was weighted toward the younger end of the spectrum:
% the average age was 13.8, while median age was 13.5, and the mode was 12 years
% old. Meaning the few older participants would have had to perform impossibly
% brilliantly (i.e., higher than the highest possible score) for a strong age to
% performance correlation to show up.

% \begin{table}[!ht] 
% \small
%     \caption[Modeling Times per user (overall)]{Modeling times per user over
%     all modeling exercises in session 1.}
%     \begin{center}
%     \begin{tabular}{| c | c | c | c | c | c | c | } \hline 
%     $Mode$ & $Combined$ & $Number$ & $PopCAD$ & $SnapCAD$ & $Girls$ &
%     $Boys$ \\
%     &  $Average Time$  & $Correct$ & $Only$     & $Only$ & & \\ \hline     
%      Convex Hull &  21:30:21 & 40 / 76 &  30 & 10 & 14 & 26\\ \hline 
%      Path &  17:21:57 & 48 / 76 &  33 & 15 & 17 & 31 \\ \hline 
%      Minimal Spanning Tree &  12:11:52 & 39 / 76 &  27 & 12 & 14 & 25 \\ \hline 
%      Overall &  51:04:10 & 127 / 228 &  90 & 37 & 45 & 82 \\ \hline 
% 	\end{tabular}
%    \\ \rule{0mm}{5mm}
% \end{center}
% \label{modelingTimes}
% \end{table}


\subsubsection{Mental Transformation Task Results}

Subjects were given two sets of 10 mental transformation problems, as discussed
previously in the procedure section. The first set was given before the first
modeling session, as a sort of pre-assessment. The second set was given after
the second modeling session as a post-test. We recorded performance data by
session and by user, and present the results in Figure \ref{mttbreakdown} broken
out by the type of symmetry represented in the shape (unilateral or bilateral)
and the type of translation or rotation performed on the shape (direct or
diagonal translation, direction or diagonal rotation), meaning that each shape
had both a symmetry type and a translation type.


\begin{figure}[!ht]
\begin{center}
\includegraphics[width=.9\linewidth]{images/mttShape}
\end{center}
\caption{A view of the Mental Transformation Task results, broken out by
symmetry type (B = bilateral, U = unilateral) and rotation or translation type
performed on the shape being transformed.}
\label{mttbreakdown}
\end{figure}

Overall, subjects performed very well on the Mental Transformation Task,
correctly responding to 614 of 720 questions (a little over 85\%). Performance
was remarkably equal across genders, with girls correct on 256 of 300 (85.3\%)
and boys on 358 of 420 (85.2\%). Accordingly, we found no sigificant difference
in gendered responses across any symmetry or translation type - girls and boys
succeeded and struggled on the same sorts of tasks. Bilateral symmetry was
significantly easier than unilateral, with over 90\% of bilateral tasks and only
78\% of unilateral tasks performed correctly. Rotation was more difficult than
translation, and diagonal transformations were more problematic than direct
ones. Hence, diagonal rotations scored the lowest (75\%), followed by direct
rotations (82\%), diagonal translations (91\%), and direct translations (93\%).

% \begin{table}[!ht] 
% \small
%     \caption[Mental Transformation Task by Shape Profile and
%     Translation Type]{Mental Transformation Task by Shape Profile and
%     Translation Type}
%     \begin{center}
%     \begin{tabular}{| c | c | c | c | c | c | c | c | } \hline 
%     $  $ & $Bilateral$ & $Unilateral$ & $Direct$ & $Direct$ & $Diagonal$ &
%     $Diagonal$ & $Total$ \\
%     & $Symmetry$ & $Symmetry$ & $Translation$ & $Rotation$ &
%     $Translation$ & $Rotation$ & $Correct$ \\ \hline
%     $Session 1$ & 109/120 & 57/80 & 54/60 & 34/40 & 56/60 & 22/40 & 166/200 \\
%     \hline 
%     $Girls$ & 43/48 & 23/32 & 22/24 & 13/16 & 23/24 & 8/16 & 66/80 \\ \hline
%     $ Boys $ & 66/72 & 34/48 & 32/36 & 21/24 & 33/36 & 14/24 & 100/120 \\ \hline
%     $ $ &  &  &  &  &  &  & \\ \hline
%     $Session 2$ & 72/80 & 69/80 & 31/32 & 39/48 & 28/32 & 43/48 & 141/160 \\
%     \hline 
%     $Girls$ & 32/35 & 30/35 & 13/14 & 18/21 & 12/14 & 19/21 & 62/70 \\ \hline
%     $Boys$ & 42/45 & 39/45 & 18/18 & 21/17 & 16/18 & 24/27 & 79/90 \\ \hline
%     $ $ &  &  &  &  &  &  & \\ \hline
%     $Combined$ & 181/200 & 126/160 & 85/92 & 73/88 & 84/92 & 65/88 & 307/360 \\
%     \hline 
%     $Girls$ & 75/83 & 53/67 & 35/38 & 31/37 & 35/38  & 27/37 & 256/300 \\ \hline
%     $Boys$ & 108/117 & 73/93 & 50/54 & 42/51 & 49/54 & 38/51 & 360/420 \\ \hline
% 	\end{tabular}
%    \\ \rule{0mm}{5mm}
% \end{center}
% \label{MTTbyShape}
% \end{table}

% \begin{table}[!ht] 
% \tiny
%     \caption[Mental Transformation Task
%     Performance Per User]{Mental Transformation Task Performance Per User.}
%     \begin{center}
%     \begin{tabular}{| c | c | c | c | c | c | c | c | c | c | c | c | c | c | c
%     | c | c | c | c | c | c |} \hline 
%     $User$ & 1 & 2 & 3 & 4 & 5 & 6 & 7 & 8 & 9 & 10 & 11 & 12 & 13 & 14 & 15 &
%     16 & 17 & 18 & 19 & 20 \\ \hline
%     $Set$ $1$ & 7 & 10 & 8 & 7 & 9 & 8 & 10 & 9 & 9 & 7 & 9 & 8 & 10 & 9 & 6 &
%     8 & 10 & 8 & 7 & 7 \\\hline 
%     $Set$ $2$ & 9 & 9 & 10 & 6 & 10 & - & 10 & 10 & 10 & 10 & 9 & 9 & - & 6 &
%     - & - & 10 & 8 & 5 & 10 \\ \hline 
%     $Total$ & 16 & 19 & 18  & 13  & 19 & - & 20  & 19 & 19  & 17 & 18  & 17  & - 
%     & 15 & - & - & 20 & 16 & 12 & 17\\ \hline 
%     $Change$ & +2 & -1 & +2 & -1 & +1 & - & 0 & +1 & +1 & +3 & 0 & +1 & - &
%     -3 & - & - & 0 & 0 & -2 & +3 \\ \hline
% 	\end{tabular}
%    \\ \rule{0mm}{5mm}
% \end{center}
% \label{MTTperUser}
% \end{table}

\begin{figure}[!ht]
\begin{center}
\includegraphics[width=.9\linewidth]{images/mttPerf}
\end{center}
\caption{Mental Transformation Task results, broken down by session and by
user.}
\label{MTTPerformance}
\end{figure}

Figure \ref{MTTPerformance} shows the Mental Transformation Task results broken
down into sessions by user. We observed a +7 net improvement in the second round
among the 16 users who participated in both sessions. Both girls and boys
improved in the second session, though girls improved by a greater percentage
when compared to boys - from 82.3\% to 88.6\% while boys improved from 83.3\% to
87.7\%, a 2\% greater improvement among girls. Four users did worse on the
second set of tasks, four did the same, and eight improved; the greatest change
in both directions was +/- 3. There was a no real correlation between
improvement between sessions (or lack thereof) and modeling performance overall
($r = .20, $ $p < .5$), nor was there a real correlation between improvement on
the Mental Transformation Task and improvement in modeling score from session 1
to session 2 ($r=-.20,$ $p <.5$), suggesting that the \emph{change} between
sessions on the spatial reasoning test and modeling performance are mildly
related, if at all.

\subsubsection{Speech and Gesture Coding Results}

During the modeling exercises, if a subject believed (correctly or not) that
they had successfully modeled a shape, the facilitator asked the subject to
describe the modeling strategy they used to arrive at their answer. During these
explanations, video recordings were analyzed for five types of speech and
gesture behaviors: those referring to movement, to the perceptual whole of the
shape being modeled, to a perceptual feature of the shape being model, as well
as behaviors that were vague or unintelligible, and those that did not fit into
any of the above categories (labeled as ``other'' - a more detailed description
is available in the procedure section above). A given strategy was only recorded
once per modeling task, but multiple strategies per explanation occurred often
and were recorded (as was also the case in \cite{ehrlich2006importance}). The
table below breaks down the numbers and types of speech and gesture observed
over the two sessions; as such, we only report on the 16 subjects who completed
both sessions. For further insight into how some of the modeling strategies were
expressed by the users as well as the associated coding and gesture
observations, we have compiled a set of excerpts in Appendix B. These excerpts
contain quotes from users while explaining their modeling strategy, the observed
gestures that occurred during the spoken explanation, and the speech and gesture
codes generated from those expressions.

% \begin{table}[!ht] 
% \tiny
%     \caption[Session 1 Gesture and Speech Observations]{Gesture and
%     Speech Observations over both sessions.}
%     \begin{center}
%     \begin{tabular}{| c | c | c | c | c | c | c | c | c | c | c | c | c | c | c
%     |} \hline
%     User & G.M & G.PW & G.PF & G.V & G.O & G Tot. & S.M & S.PW & S.PF & S.V &
%     S.O & S Tot. & G Tot. + & No. \\   
%     & & & & & & & & & & & & & S Tot. & Corr. \\ \hline
%     1 &4 &0 &8 &7 &0 &19 &4 &3 &7 &13 &4 &31 &50 &20 \\ \hline
%     2 &10 &1 &10 &6 &0 &27 &7 &7 &9 &6 &2 &31 &58 &14  \\ \hline
%     3 &1 &0 &9 &1 &0 &11 &4 &2 &11 &4 &0 &21 &32 &9  \\ \hline
%     4 &0 &0 &5 &9 &0 &14 &1 &7 &12 &9 &2 &31 &45 &9  \\ \hline
%     5 &6 &0 &11 &5 &0 &22 &8 &2 &11 &6 &4 &31 &53 &11  \\ \hline
%     7 &4 &0 &15 &6 &0 &25 &6 &3 &15 &2 &12 &38 &63 &23  \\ \hline
%     8 &1 &0 &2 &4 &0 &7 &1 &0 &2 &4 &0 &7 &14 &12  \\ \hline
%     9 &4 &2 &3 &10 &0 &19 &10 &6 &4 &8 &1 &29 &48 &11  \\ \hline
%     10 &10 &4 &19 &8 &2 &43 &5 &9 &19 &6 &4 &43 &86 &22  \\ \hline
%     11 &11 &2 &15 &5 &1 &34 &8 &3 &16 &5 &7 &39 &73 &21  \\ \hline
%     12 &6 &0 &10 &8 &0 &24 &5 &4 &7 &10 &3 &29 &53 &11  \\ \hline
%     14 &4 &1 &8 &7 &0 &20 &6 &3 &6 &7 &4 &26 &46 &6  \\ \hline
%     17 &16 &2 &23 &6 &0 &47 &14 &12 &22 &0 &7 &55 &102 &20  \\ \hline
%     18 &12 &1 &19 &2 &0 &34 &13 &3 &22 &2 &10 &50 &84 &20  \\ \hline
%     19 &17 &0 &14 &7 &2 &40 &13 &2 &14 &16 &3 &48 &88 &7  \\ \hline
%     20 &7 &0 &9 &9 &2 &27 &2 &2 &9 &6 &7 &26 &53 &5  \\ \hline
% 	\end{tabular}
%    \\ \rule{0mm}{5mm}
% \end{center}
% \label{MTTperUser}
% \end{table}


\begin{table}[!ht] 
\small
    \caption[Gesture and Speech Observations]{Gesture and Speech
    Observations over both sessions. Numbers in this table exclude the
    totals from the three subjects who finished the first session but not the
    second. \\ G = Gesture, S = Speech, .M = Movement, .PW = Perceptual Whole,
    .PF = Perceptual Feature, .V = Vague, .O = Other.}
    \begin{center}
    \begin{tabular}{| c | c | c | c | c | c | c | c | c | c | c |} \hline
	& $Total$ & & $PopCAD$ & $SnapCAD$ & & $Girls$ & $Boys$ & & $Session$ $1$ &
	$Session$ $2$\\ \hline 
	$G.M$ &113 & &62 &51 & &73 &40 & &39 &74 \\ \hline
	$G.PW$ &13 & &8 &5 & &8 &5 & &9 & 4 \\ \hline
	$G.PF$ &180 & &102 &78 & &96 &84& &93 &87 \\ \hline
	$G.V$ &100 & &50 &50 & &42 &58& &34 & 66 \\ \hline
	$G.O$ &7 & &4 &3 & &6 &1& &4 & 3\\ \hline
	 & & & & & & & & & &\\ \hline
	$S.M$ &107 & &64 &43 & &55 &52& &46 & 61\\ \hline
	$S.PW$ &68 & &39 &29 & &35 &33& & 38& 30 \\ \hline
	$S.PF$ &186 & &103 &83 & &97 &89 & &101 &85 \\ \hline
	$S.V$ &104 & &55 &49 & &40 &64 & &32 &72 \\ \hline
	$S.O$ &70 & &35 &35 & &33 &37 & & 18 & 52 \\ \hline
	 & & & & & & & & & & \\ \hline
	$Gesture$ &413 & &226 &187 & &225 &188& &179 & 234 \\ \hline
	$Speech$ &535 & &296 &239 & &260 &275 & & 235& 300\\ \hline
	$Combined$ &948 & &522 &426 & &485 &463& & 414& 534 \\ \hline
	\end{tabular}
   \\ \rule{0mm}{5mm}
\end{center}
\label{MTTperUser}
\end{table}

Table \ref{MTTperUser} shows the total number of gesture and speech types we
recorded, as well as how they were split between each devices, genders, and
sessions. The most common gesture and speech types (by a significant margin)
were about specific perceptual features of the models, those relating to
movement came next, followed closely by vague gestures and speech.
The other two categories, perceptual whole and ``other'' strategies, were barely
represented in gesture - they were far more common in speech, but still ranked
as the least frequently recorded. Many users explained their modeling strategy
by doing a ``step-by-step'' recounting of their process that referred at each
step to the part of the shape they were modeling at that point. For example, it
was common for a subject to point to a segment of the model and say (for
instance), ``and then I put a point here, for this part\ldots'', generating
perceptual feature scores in both gesture and speech for nearly every
explanation they gave. Movement was often explained along the same lines (though
less frequently), often with subject using specific words that indicate motion
(e.g. ``then I move over here'', ``I had to go up here, then follow the path
back down again'') while simultaneously motioning along the directions they were
indicating. Figure \ref{gestureSeries} shows a series of six still shots taken
from the video that depict (as best as possible in a single frame) various
gestures made during a single explanation of a modeling task strategy.

\begin{figure}[!ht]
\begin{center}$
\begin{array}{ccc}
\includegraphics[width=.3\linewidth]{images/g1}&
\includegraphics[width=.3\linewidth]{images/g2}&
\includegraphics[width=.3\linewidth]{images/g3}\\
\includegraphics[width=.3\linewidth]{images/g4}&
\includegraphics[width=.3\linewidth]{images/g5}&
\includegraphics[width=.3\linewidth]{images/g6} \\
\end{array}$
\end{center}
\caption{A series of screen grabs from the video recording showing various
gestures from a user explaining her modeling strategy on one of the modeling
tasks.}
\label{gestureSeries}
\end{figure}


Interestingly, even without accounting for the difference in number of subjects,
girls ``out-gestured'' the boys overall (225 to 188), and in every category
\emph{except} for vague gestures, where boys were vague in describing their
strategies 24 more times over the course of the study. Speech types were more
gender-balanced, with the final tally being 260 for girls and 275 for boys,
however seeing as boys had more participants in both sessions of the study, the
speech-per-participant count actually favors the girls as well. The PopCAD
interface produced more gestures (226 to 187) and speech (296 to 239) than the
SnapCAD, a finding mitigated somewhat by the fact that users modeled so poorly
on the SnapCAD in the first round and therefore did not arrive at a point where
a modeling strategy could be explained. If we isolate the second round only,
where the performance breakdown was much more even (62 to 54 in favor of
PopCAD), then SnapCAD actually produced more gestures (124 to 110) and more
speech elements (159 to 141). 

Perhaps the most curious data from Table \ref{MTTperUser} is the big increase in
both gesture and speech from round one to round two of the study. Even with
three less participants in round two, overall instances of gestures increased
from round one by 55 (179 to 234, a 31\% increase), and speech instances
increased by 65 (235 to 300, a 28\% increase), yet the overall modeling
performance only increased by 5\% in round two. A bit of a closer look at the
types of gesture and speech gives a plausible explanation: in both gesture and
speech, the number of \emph{vague} indications rose dramatically (+32 for
gesture, +40 for speech), while the number of perceptual feature indications
dropped in both cases (-6 for gesture, -16 for speech). If we look at Figures
\ref{gesturebreakdown} and \ref{speechbreakdown} these numbers start to make
more sense.

\begin{figure}[!ht]
\begin{center}
\includegraphics[width=.85\linewidth]{images/gestureGraph}
\end{center}
\caption{A plot of the five types of gestures we coded (movement, perceptual
whole, perceptual feature, vague, and other) over the number of correctly
modeled shapes. The slope of the lines indicate the strength of correlation
between each gesture type and overall modeling performance.}
\label{gesturebreakdown}
\end{figure}

Figure \ref{gesturebreakdown} shows a plot of the number and kind of gestures
produced by a user over the number of shapes they modeled correctly over the two
rounds of the study.\footnote{Data from the three users who dropped out of the
study has been omitted from this graph as well as Figure \ref{speechbreakdown}}
The lines associated with each scatter plot shows the strength of the
correlation between instances of that gesture type and modeling performance; the
steeper the positive slope, the higher the positive correlation and vice versa.
As we can see from the graph, three of the conditions have positive slopes
(perceptual feature, movement, and perceptual whole), while two have negative
slopes (vague and other). By far the strongest positive correlation\footnote{All
correlation calculations were done using Pearson's Correlation Coeffcient.} is
between perceptual feature gesturing and modeling performance ($r = .61,$ $p <
.025$), while vague gesturing has a weak negative correlation ($r = -.22,$ $p <
.5$). Going back to our earlier table, then, the sharp uptick in vague gestures
and mild decline of perceptual features may help to explain why such an increase
in gesturing did not result in a similar upswing in modeling performance.


\begin{figure}[!ht]
\begin{center}
\includegraphics[width=.85\linewidth]{images/speechGraph}
\end{center}
\caption{A plot of the five types of speech we coded (movement, perceptual
whole, perceptual feature, vague, and other) over the number of correctly
modeled shapes. The slope of the lines indicate the strength of correlation
between each speech type and overall modeling performance.}
\label{speechbreakdown}
\end{figure}

One might expect that correlation patterns would be similar between gestures and
speech of the same type (e.g. instances of movement in gesture would be as
correlated to modeling performance as instances of movement in speech), and
while we did find some similarities, some surprising differences appeared as
well. Speaking about perceptual features was (as with gesturing) the most highly
correlated type to modeling success ($r = .58,$ $p < .025$), but where gestures
marked as ``other'' had a very weak negative correlation, ``other'' categories
of speech were second most \emph{highly} correlated with modeling aptitude ($r
=. 56,$ $p < .025$) - nearly as much as utterances on perceptual features. Part
of this explanation lies in the frequency discrepancy between ``other'' gestures,
of which there were only seven, and ``other'' speech utterances, of which there
were ten times more (70). The other (pardon the pun) part of the explanation
lies in the fact that we have many more words with specific meanings than
gestures that are precisely defined, so (for example) explanations referring to
looking at the software itself (e.g., ``I looked at the screen and it looked
like it''), or reasoning about the nature of how the mode works (e.g., ``Since
it was path I knew it would work''), or internal operations (e.g., ``I just
look at it and see it''), are harder to perceive in gesture. A possible relation
to the potential for specificity in speech lies in the stronger observed
negative correlation between modeling ability and speech marked as vague ($r=
-.41,$ $p < .25$), compared to gestures marked as vague ($r=-.22,$ $p < .5$),
indicating (perhaps) that a failure to speak specifically (given more abundant
options) is more harmful than a similar failure when gesturing.


\subsection{Observations}

A few notes on the above findings are worth making here. Broadly speaking, the
study indicated many positive outcomes: overall modeling ability went up while
average modeling time went down, the participants improved on every modeling
mode in the second session, there was a net positive performance on the second
mental transformation task when compared with the first, and participants were
generally engaged by the experience, which for most subjects was their first
computer-based 3D modeling experience. However, even though no user saw the same
device or shape twice, it is as yet unclear how much of the improvement might be
contributed to a ``practice effect''. Due to the ``drop-in'' nature of the user
study environment, the time between each single participants' sessions varied,
based on their attendance and availability (i.e., in some cases users had
homework or other activities to finish).

We observed some moderate correlations between types of speech and gesture and
modeling success, though not necessarily the kinds of correlations we might have
expected based on prior related studies. Nor were speech and gesture correlated
to modeling acumen in the same ways - some types of gesture were less effective
than their corresponding spoken elements, and vice versa. In some cases, the
results were observed were counter-intuitive - such as the anomaly in average
modeling times of subjects when using the minimal spanning tree mode, the fact
that boys performed worse during the second modeling session while girls
performed much better, and that some subjects performed worse on the second
mental transformation task, even though they were arguably ``primed'' by going
through the modeling exercises beforehand. Also unexpected is the sharp decline
in performance on the PopCAD device in the second session - over 10\% -
especially after such a high percentage in the first round and given more
``experienced'' users in the second session. Equally surprising, given a rather
unimpressive first round performance, is the sharp increase in modeling success
on the SnapCAD in the second round (a jump of 12\%), so much so that when
coupled with the decline in PopCAD performance, we may wonder on the possible
disparity between the groups in ``inherent'' ability for these kinds of tasks.
Another possibility is of course that the order in which subject encounter the
devices is more important than we had originally surmised - perhaps the users
who started with PopCAD did better on the SnapCAD (and overall) \emph{because}
they started with PopCAD. We examine these, as well as the relevance of age and
shape complexity on modeling ability, along with a deeper discussion of results
across all three studies in the following chapter.






% Given the massive development (both cognitively and physically) that occurs
% between the age extremes in our subject population (11 to 18), it would be
% tempting (and even logical) to assume that the older subjects would perform much
% better on the modeling tasks than their younger counterparts. However, we found
% a very modest correlation ($r = .39, p < .15$) between age and the number of
% correctly modeled shapes, suggesting it may play less of a role then we would
% have suspected. It is possible that the statistics are slightly misleading here
% - the subject population was weighted toward the younger end of the spectrum:
% the average age was 13.8, while median age was 13.5, and the mode was 12 years
% old. Meaning the few older participants would have had to perform impossibly
% brilliantly (i.e., higher than the highest possible score) for a strong age to
% performance correlation to show up.
% 
% Popcad first avg age: 14
% snapcad first : 13.75
% 
% 
% In order to attempt to judge each shape's complexity, we sought out a
% previously-defined set of criteria by which to judge ``complexity''. As it turns
% out, there is a long and thorough discussion of complexity in relation to
% \emph{two-dimensional} shapes, starting seemingly with Fred Attneave and Malcolm
% Arnoult\cite{attneave1956quantitative}\cite{attneave1957physical} in the mid
% 1950's, who define methods of generating random two-dimensional shapes and
% examine their physical characteristics in relation to their judged complexity.
% As it turns out (in \cite{attneave1957physical} as well as others' follow-up
% work) the ``Number of Turns'' in the shape was responsible for significant
% amount (nearly 80\% in Attneave's study) of the preceived complexity of a shape.
% ``Number of Turns'' is defined as, ``the number of maxima (regardless of sign)
% in one cycle of the function relating curvature to distance along the contour.
% This function is a series of spikes for any angular shape, and a step-function
% for any curved shape\ldots'' (see pp. 226 of the aforementioned article).
% Symmetry, angular variability, and squared perimeter over area also had some
% affect.
% 
% However, as it seems unclear to us how one might adapt a ``Number of Turns''
% rating to a true \emph{three dimensional} model. Although many studies claim to
% have studied complexity in relation to mental transformation tasks, starting
% with Shepard and Metzler\cite{shepard1971mental}, who instead used perspective
% line drawings, not actual physical models. This had an advantage for the types
% of mental rotation tasks they were performing (recognition of matching pairs),
% and similarly set off a wave of studies using the same (or similar) ``faux 3D''
% stimuli\cite{metzler1974transformational}\cite{shepard1988mental}\cite{vandenberg1978mental}.
% 
% All of which leads us to determine (as best we way) the complexity of the shapes
% we presented in the study, as a way of teasing out any correlation between
% complexity and performance. In lieu of attempting an exact number of turns
% estimate, we included three criteria: (a) the minimum number of lights necessary
% to guarantee the correct shape,\footnote{In minimal spanning tree models where a
% placement of lights results in several possible correct formations, only one of
% which is the desired shape, we add points necessary to ``force'' the correct
% representation.} (b) the number of faces (for convex hull models only), the
% number of line segments (for path models only), or the number of distinct
% branches (for tree models only), and (c) a symmetry score based on number of
% lines of symmetry, from 3 (indicating asymmetry) to 0 (indicating 3 or more
% lines of symmetry). The scaling for symmetry comes from the belief that
% indicators (a) and (b) above are more closely aligned with Attneave's ``number
% of turns'' metric (being highly correlated to perceived difficulty), while
% symmetry was much less correlated to complexity (although symmetry did still
% play a part), so we made the scale as low as possible so that it would weigh
% less on the overall complexity score of a model. So, for example, a regular
% octahedron would have 6 points, 8 faces, and a 0 symmetry score for an overall
% difficulty score of 14. The complexity score of each shape is shown in
% \ref{modelComplexity} next to the number of times it was modeled correctly. The
% shapes in each session were of course different, but are labeled the same in
% this table, indicating the order in which they were presented.
% 
% 
% 
% \begin{table}[!ht] 
% \small
%     \caption[Complexity of Models and Modeling Performance]{Complexity of Models
%     and Modeling Performance (CH = Convex Hull, P = Path, T = Minimal Spanning
%     Tree)}
%     \begin{center}
%     \begin{tabular}{| c | c | c | c | c | c | c | c | c | c | c | c | c | }
%     \hline $ $ & CH1 & CH2 & CH3 & CH4 & P1 & P2 & P3 & P4 & T1 & T2 & T3 & T4 \\
%    	\hline
%    	$Session$ $1$ & & & & & & & & & & & &  \\ \hline
%    	$Complexity$ $Score$ & 14 & 19 & 19 & 19 & 7 & 18 & 20 & 24 & 9 & 13 & 17 &
%    	17 \\ \hline 
%    	$Performance$ of 19 & 8 & 13 & 8 & 11 & 17 & 14 & 8 & 9 & 12 & 8 & 8 & 11 \\
%    	\hline 
%    	$Session$ $2$ & & & & & & & & & & & & \\ \hline
%    	$Complexity$ $Score$ & 12 & 20 & 15 & 13 & 14 & 14 & 18 & 17 & 21 & 17 & 16
%    	& 25 \\ \hline 
%    	$Performance$ of 16 & 7 & 9 & 11 & 11 & 11 & 13 & 8 & 12 & 7 & 11 & 7 & 9 \\
%    	\hline
%    	
% 	\end{tabular}
%    \\ \rule{0mm}{5mm}
% \end{center}
% \label{modelComplexity}
% \end{table}
% 
% 
% One might expect to see a strong negative correlation between a given model's
% complexity score and the number of subjects who were able to model it correctly,
% however the observed correlation (using Pearson's correlation coefficient) was
% only moderate: $r = -0.41, p < .05$ over both sessions.
% 
% 
% Levine and Goldin-Meadow's research found the highest correlation occurred
% between the number of movement gestures produced over the number of correct
% answers given. 
% 
% 
% 
% The process of modeling is different than that of shape matching - modeling is
% by necessity a series of step-by-step, piecemeal operations, whereby in a mental
% transformation task, it is possible both to look at specific features of a shape
% as well as receive a more holistic mental image of the shape at hand. This may
% account in part for the relative lack of correlation between movement gestures
% and performance when compared to \cite{ehrlich2006importance}. Instead, we found
% a much higher correlation between gestures relating to perceptual features of
% the shape and modeling performance. One might argue that by looking at
% perceptual features of the shape, one is mentally ``breaking apart'' the shape
% into discrete chunks that can be turned into an order of operations for
% successfully modeling a shape.

\chapter{Discussion}
\label{discussion}

We devote this chapter to a deeper look at some of factors that may (or may not)
shine a light on a few of the more perplexing and intriguing observations made
through the user studies mentioned in the previous chapter, as well as a
meta-review of what, if anything, we may conclude based on our reported data and
observations. We spend the most attention on our latest study (with the PopCAD
and SnapCAD) as it is not only the most recent, but the most significant, as it
concerns not only whether or not our devices can be used effectively for
modeling by novice users, but touches on the relationships between our devices,
spatial reasoning, and embodied cognition. We start with the significance of the
gesture and speech observations made in the PopCAD and SnapCAD study, discuss
the role of age in relation to modeling performance, followed by an analysis of
the effects of shape complexity on modeling acuity, and finally some
meta-analysis of the observed data over the three user studies described in the
last chapter.


\section{Gesture and Speech Significance}

The work of Ehrlich\cite{ehrlich2006importance}, Levine\cite{levine1999early},
and Goldin-Meadow especially\cite{goldin}\cite{goldin2005hearing}, serves as a
rough guide to our most recent study design, as they touch on the role of
gesture in determining spatial reasoning performance, and later provide strong
evidence that gesture is a valuable window into the mind, all of which supports
the notions inherent in embodied cognition - that body and mind are far more
tightly linked than we have traditionally be led to believe. As we operate under
these assumptions as reasoning to create tangible, physically involved
interfaces (as opposed to pure 2D software) it is worth taking a deeper look
into how our study results compare and contrast with this earlier work.

Ehrlich and Levine's studies focus on the gestures and speech produced during
children's explanations of how they solved a series of mental transformation
tasks. The participants were presented with the same sorts of instruments used
in our PopCAD and SnapCAD study, although our studies differ significantly in
several ways. Of course, Ehrlich and Levine had no devices, and evaluated the
speech and gestures produced in explaining the mental transformation task,
whereas we examined the strategies expressed when modeling on the PopCAD and
SnapCAD. Apart from one practice condition where wooden blocks were used (which
in their study had no effect) all of the tasks in \cite{ehrlich2006importance}
were based on 2D paper representations, and the subject had no physical contact
with any of the objects they were trying to model. In our study, subjects were
handed a 3D-printed (and thus 3D) model of the shape they were attempting to
reproduce. Additionally, in Ehrlich and Levine's studies, subjects were
instructed to (in their mind) ``move the pieces together'' or to ``observe the
movement'' of the pieces as manipulated by the experimenter. These factors, as
well as differences in age (our subject population was 5-13 years older than
those in \cite{levine1999early} and \cite{ehrlich2006importance}) likely
contribute to (at least of some of) the differences observed in our study. We
spend the next two subsections dissecting the similarities and contrasts in our
results compared to the finding reported by Ehrlich, Goldin-Meadow, and Levine.


\subsection{Contrasts}

Given the differences in the nature of our study compared with Ehrlich, that our
finding should differ should come as no surprise. While both of the gesture and
speech analyses occurred while subjects were explaining a modeling strategy, the
type of modeling activity they had been asked to perform was substantively
dissimilar. As noted above, subjects in our study were handed physical, 3D
models of the target shape they were tasked with modeling, and could hold on to
that object (and rotate it, look at it from different angles, hold it in front
of the device or the computer screen, etc.) during their modeling process.
Afterward, when asked about their strategy, they still had that object, and
often gestured to it (or with it) and (of course) talked about it. No such
``hands-on'' activity was involved in the studies we reference above, nor (as
we note later in the chapter) could we find substantive work involving such
manipulative activities to examine spatial reasoning ability. 

We contend that the ``embodied'' nature of the tasks in our study help explain
some of the differences in gesture and speech patterns and correlations that we
observed. In fact, it is likely that given Goldin-Meadow's body of work and
studies involving cognition and gesture, she would concur with us. Furthermore,
the way in which the examiner in the above studies introduces the tasks to the
subjects involves several direct references to movement (e.g. ``In your mind,
move the pieces together and then move them back apart.''). This is significant,
as gestures and speech relating to movement was (in their study) both the most
frequent type of strategy expressed, but, as far as gesture correlations with
task performance is concerned, gestures coded as relating to movement was the
only type of response they recorded that was exclusively related to answering
the test questions correctly. To take an excerpt from
\cite{ehrlich2006importance} (pp. 1265):
\begin{quote}
``Gesturing about moving the pieces was
correlated with the number of problems answered correctly ($r = .461, p <
.001$), but it was not correlated with the number of problems answered
incorrectly ($r = .202, ns$). Thus, gesturing about moving the pieces together
was uniquely related to correct performance, whereas talking about moving the
pieces was not.''
\end{quote}  

To summarize our related findings, then: in our study, gestures about movement
were the second most observed expression, after those related to perceptual
features (113 to 180); speech about movement was also second to perceptual
features in frequency (186 to 107), and neither gesture nor speech was
significantly tied to performance in our modeling tasks ($r = .29, p < .29$ for
gesture, $r= .27, p < .32$). Instead, we found the highest correlation (and
highest frequency) in speech and gestures relating to perceptual features ($r =
.61, p < .025$ for gesture, $r = .58, p < .025$ for speech). Ehrlich instead
found only a negative correlation between modeling performance and perceptual
feature coding, both from speech and from gesture.

We are then left to wonder about the rather drastic differences in our findings.
What might account for both the frequency and correlation differences in
movement versus perceptual feature strategies? Although of course we cannot know
for certain, we hinted at some of the possibilities above: the ``embodied''
nature of our tasks, having the subjects hold onto an object representative of
the solution they were striving for, the participants being' ``primed'' for
certain kinds of responses in the earlier studies, and the differences in age
all may account for some of the differences. The kind of mental processes
involved in a mental transformation task are not all that different
(necessarily) from those involved in modeling an object with the PopCAD (for
example) - a robust mental image of the shape in question is likely a boon in
either case. However, a model can be built step-by-step, and the results
observed and reflected upon. When picking a correct shape in a mental
transformation task, one may mentally operate upon features of the shape in a
step-by-step manner, but there is no opportunity to reflect upon various
strategies, a holistic decision has to be made. In a step-by-step modeling
process, it seems common (from the data and from our own experiences and
intuitions) for a modeling ``step'' to focus on a perceptual feature of the
shape being modeled (e.g. the next segment in a path or the top point in a
pyramid-shaped hull), and to do so in a very conscious way. Additionally,
subjects in our study were allowed to continue holding the model while they gave
their strategy, providing a ready ``facsimile'' on which to project their
modeling intentions. These factors may have ``paved the way'' for a high number
of speech and gesture about perceptual features of the models, as the step-wise
nature of the task and the physical surroundings lead themselves toward thinking
in terms of the characteristics of the shapes. Although we observed a high
number of expressions coded as movement, there is nothing inherently
``movement-oriented'' in 3D modeling (one's body moves in using our devices, of
course, but models can be created in other ways, e.g. strictly from text
coordinates). In contrast, a mental transformation task is, explicitly asking
the user to ``move'' the object, mentally, into the correct formation. Thus it
is unsurprising that the examiners in Ehrlich's study repeatedly used the word
``move'' and appealed to references about movement (as noted above). It is
equally unsurprising then, that by using this sort of language and then looking
at ``movement'' as a gestural and spoken strategy, many instances were found, as
the task and the instructions surrounding the task are both ``movement
oriented'' in a way that the modeling tasks in our study were not. 

One of the other major findings in \cite{ehrlich2006importance} and
\cite{levine1999early} is that significant performance differences exist between
genders on these tasks, and are evident at younger ages than previously thought.
Existing research at the time claimed that gendered differences in spatial
reasoning developed around puberty, but that several studies had challenges this
assumption. In either case, gender differences should have shown up in our study
based on the age range of our subjects (11-18). Boys did outperform girls in
session one of the modeling task,  though it was not by a terribly significant
portion (4\%), they did model faster than the girls in each round of the
modeling exercise (by about 7 minutes total in the first session, 5 minutes
overall in the second session), and they produced more speech instances than the
girls did overall (275 to 260), although since there were more boys in the
study, this advantage is negligible at best. Interestingly, our findings had
girls performing better in many areas; they outperformed boys in the second
modeling session by 9\%, and across both sessions by almost 3\%. Despite a
disadvantage in numbers, girls produced more gestures (225 to 188) including
those most closely linked to modeling success, perceptual features (96 to 84).
Girls also produced more speech elements about perceptual features (97 to 89).
The results from our mental transformation task have boys and girls performing
about equally, with girls edging out the boys by one tenth of a percent (85.3\%
to 85.2\%, respectively).

Did we have an exceptionally bright group of girls? Possibly, though no
independent tests were done for intelligence or other factors that would have
indicated an advantage - remember, the girls who enrolled in the study averaged
almost a full year younger than the boys (13 years, 7 months for girls and 14
years 6 months for boys), so age and experience advantages are unlikely (none of
the girls reported any previous 3D modeling experience). Although the nature of
the modeling exercises in our study were more piecemeal, possibly allowing girls
more of a chance to reflect and correct their mistakes than in the mental
transformation tasks\footnote{There is some evidence, relayed in
\cite{ehrlich2006importance}, that girls tend to utilize a step-by-step strategy
in mental rotation problems, whereas boys tend to deal with the whole shape at
once.}, we saw no significant difference in the MTT tasks we administered (in
fact girls did slightly better).

One possibility is that modeling with the sorts of devices we created are
somehow more beneficial to girls than to boys; that the spatial reasoning
advantages that boys have are either negated, or that the types of modeling
exercises we did significantly altered boys' normal spatial reasoning
strategies, which has been known to have a detrimental effect on
performance\cite{beilock2002paying}\cite{lutz2001procedural}. Plenty of other
possibilities exist (e.g., the girls simply tried harder) and there is no clear
way of determining the source(s) for our results, so we hesitate to make any
claims. However, we find it encouraging that girls were able to perform (even
out perform) when compared to the boys in our study.

\subsection{Commonalities}

Despite the differences mentioned in the previous section, some of our
observations did agree (or at least failed to disagree) with the previous
studies. In Ehrlich's study as well as ours, the study population improved
overall. In each case, girls improved by a markedly greater percentage, whereas
boys improved less so, and in some cases performance actually decreased (in our
second session overall and in the post-test for Ehrlich's ``Imagine Movement''
condition). Although flip-flopped in order, movement and perceptual feature
strategies were the most common, with perceptual whole instances far behind.
Generally speaking, gesture expressions deemed most ``task-appropriate'' (per
our discussion in the previous section) served as the highest observed
correlation to modeling success; perceptual feature gesturing in our study,
movement gesturing in Ehrlich's study. This is, we believe, the main ``gist'' of
both these experiments as far as gesture analysis goes - that gesture of a
strategy appropriate to the task at hand is correlated with success on that
task, more so than speech alone. This holds with the core of Goldin-Meadow's
findings, that gesture is a window into the cognitive process and that by
analyzing gesture we can gain insight into the mind of the learner.


% Girls improved, boys did not
% 
% Mis matchers?


\section{Age}

One of the more profound and noticeable results from the PopCAD and SnapCAD
study was the difference in modeling success between the devices in the first
round, and how much that difference was erased on the second round. In the first
session, users modeling with the PopCAD correctly modeled 75\% of the given
shapes, the highest percentage of any device in any round. Conversely, those
starting on the SnapCAD modeled only 34\% of their shapes properly. Given just
this data, we might be tempted to conclude that the PopCAD is a much easier
introductory device - or that the SnapCAD is insufferably difficult. However,
when we factor in the second round data, in which each subject modeled on the
device they did not use the first time around, a different picture emerges:
PopCAD modelers in the second round modeled 65\% of the shapes correctly while
SnapCAD modelers achieved a 56\% success rate. If we track each group (let us
call the first round PopCAD modelers group A, and the first round SnapCAD
modelers group B), we would be sorely tempted to declare that the groups
themselves are unevenly talented: Group A scored 75\% on PopCAD and 56\% on
SnapCAD, while group B scored 65\% on PopCAD and 34\% on SnapCAD. 

Since we did not perform a intelligence test or any sort of generalizable
aptitude test, we are left to guess using other means. Given the massive
development (both cognitively and physically) that occurs between the age
extremes in our subject population (11 to 18), it would be tempting (and even
logical) to assume that the older subjects would perform much better on the
modeling tasks than their younger counterparts. As it turns out, the average age
of group A was higher than that of group B - but only by 3 months (group A
average age was 14, group B was 13.75). We found a very modest correlation ($r =
.39, p < .15$) between age and the number of correctly modeled shapes,
suggesting that while not to be overlooked, it may play less of a role then we
would have suspected. It is also possible that the statistics are slightly
misleading here - the subject population was weighted toward the younger end of
the spectrum: the average age was 13.8, while median age was 13.5, and the mode
was 12 years old. Meaning the few older participants would have had to perform
impossibly brilliantly (i.e., higher than the highest possible score) for a
strong age to performance correlation to show up. In keeping with these
findings, we also found no real correlation between age and overall performance
on the mental transformation tasks ($r = .23, p < .45$). This data of course
does not discount that age plays a factor, nor that group A may have been more
talented than group B in the PopCAD/SnapCAD; simply that within rough
parameters, age matters, just not as much as one might think. Take for example
our oldest participant, an 18 year-old boy. He correctly performed 11 of the 24
modeling tasks, while the four 12 year-old participants scored 14, 11, 12, and
11. Our youngest participant, and 11 year old, reproduced seven shapes
correctly, while a 13 year old did five correctly, and a 14 year old got six
right.


\section{Shape Complexity}

In order to attempt to judge each shape's complexity, we sought out a
previously-defined set of criteria by which to judge ``complexity''. As it turns
out, there is a long and thorough discussion of complexity in relation to
\emph{two-dimensional} shapes, starting seemingly with Fred Attneave and Malcolm
Arnoult\cite{attneave1956quantitative}\cite{attneave1957physical} in the mid
1950's, who define methods of generating random two-dimensional shapes and
examine their physical characteristics in relation to their judged complexity.
As it turns out (in \cite{attneave1957physical} as well as others' follow-up
work) the ``Number of Turns'' in the shape was responsible for significant
amount (nearly 80\% in Attneave's study) of the preceived complexity of a shape.
``Number of Turns'' is defined as, ``the number of maxima (regardless of sign)
in one cycle of the function relating curvature to distance along the contour.
This function is a series of spikes for any angular shape, and a step-function
for any curved shape\ldots'' (see pp. 226 of the aforementioned article).
Symmetry, angular variability, and squared perimeter over area also had some
affect.

However, as it seems unclear to us how one might adapt a ``Number of Turns''
rating to a true \emph{three dimensional} model. Although many studies claim to
have studied complexity in relation to mental transformation tasks, starting
with Shepard and Metzler\cite{shepard1971mental}, who instead used perspective
line drawings, not actual physical models. This had an advantage for the types
of mental rotation tasks they were performing (recognition of matching pairs),
and similarly set off a wave of studies using the same (or similar) ``faux 3D''
stimuli\cite{metzler1974transformational}\cite{shepard1988mental}\cite{vandenberg1978mental}.

All of which leads us to determine (as best we way) the complexity of the shapes
we presented in the study, as a way of teasing out any correlation between
complexity and performance. In lieu of attempting an exact number of turns
estimate, we included three criteria: (a) the minimum number of lights necessary
to guarantee the correct shape,\footnote{In minimal spanning tree models where a
placement of lights results in several possible correct formations, only one of
which is the desired shape, we add points necessary to ``force'' the correct
representation.} (b) the number of faces (for convex hull models only), the
number of line segments (for path models only), or the number of distinct
branches (for tree models only), and (c) a symmetry score based on number of
lines of symmetry, from 3 (indicating asymmetry) to 0 (indicating 3 or more
lines of symmetry). The scaling for symmetry comes from the belief that
indicators (a) and (b) above are more closely aligned with Attneave's ``number
of turns'' metric (being highly correlated to perceived difficulty), while
symmetry was much less correlated to complexity (although symmetry did still
play a part), so we made the scale as low as possible so that it would weigh
less on the overall complexity score of a model. So, for example, a regular
octahedron would have 6 points, 8 faces, and a 0 symmetry score for an overall
difficulty score of 14. The complexity score of each shape is shown in
\ref{modelComplexity} next to the number of times it was modeled correctly. The
shapes in each session were of course different, but are labeled the same in
this table, indicating the order in which they were presented.



\begin{table}[!ht] 
\small
    \caption[Complexity of Models and Modeling Performance]{Complexity of Models
    and Modeling Performance (CH = Convex Hull, P = Path, T = Minimal Spanning
    Tree)}
    \begin{center}
    \begin{tabular}{| c | c | c | c | c | c | c | c | c | c | c | c | c | }
    \hline $ $ & CH1 & CH2 & CH3 & CH4 & P1 & P2 & P3 & P4 & T1 & T2 & T3 & T4 \\
   	\hline
   	$Session$ $1$ & & & & & & & & & & & &  \\ \hline
   	$Complexity$ $Score$ & 14 & 19 & 19 & 19 & 7 & 18 & 20 & 24 & 9 & 13 & 17 &
   	17 \\ \hline 
   	$Performance$ of 19 & 8 & 13 & 8 & 11 & 17 & 14 & 8 & 9 & 12 & 8 & 8 & 11 \\
   	\hline 
   	$Session$ $2$ & & & & & & & & & & & & \\ \hline
   	$Complexity$ $Score$ & 12 & 20 & 15 & 13 & 14 & 14 & 18 & 17 & 21 & 17 & 16
   	& 25 \\ \hline 
   	$Performance$ of 16 & 7 & 9 & 11 & 11 & 11 & 13 & 8 & 12 & 7 & 11 & 7 & 9 \\
   	\hline
   	
	\end{tabular}
   \\ \rule{0mm}{5mm}
\end{center}
\label{modelComplexity}
\end{table}


One might expect to see a strong negative correlation between a given model's
complexity score and the number of subjects who were able to model it correctly,
however the observed correlation was only moderate: $r = -0.41, p < .05$ over
both sessions.


\section{Error Analysis}

For each modeling task, one of seven error codes was recorded, based on the
outcome of the task. A more complete detail of the error codes can be found in
table \ref{modelingError}, this section focuses instead on what significance
(if any) these error codes have on our observations\footnote{Again, this data
is from the PopCAD/SnapCAD study only.}. To briefly recount the codes and their
associations, then: C = correct, EP = error in proportion (general shape is
correct, but model is too tall, too wide, etc.), E1 = error in recognition
(subject had the correct shape but did not recognize it), E2 = error in belief
(thought the shape was correct when it was not), E3 = error in implementation
(knew shape was incorrect, but knew why), E4 = error in strategy (subject knew
shape was incorrect, but could not explain why), I = incomplete (includes
giving up, asking to move on).

\begin{table}[!ht] 
\small
    \caption[Modeling Error Code Breakdown]{Error Code Breakdown. C = Correct,
    EP = Error in Proportion, E1 = Error in Recognition, E2 = Error in Belief,
    E3 = Error in Implementation, E4 = Error in Strategy, I = Incomplete.}
    \begin{center}
    \begin{tabular}{| c | c | c | c | c | c | }
    \hline $ $ & $Total$ & $PopCAD$ & $SnapCad$ & $Girls$ & $Boys$ \\ \hline
	$C$ & 221 &	98 & 123 & 100 & 121 \\ \hline
	$EP$ & 42 &	17 & 25 & 20 & 22 \\ \hline
	$E1$ & 1 &	0 &	1 &	1 &	0 \\ \hline
	$E2$ & 41 &	23 & 18 & 13 & 28 \\ \hline
	$E3$ & 2&	0&	2&	0&	2 \\ \hline
	$E4$ & 15&	11&	4&	6&	9 \\ \hline
	$I$ & 62&	40&	22&	28&	34 \\ \hline
	\end{tabular}
   \\ \rule{0mm}{5mm}
\end{center}
\label{modelErrors}
\end{table}



\section{Cross-Study Comparisons}
When we compare across studies, we see the first study resulted in 80\% (24 of
30) correct models, the second UCube study resulted in 82\% (41 of 50), while
the third study resulted in only 243/420, or about 58\%.

Complexity score


\section{Demographics}

An important factor to take into account here is the subject demographics. Only
two participants in the PopCAD/SnapCAD study had indicated previous experience
with 3D modeling in any capacity, and the study site was a drop-in program
serving primarily disadvantaged youth in a low socioeconomic neighborhood. In
contrast, the two earlier studies with the UCube were performed at a fairly
affluent, predominantly Caucasian middle school, with their multimedia class,
most of whom had been exposed to 3D modeling software (like Goggle
SketchUp\cite{SketchUp}) as part of the multimedia curriculum. So while we saw
performance in the 80\% range for the middle school and less than 60\% for our
drop-in program, not all of the difference is likely from modeling ability or
shape complexity or the ease of use of the UCube.

\chapter{Vision}
\label{vision}


Given the different medium of the pop-up book (paper as opposed to circuit
boards), it is worth exploring the possibilities afforded by a cheaper, more
flexible material. For instance, the flexibility of paper might provide the
means for new types of modeling actions. It is plausible to imagine paper tabs
or other mechanisms that perturb the LEDs off the integer lattice, or alter the
overall topology in such a way that new shapes are possible (e.g. by deforming
an equidistant grid into a spherical shape). There may be additional sensors or
hardware that could be embedded into the book to provide new functionality
(rotation, proximity, pressure). Additionally, due the inexpensive and portable
nature of the pop-up book, it is worth exploring the sorts of interactions that
could occur between several pop-up books (e.g., extending the input field to
include two or more grids, networked interactions like cooperative modeling
tasks, or competitive games like 3D-battleship). By using paper as a material to
think with, we may find further possibilities as development continues.
%\chapter{Conclusions}
\label{conclusions}

During the four years since beginning work on the first UCube prototype, our
thoughts and beliefs about the nature of our work evolved, became steadily more
focused and resulted in the work we present here on devices for embodied
fabrication. Over the course of three devices and three user studies, we gained
valuable insight into how young people think about 3D modeling, how they
interact with never-before-seen tangible devices, the kinds of models they
create, and the strategies they use (both verbal and gestural) to explain their
modeling decisions. This chapter is devoted to ``taking a look back'' at the
body of work that comprises this thesis and attempting to distill the most
relevant and unique findings into their most basic parts.

%%%%%%%%%   then the Bibliography, if any   %%%%%%%%%
\bibliographystyle{plain}	% or "siam", or "alpha", etc.
\nocite{*}		% list all refs in database, cited or not
\bibliography{refs}		% Bib database in "refs.bib"

%%%%%%%%%   then the Appendices, if any   %%%%%%%%%
\appendix
\input {appendixA.tex}
\chapter{Selected Transcriptions}

This appendix contains several illustrative transcriptions from the third user
study with the PopCAD and SnapCAD, where we recorded the speech and gesture
expressions of users explaining their modeling strategy. These excerpts contain
quotes from users generated while explaining their modeling strategy for a given
shape, the observed gestures that occurred during the spoken explanation, and
the speech and gesture codes generated from those expressions. A reminder of the
codes and examples used are provided in Table \ref{GcodingStrategy}.

\begin{table}[!ht]
\small
    \caption[Coding rubric for speech and gesture during user explanation of
    modeling strategy]{ The various coding strategies used in the video
    analysis of subjects' modeling strategy explanations. Borrowed and adapted
    from \cite{ehrlich2006importance}.}
    \begin{center}
    \begin{tabular}{| p{1.5cm} | p{4.2cm} | p{4.2cm} | p{4.2cm} |} \hline
	$Category$ & $Definition$ &   $Speech$ $Examples$  & $Gesture$ $Examples$ \\
	\hline Movement & Any indication of movement & ``Just slide them together and then it
	looks like that'' (S.M) & Miming movement with the hands (G.M) \\ \hline 
	Perceptual Features & Focus on a particular feature of the model & ``Because
	there is a little bend in here and a point thing here'' (S.PF) & Pointing to a
	specific feature on the model (G.PF) \\ \hline 
	Perceptual Whole & Any indication of seeing the model as a whole & ``It looks
	like an arrow!'' (S.PW) & Gesture indicating inclusion of the whole shape 
	(G.PW) \\
	\hline Vague & An expression of strategy that the coder cannot decipher & ``Because I
	looked at that and I looked at the differences'' (S.V) & Waving gestures above
	the computer device that do not indicate any specific strategy (G.V) \\ \hline 
	Other & Any strategy not listed above & ``And here is like half of it.
	But so and two halves make a whole'' (S.O) & Using the hand to form a straight
	line through the middle of the whole shape to represent the line of symmetry
	(G.O)
	\\
	\hline
	\end{tabular}
   \\ \rule{0mm}{5mm}
\end{center}
\label{GcodingStrategy}
\end{table}


\emph{Excerpt 1}

\begin{quotation}
``It goes upward, downward, over, up,across, over downward, over, and doesn't
go up again."
\end{quotation}
Gesture: (Pointing along sides of the shape)\\
Codes: S.M, G.PF \\ 
- User3, Shape P2, Session 1\\


\emph{Excerpt 2}

\begin{quotation}
``I saw that is was a series of L shapes\ldots I just thought that if I
could make a series of L, I could soon make it." 
\end{quotation} 
Gesture: (rotating model in hand repeatedly)\\
Codes: S.PW, S.PF, G.V\\ 
- User 4, Shape P2, Session 1\\

\emph{Excerpt 3}

\begin{quotation}
``Well it's based on the points, so the shape has 6 points, so needed 5
towers for the 6 points\ldots because of how to works you get one point there, you go
down, then another point, then you go over, on the same kind of layer\ldots and
then in the middle, so you don't just have a flat square, and you're pulling it up, you get
two points, so it is pulling the whole thing."
\end{quotation} 
Gesture: (pulls ``up'' with hands, points to towers, makes flat hand for
layers)\\
Codes: S.PW, S.PF, S.M, S.O, G.PF, G.O, G.M\\ 
- User 11, Shape CH1, Session 1\\

\emph{Excerpt 4}

\begin{quotation}
``Because the lines are the same way, as, the, uh, that shape."
\end{quotation}
Gesture: (shrugs)\\
Codes: S.V, G.V\\ 
- User 8, Shape P2, Session 1\\

\emph{Excerpt 5}

\begin{quotation}
``By taking each point, each side, and creating it into a point on the
graph" "on this shape, uh, this is about, uh 3 points wide, and 2 points tall, like, on
this, and then also, like 2 points out."
\end{quotation}
Gesture: (at `like this' points to two lights on the device one after the
other.)\\
Codes: S.O, S.PF, G.PF \\
- User 7, Shape T2, Session 1\\

\emph{Excerpt 6}

\begin{quotation}
``If it was hull then I knew it would automatically connect and fill in between
the points, so all I really needed to do was make two triangles, and it would
fill in the rest" 
\end{quotation}
Gesture: (gestures with hands to indicate two parts moving together)\\
Codes: S.O, S.PF, G.M, G.PF\\
- User 11, Shape CH1, Session 2\\

\emph{Excerpt 7}

\begin{quotation}
``Because, it didn't look like a pyramid, but like a right pyramid\ldots like a
right angle, one side is a right\ldots I put four there, in each corner\ldots I
put one on top because it was the point, a vertex\ldots and then I looked at the laptop and it
looked like the shape." 
\end{quotation}
Gesture: (points to different corners of the shape) \\
Codes: S.PW, S.PF, S.O, G.PF, G.V\\
- User 17, Shape CH4, Session 2\\

\emph{Excerpt 8}

\begin{quotation}
``When I was trying to make this part, I was trying to make it small, from the
middle\ldots I made it look like it's supposed to look, I made it down,
straight, up, right, down, straight, up\ldots so I try to just mimic the shape
and see how it works."
\end{quotation}
Gesture: (gestures along the path on the device that the user is describing)\\
Codes: S.PF, S.PW, S.M, G.M, G.PF\\
- User 1, Shape T3, Session 2\\

\emph{Excerpt 9}

\begin{quotation}
``There's two U's, and um, like an entrance to a doorway."
\end{quotation}
Gesture: (traces with finger along different parts of the shape)\\
Codes: S.PF, S.O, G.PF\\
- User 5, Shape T3, Session 1\\

\emph{Excerpt 10}

\begin{quotation}
``Basically drawing it out with my mind." 
\end{quotation}
Gesture: (makes drawing motion with hand, then a sort of vague waving motion)\\
Codes:  S.O, S.V, G.M, G.V\\
- User 2, Shape P3, Session 2\\





\end{document}

