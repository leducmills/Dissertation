\documentclass[defaultstyle,11pt]{thesis}

\usepackage{amssymb}		% to get all AMS symbols
\usepackage{graphicx}		% to insert figures

\usepackage{rotating}
\usepackage{graphics}
\usepackage{colortbl}
\usepackage{color}
\usepackage{hyperref}
\usepackage{subfig}
\usepackage{amsmath}

%%%%%%%%%%%%   All the preamble material:   %%%%%%%%%%%%

\title{Devices for Embodied Fabrication}

\author{Benjamin A.}{Leduc-Mills}

\otherdegrees{B.A., University of Santa Cruz, 2003 \\
	      M.P.S., New York University, 2008 \\
	      M.S., University of Colorado, 2013}

\degree{Doctor of Philosophy}		%  #1 {long descr.}
	{Ph.D., Computer Science}		%  #2 {short descr.}

\dept{Department of}			%  #1 {designation}
	{Computer Science}		%  #2 {name}

\advisor{Prof.}				%  #1 {title}
	{Michael Eisenberg}			%  #2 {name}

\reader{Prof.~Clayton Lewis}		%  2nd person to sign thesis
\readerThree{Prof.~Tom Yeh}		%  3rd person to sign thesis

\abstract{  \OnePageChapter	% because it is very short

Digital fabrication technologies are increasingly finding their way into
educational spaces of all shapes and sizes. These new technologies 
(3D printers, laser cutters, etc.) afford opportunities for exploring these new
ways of `making' and how they may change the way we learn, explore, and play.
Although there is much excitement surrounding the `maker movement' - and 3D
printing in particular - there has been little examination of how to introduce
a younger audience to 3D printing in an empowering way.
This proposal argues that tangible interfaces - as opposed to 2D screen-based
media - can be designed not only to support spatial reasoning and mathematical
intuitions in children by engaging them in exploratory modeling and play, but
that these interfaces can act as a democratizing force by enabling children to
create physical objects with digital fabrication devices.
The proposed work presents a series of novel tangible input devices for
enhancing mathematical and spatial reasoning in kids with a focus on generating
output for 3D printing.
We discuss related work, the status of the proposed work, additional
improvements to be made, a timeline for completion,
and a discussion of risks, limitations, and outcomes inherent in the proposal.
	}

\dedication[Dedication]{	% NEVER use \OnePageChapter here.
	To all of the fluffy kitties.
	}

\acknowledgements{	\OnePageChapter	% *MUST* BE ONLY ONE PAGE!
	Here's where you acknowledge folks who helped.
	But keep it short, i.e., no more than one page,
	as required by the Grad School Specifications.
	}

% \IRBprotocol{E927F29.001X}	% optional!

\ToCisShort	% use this only for 1-page Table of Contents

\LoFisShort	% use this only for 1-page Table of Figures
% \emptyLoF	% use this if there is no List of Figures

\LoTisShort	% use this only for 1-page Table of Tables
% \emptyLoT	% use this if there is no List of Tables

%%%%%%%%%%%%%%%%%%%%%%%%%%%%%%%%%%%%%%%%%%%%%%%%%%%%%%%%%%%%%%%%%
%%%%%%%%%%%%%%%       BEGIN DOCUMENT...         %%%%%%%%%%%%%%%%%
%%%%%%%%%%%%%%%%%%%%%%%%%%%%%%%%%%%%%%%%%%%%%%%%%%%%%%%%%%%%%%%%%

\begin{document}

\input{macros.tex}
\chapter{Introduction}
\label{introchap}


% what is embodied fabrication?
% what are the goals of this work? (democratization of 3d printing)

Ten years ago, 3-dimensional printing was solely the purview of large
fabrication studios and industrial manufacturing; five years ago the first
desktop ``homebrew'' 3D printers hit the market, though few people seemed to pay
much attention; today, desktop 3D printing is one of the big headlines at the
annual Consumer Electronics Showcase in Las Vegas, media outlets from
Forbes\cite{forbes} to the Economist\cite{economist} are discussing it, and most
of the teenagers we talked to during our user studies know what 3D printing is.
3D printing is part of a forceful trend often referred to as the ``maker
movement''\cite{anderson2012makers}, that puts do-it-yourself ethics and
emerging technology together in a way that has inspired people the world over to
put down the TV remote and pick up a soldering iron. A subset of this movement
has focused on ``digital fabrication'' technology, of which 3D printers, laser
cutters, CNC mills, and vinyl plotters (among other devices) belong. Digital
fabrication machines take computer-generated files as input and fabricate
physical objects from those files. With the help of the maker movement and
visionary works on the upcoming age of ``personal
fabrication''\cite{Gershenfeld:2007:FCR:1211574}, 3D printing machines that once
cost tens of thousands of dollars are now available as DIY kits for less than
one thousand. Do not misunderstand us - this is a wonderful thing; cost is one,
if not the main barrier to the spread of most technologies.
However, the maker movement is not without its blind spots. Most innovators
behind these desktop 3D printers are of a very privileged socioeconomic
background; many of them retired engineers or otherwise possessing technical
training far beyond the average person. For the most part, they have not (nor is
it necessarily their responsibility to have) truly thought about how to make
their low-cost 3D printers accessible to the average Joe and Sally - much less
Joe Jr. - and to be fair, they are not the only actors contributing to the
barrier of entry for 3D novices who wish to design for 3D printers.

For those readers who may be unfamiliar with 3D printing, it is indeed what it
sounds like - an umbrella term for one of several processes capable of creating
3-dimensional objects (usually fine layers of extruded plastic filament) from
digital files, much like a laser printer prints 2D images on paper. 3D printers
primarily take in a file format called stereolithography - or .STL for short.
Normally, to create an .STL file one needs a rather complicated,
professional-level piece of 3D-modeling software, such as Rhino\cite{Rhino} or
Solidworks\cite{Solidworks}; programs which are rich with features that only
seasoned users will find a need for, with sub-menus upon sub-menus and decidedly
particular behaviors that any novice - especially a young one - would find quite
intimidating. As anyone who has used these software programs knows, one must be
very precise and conscious of every operation for a model to turn out properly -
and this order of operations is learned slowly (often agonizingly) over time. It
is a user interface nightmare; hardly the soft of intuitive environment one
might want to learn with. It should be noted that some efforts have been made to
create entry-level 3D modeling software - most notably Google
SketchUp\cite{SketchUp} - although last we checked SketchUp did not export
directly into .STL format (there are some rather troublesome looking workarounds
however), leaving the average newcomer facing an incredibly steep learning curve
in order to produce any original, 3D-printable objects. We emphasize
``original'' because there are several fairly simple ways to download and print
out a pre-created .STL file from the Internet (most notably from the on-line
repository Thingiverse\cite{thingiverse}).

\begin{figure}[!ht]
\begin{center}$
\begin{array}{cc}
\includegraphics[width=.4\linewidth]{images/makerbot}&
\includegraphics[width=.5\linewidth]{images/MB05_REP_Group}
\end{array}$
\end{center}
\caption{Left: One of the first popular desktop 3D printers, the MakerBot
``Cupcake CNC'', released in 2009. Right: The latest group of MakerBot models,
released at the Consumer Electronics Showcase in January 2014.}
\label{makerbot}
\end{figure}

Although printing out dozens of army men or barnyard animal figurines may be
satisfying for a time, and indeed speaks to the compelling nature of 3D
printing, it seems fair to say that children do not learn much about 3D modeling
from a ``download and print'' paradigm. Herein lies the crux of the problem - 3D
printing offers a wonderfully rich new platform for design, creativity, and
exploration, but neither the 3D printer manufacturers nor the companies who
produce the software necessary to author files suitable for 3D printing have
made accessibility for novices a priority. This is where our journey begins:
the desire to democratize 3D printing in a way that empowers newcomers,
particularly youngsters, in designing their own objects for 3D printing;
engaging them in such a way that intuitively introduces many of the core
concepts of 3D modeling, while helping to solidify cognitive processes around
spatial reasoning and 2D/3D translations, by building a set of devices that act
as a new genus amongst an ecosystem of next-generation digital fabrication
interfaces.

How, then, did we arrive at the term ``embodied fabrication'' to describe this
new genus? The simple answer, at the risk of over-extending the genealogical
metaphor, is that we selected what we deemed to be the best, most relevant
traits from a number of related areas (computer science, cognitive science,
developmental psychology, pedagogical theory, and digital fabrication
technology, amongst others) and attempted to splice them together in such a way
as to meaningfully address the issues with 3D printing outlined above. 

We derive the term ``embodied'' from cognitive science, and the fairly recent
advances in an area known as ``embodied cognition''. Embodied cognition posits
that our physical bodies and their interactions with the world are more closely
bound to our cognitive processes than previously thought. Evidence from research
in this area (discussed more thoroughly in Chapter 3 on related work) points to
cognitive benefits in basic arithmetic, ratios, proportions, and spatial
reasoning - all of which are useful (if not essential) tools in 3D modeling,
simply by involving the body more closely in the learning process.
This evidence, combined with the simple intuition that learning the skill of
3-dimensional modeling ought to be done in 3-dimensions as much as possible and
not solely on a 2-dimensional screen, provided the impetus for us to look toward
a physical solution that involves the body more than a typical piece
of software.

Physical, or ``tangible'' user interfaces are nothing new; wooden blocks have
been a part of children's education in a pedagogical sense since the beginning
of kindergarten over 150 years ago\cite{froebel}. Montessori ``manipulatives''
developed in the early part of last century inspired some of the first attempts
at creating physical, computationally-enhanced construction kits for children in
the 1980's\cite{Resnick:1998:DMN:274644.274684}. Tangible user interfaces, or
TUIs have been a growing part of human-computer interaction in a formal way
since Hiroshi Ishii's work on ``tangible
bits''\cite{Ishii:1997:TBT:258549.258715} in the mid 1990's, and of course the
influence of icons such as Doug Engelbart\cite{engelbart1968research} - one
might argue the mouse was the first ``embodied'' peripheral for a computer, in
the 1960's - and Mark Weiser\cite{weiser1991computer} (who presaged many of the
devices we take for granted today) as well as many others, should not be
overlooked - we give a more detailed account of this lineage when discussing
related work. For us, the longevity, breadth of applications, and numerous
achievements of mediating human-computer interaction though different physical
interfaces further suggests that a tangible user interface, coupled with the
proper software is more than capable of providing an accessible and embodied
foundation for our work.

Taking design principles from the lexicon of tangible user interfaces, adapting
them to better fit an embodied cognition world-view, and focusing on enabling 3D
modeling specifically for 3D printers, we designed and built a suite of
functional prototype devices for an embodied mode of digital fabrication; hence
the title of our work. To this end, we present a class of tangible user
interfaces designed to scaffold a child's ability to design, explore, and play
in three dimensions, with a particular focus on enabling original output for 3D
printing. We present three prototype devices (called UCube, SnapCAD,
and PopCAD) as well as piece of companion software that translates the physical
actions performed on the devices into screen-based content in real-time.

To give a brief preview of our creations; with their hands, users manipulate a
device to specify points (as coordinates in 3-space) that simultaneously display
as active dots against a ghosted 3D grid in real-time on a computer. The
software on the computer allows for certain modeling operations on this set of
input points (e.g., taking the convex hull, making a path through space),
exporting shapes to stereolithography (.STL) format with the click of a button,
the preferred format for 3D printers, as well as other functionality that we
explore more thoroughly in the next chapter.

We propose that these designs form a novel class of embodied input devices aimed
at enabling novice output for digital fabrication machines. Over three separate
user studies with 11 to 18 year olds, we investigate the ability for children to use
our devices to model a given shape (with and without the companion software) and
to match configurations on our device to a printed 3-dimensional object (without
the aid of the software). In our last study we compare two of our devices over a
multi-session study, while also administering a set of spatial reasoning tasks
as a pre and post test. We video record the subjects (with parental consent)
and analyze the gesture and speech expressions the participants make when
explaining a modeling strategy to reproduce a given object.

Through our studies, we show evidence that our suite of devices can be used
effectively by young adolescents with very minimal instruction, that a wide
variety of shapes can be recreated by the majority of subjects who used our
devices, that spatial test scores and modeling performance tends to improve over
multiple sessions with our devices, and that the kinds of gestures produced
while explaining modeling strategy correlates to modeling success on our
devices, a finding which supports prior research on gesture analysis by other
authors.

By providing a feedback loop between the bodily interaction with
tangible interfaces and the observed changes in real-time on a computer screen,
this body of work presents strong new motives for the inclusion of embodied
cognition in tangible interface design, while tackling the lack of appropriate
tools for novices to create for 3D printers, and evaluating the efficacy of our
devices as modeling tools and as devices for strengthening spatial reasoning and
cognition. We continue in Chapter 2 to present our prototype devices and the
software they operate with, explaining the evolution of our design choices as
well as the technical details behind their operation. Chapter 3 details the lineage
of related work, hinted at somewhat in this introduction, drawing connections
between the childhood developmental theories and conceptions of space developed
by Piaget and refined by Papert, the notions of cognitive development and
embodied mathematics discussed by Lakoff and Nu\~nez, the democratization of
digital fabrication technologies discussed by Gershenfeld and Lipson, and the
previous adaptation of these achievements into computer science. Chapter 4 is
devoted to the evaluation of our work, presenting three user studies, their
procedures and results. Chapter 5 delves deeper into the discussions which
surround the observations from our studies, comparing them with prior research,
and against each other. Finally, Chapter 6 provides a vision for immediate
future work on our devices, a more expansive vision of the possibilities
inherent in embodied fabrication, and ends with our concluding thoughts.



% The work presented here draws on the stages of childhood developmental theories
% and conception of space developed by Piaget and refined by Papert, notions of
% cognitive development and embodied mathematics discussed by Lakoff and Nu\~nez,
% the democratization of digital fabrication technologies discussed by Gershenfeld
% and Lipson, and the previous adaptation of these achievements into computer
% science.
























%old intro
% Digital fabrication technologies are increasingly finding their way into
% educational spaces of all shapes and sizes. These new technologies 
% (3D printers, laser cutters, etc.) afford opportunities for exploring these new
% ways of `making' and how they may change the way we learn, explore, and play.
% Although there is much excitement surrounding the `maker movement' - and 3D
% printing in particular - there has been little examination of how to introduce
% a younger audience to 3D printing in an empowering way.
% This proposal argues that tangible interfaces - as opposed to 2D screen-based
% media - can be designed not only to support spatial reasoning and mathematical
% intuitions in children by engaging them in exploratory modeling and play, but
% that these interfaces can act as a democratizing force by enabling children to
% create physical objects with digital fabrication devices.
% The proposed work presents a series of novel tangible input devices for
% enhancing mathematical and spatial reasoning in kids with a focus on generating
% output for 3D printing. We discuss related work, the status of the proposed
% work, additional improvements to be made, a timeline for completion,
% and a discussion of risks, limitations, and outcomes inherent in the proposal.
% 
% 
% %from proposal
% A number of computer scientists, technologists, and educators have declared that
% the era of personal fabrication is upon
% us\cite{anderson2012makers}\cite{Gershenfeld:2007:FCR:1211574}. New devices
% aimed at increasing the ability of the individual to physically manufacture
% their own ideas are being released at breakneck speed. The cultural and
% technological shifts caused by this change are taking many forms, yet few
% technologies associated with the `maker' movement have received as much
% attention as 3D printing - the ability (by various means) to digitally design
% and then print out physical 3-dimensional objects. Media outlets from
% Forbes\cite{forbes} to The Economist\cite{economist} have extolled the
% disruptive and democratizing possibilities that 3D printing offers - at least as
% it affects the traditional manufacturing supply chain. Less examined has been
% how to introduce novices, specifically pre-teens and early adolescents, to 3D
% printing - and perhaps more importantly - discussing what (and how) they might
% learn by being exposed to it.
% While the variety of desktop 3D printers continues to increase and the cost of
% adding a `fab lab' of digitally-based manufacturing tools in the home or
% classroom steadily declines, the types of interfaces by which children can
% easily and intuitively design and explore the capabilities of 3D printers still
% remains a barren landscape consisting primarily of software-only solutions. It
% is this landscape that we are interested in seeding, following the best
% practices in computational and cognitive science with particular attention to
% children-centered design.




% \section{A Brief Overview of this Thesis}
% 
% \section{Motivations}

\chapter{Prototype Systems}
\label{prototypes}

Over the past several years we have been exploring various means of creating a
child-friendly tangible user interface that would serve as an input device for
exploring 3D modeling and digital fabrication. To this end, we have created
three prototypes: the UCube, an initial proof-of-concept device, SnapCAD, a more
expressive and study iteration of the UCube, and PopCAD a paper-based interface
addressing several of the concerns raised by SnapCAD. These systems all
communicate with versions of a companion software program running on  desktop
computer. This chapter describes these systems, the motivations behind their
design, and the technical work involved in their creation.

\section{UCube}

The UCube represents our first attempt to create a cooperative system of
hardware and software that encapsulated and combined our beliefs about embodied
cognition and the importance of accessible digital fabrication. The idea for the
UCube originally came from the attempt to create a ``3D Geoboard''.
\autoref{fig:geoboard} shows a rudimentary 2D geoboard consisting of a 3x3 grid
of nails stuck into a wooden block. Simple geometries, such as the triangle shown
in the referenced image, can be made by stretching rubber bands around some
number of ``pegs''. The geoboard invites a kind of tangible, exploratory, and
embodied play that (as we discuss in Chapter 3) promotes children's learning in
powerful ways. The goal, then, was to capture the ``gestalt'' of the traditional
2-dimensional geoboard and extend it - into 3-dimensions, and with a
computationally-enhanced interface that could translate physical modeling on a
device into a software program that could display the input from the geoboard in
a meaningful way.

\begin{figure}[ht]
\begin{center}$
\begin{array}{cc}
\includegraphics[width=.5\linewidth]{images/Geoboard}
\end{array}$
\end{center}
\caption{A simple 3x3 geoboard, with a rubber band stretched around several
pegs, forming a triangle.}
\label{fig:geoboard}
\end{figure}

The UCube (as seen on the left in \autoref{fig:cubev1}) was the initial result
of this goal. The physical interface consists of a set of vertical ``towers''
that are placed (and optionally re-placed) onto a board, acting somewhat like
the nails in the 2D geoboard. These towers are moved around a grid of 4x4 evenly
spaced nodes or sockets into which the towers are placed. The towers themselves
contain four switches placed vertically along the tower, creating a potential
for 64 (4x4x4) distinct points. Thus, when a tower is placed in a specific node
on the board and a switch is flipped on, a particular (x,y,z) coordinate in
three-dimensional space is activated and sent through a microcontroller to a
piece of software on the computer. An abstracted illustration of the hardware
system is seen on the right in \autoref{fig:cubev1}.

In turn, the UCube software takes the incoming coordinate data from the
microcontroller and translates it into a real-time visualization on screen. The
graphical user interface centers around a ``ghosted'' grid of all the potential
points, with the active points being highlighted. In the first version of the
software, the interface also provides a set of operations that can be performed
on the set of active points in addition to normal scene manipulations like zoom
and rotate. These functions include: taking the convex hull of the point set (as
imagined in \autoref{fig:cubev1}), creating a sequential path or knot through
the active points, exporting the convex hull or knot to .STL format for 3D printing,
drawing a (non-printable) spline through the active points, saving and loading a
shape, and editing the vertices of a convex hull via a click-and-drag interface
(a more complete review of the software occurs later on in this chapter).


\begin{figure}[ht]
\begin{center}$
\begin{array}{cc}
\includegraphics[width=.35\linewidth]{images/UCube-2}&
\includegraphics[width=.48\linewidth]{images/ucube_diagram}
\end{array}$
\end{center}
\caption{Left: The UCube device, with four towers and eight lit switches,
representing the eight vertices of a cube. Right: a schematic illustration of
the UCube hardware.}
\label{fig:cubev1}
\end{figure}

% We performed two separate studies using the first UCube interface with
% middle school children aged 11-14. The first (informal) study, detailed in
% \cite{Leduc-Mills:2011:UCD:1999030.1999039} had fourteen participants,
% consisting of five girls and nine boys, who were divided into six groups (five
% groups of two, one group of four). Participants were asked to model a sequence
% of five shapes of increasing complexity using the UCube along with the companion
% software. The target shapes were displayed on one half of a computer screen,
% while the UCube software showing the live model was displayed on the other half.
% The shapes were as follows:  a straight vertical line, a diagonal line, a cube,
% a triangular prism, and finally an irregular polyhedral object. No shape
% required more than four towers to complete, and shapes were always presented in
% the same order. Of the six groups who participated, four groups successfully
% modeled all five shapes, one group ran out of time after three shapes, and one
% group finished one shape. Sessions lasted between 17 and 30 minutes.


\begin{figure}[ht]
\begin{center}$
\begin{array}{cc}
\includegraphics[width=.45\linewidth]{images/ucube1_software} &
\includegraphics[width=.45\linewidth]{images/ucube1_user}
\end{array}$
\end{center}
\caption{Left: a screenshot of the UCube v1 software, showing the triangular
prism generated by performing the convex hull function on a set of 6 input
points. Right: A photograph of a middle-school student using the UCube. Here, the
student holds a tower in the platform and points simultaneously to the screen
representation of the selected point on the desktop computer.}
\label{fig:cubev2}
\end{figure}


% The second user study, from \cite{Leduc-Mills:2012:SSV:2307096.2307176},
% consisted of ten participants, eight boys and two girls, each of whom
% participated individually in two separate exercises. The first exercise was a
% modeling task, whereby the participant was handed a series of 3D-printed shapes
% and asked to recreate them on the UCube interface. The five physical models
% presented were: a cube, a tetrahedron, a diamond, a �house� (a cube with a
% pyramid on top), and a complex irregular polyhedron. The results were promising:
% overall, 21 of 50 shapes were completed from memory, 12 of 50 were completed
% while holding the shape, and a further 8 of 50 were completed with the aid of
% the UCube software, for a total of 41 out of 50 shapes modeled successfully
% (82\%). Of the nine missed shapes, seven were of the same shape, the complex
% polyhedron. The remaining two misses were from the same participant, who ran out
% of time before completion.
% 
% The second task was a matching task whereby participants were instructed to face
% away from the UCube while the facilitator modeled a set of lights on the UCube
% corresponding to one shape among a set of nine physical models laid out on the
% table next to the UCube. Once the lights on the UCube were set up, the
% participant was instructed to turn around, and indicate which physical object
% they thought the set of lights on the UCube corresponded to. Of the nine shapes,
% the participants were asked to match five of them (a cube, a triangular prism, a
% parallelogram, an elongated hexagon, and a trapezoid). Thus, only the cube was
% presented in both the matching and modeling exercises. Out of 50 matching tasks
% (5 per participant), 0 tasks resulted in the incorrect match being selected, and
% most matches were made in under 20 seconds, an encouraging result that points to
% the ability of children (of this age) to recognize convex hulls from a set of
% illuminated points.

% from IDC 2011 paper
As a first step in discussing the UCube's role in spatial design�and in
discussing the broader issue of children's three-dimensional design�this section
is devoted to a more thorough description of the UCube and its operation.
To begin with an overview, then: the UCube system is the combination of two
elements: the physical input device of ``towers'' placed on a board, and the
companion display software. These two systems work together to take the embodied
actions of the user and display corresponding points and shapes on the computer.
A sense of the scale of the device can be inferred from \autoref{fig:cubev2},
which shows a photograph of a middle-school student holding a newly-placed tower
in the UCube platform while pointing simultaneously at the desktop computer
screen beside it.
This photograph�which we will also return to in the discussion of pilot testing
in a later section�reflects the essential nature of interaction with the device:
points are designated in a spatial region provided by the platform, and then
represented in real time on the computer screen. Thus, the UCube promotes an
attention to the correspondence between the selected spatial points above the
platform and the (more abstract) representation on the computer screen.

\subsection{Hardware} 
The physical system for our first UCube prototype, as outlined earlier, consists
of a platform with a four-by-four grid of potential sites, each of which can
hold one tower with four switches, thus describing a 4x4x4 array of 64 potential
points.
The platform structure consists of three different horizontal ``layers''. The
top (or upper surface) layer has a four-by-four grid of circular holes, into
which the towers fit snugly. This layer of 1/4'' thick laser-cut clear acrylic
acts as a brace to hold the towers upright, and ensures that they are resistant
to being knocked over. The next layer down holds the headers, which allow the
towers to ``plug in'' and connect to the rest of the circuit. Wires from the
headers go down to the bottom layer, which holds the breadboarded circuit and
Arduino Mega microcontroller. The towers are made of transparent acrylic, the
side paneling of basswood. The towers were laser-cut in order to house the four
switches and corresponding circuitry elements. The switches are LED-backlit when
active, making it more apparent which points are active as well as giving a more
accessible ``gestalt'' of the shape being modeled. It also allows for some
potentially interesting applications in dimly-lit circumstances, such as
modeling constellations in a classroom or planetarium: in these situations, the
lights of the selected spatial points stand out especially vividly.

Each tower connects to the platform through a six-pin header (one pin each for
power, ground, and four switches). The switch connections are then routed
through a breadboard containing current limiting resistors for the LED switches
to pins on a microcontroller (an Arduino Mega\cite{ArduinoMega}).
The Arduino is then able to communicate (via asynchronous serial communication)
the active switches (and corresponding coordinates) to the computer through a
USB cable. \autoref{fig:cubev1}(right) depicts a schematic diagram of the UCube
hardware.

% \subsection{Software}
% 
% The UCube makes use of the Processing\cite{Processing} framework to read in
% the active coordinates from the Arduino microcontroller connected to the
% platform; the software then displays these as larger red points on a grid of
% grey dots. Users can rotate the grid along any axis by clicking and dragging
% with the mouse. In our current early prototype, there are only two buttons on
% the user interface: (i) an �export� button, responsible for taking the current
% set of active points and exporting them into "STL" file format (suitable for 3D
% printers), and (ii) a �mode� button which toggles between showing just the red
% dots as points and filling in an area (defined by a convex hull algorithm) to
% give a sense of shape.
% The software interface is intentionally minimal in order to encourage the user
% to focus on the physical interaction. We felt it was crucial not to fall into
% the trap of making another software tool for experts, so the main purpose of the
% software is to act as an aid�a means to cognitively clarify and confirm the
% user's intentions. Although it is likely that we will extend the software
% somewhat in future iterations, our goal is to support the physical experience of
% specifying a three-dimensional object, and not to add functionality beyond what
% is necessary or helpful to that end.

% \subsection{A Sample UCube Scenario}
% As a sample scenario, imagine that we wish to create a triangular prism solid
% employing the UCube. We can begin this process by selecting three points to form
% a triangle, as shown in Figure 6; then, by placing two more towers and creating
% the same triangular shape "shifted over" by two units (Figure 7) we create the
% entire prism. Naturally, there might be many alternative pathways to forming the
% same eventual shape: for example, we might begin by placing four (or more)
% towers in the platform, and then experiment or fiddle with the chosen lights to
% approach the eventual goal of creating our prism. Alternatively, we might begin
% without any towers in the device at all: by placing our hands or fingers above
% the device, roughly indicating where the prism should be, we might then use our
% imagined locations as "guides", helping us to place the necessary towers in the
% platform and select the correct lights for the vertices of the prism.
% In any event, having designed the prism using the UCube platform, and having
% checked that it looks like the correct shape on the computer screen, the final
% step is to export the shape into a format suitable for 3D printer output. The
% UCube software, as noted earlier, includes a feature for doing just this; and
% finally, we print out the prism, as shown in Figure 8.
% Figure 6. The first step in constructing our triangular prism: here, we create a
% planar triangular shape toward the left side of the platform, and can see the
% resulting shape on the computer screen shown at right.
% Figure 7. Completing the triangular prism. Here, we have added a second
% ("shifted") version of our original triangle to produce the six vertices needed
% to form the prism.

\subsection{Limitations}
It will probably not have escaped the reader's notice that the UCube, as a
three-dimensional modeling device, has significant limitations. To take the most
glaring of these: the user can only model those shapes whose vertices are among
the sixty-four locations accessible from the device. Moreover, those available
locations are evenly spaced in the form of a three-dimensional grid, or lattice;
thus, there are numerous simple-but-interesting shapes (such as the regular
dodecahedron, composed of regular pentagonal faces) that cannot be designed in
the current version of the UCube. Likewise, shapes with curved surfaces (such as
a cylinder), demanding at the very least a high resolution of accessible points,
could not be modeled in the current UCube. We will return to these issues in the
final section of the paper, in the discussion of ongoing and future work.



\section{SnapCAD}

\begin{figure}[ht] \begin{center}$
\begin{array}{cc}
\includegraphics[width=.45\linewidth]{images/BeatriceFinal-2} & 
\includegraphics[width=.45\linewidth]{images/BeatriceFinal-14}
\end{array}$
\end{center}
\caption{
Left: the SnapCAD interface, showing the hardware configuration corresponding to
the picture below in \autoref{fig:ucubev22}. Right: a detail of the SnapCAD
hardware - the PCB tower is housed in a 3D-printed shell, which plugs into a
shift-register board. The LED boards snap on to the towers via magnetic snaps.}
\label{fig:ucubev21}
\end{figure}

Based on the feedback from these two user studies, a second, more powerful
instantiation of the ideas from the UCube has been created. SnapCAD (formerly
known as UCube v2) consists of a total input space of 7x7x7 points, forming 343
distinct coordinates. In our user studies with UCube v1, we noticed that users
often encountered initial difficulties when required to `find a middle' in the
shape they were attempting to model, given an even number of total grid spaces.
For example, to model a pyramid on on a 4x4x4 grid, one needs to construct a 3x3
subset of the 4x4 grid, using the middle point within the 3x3 set as the top of
the pyramid. This influenced our decision to create an odd-numbered layout,
creating a more `natural' middle point in the hardware. The greater number of
inputs vastly increases the expressive potential of SnapCAD (compared to the
UCube) while still maintaining a manageable interface.
Working on the scale of multiple hundreds of inputs necessitated the design of
custom circuit boards to relay information effectively to the microcontroller.
This change in scale also meant rewriting most of the modeling software to
effectively handle the greater expressiveness of the physical system.



\begin{figure}[ht] \begin{center}$
\begin{array}{cc}
\includegraphics[width=.45\linewidth]{images/twoHulls} &
\includegraphics[width=.45\linewidth]{images/mst}
\end{array}$
\end{center}
\caption{Left: The SnapCAD software showing two convex hulls of different
colors. Right: the SnapCAD software showing a minimal spanning tree model.}
\label{fig:ucubev22}
\end{figure}


The use of conductive, magnetic snaps along towers constructed of custom-printed
circuit board allow for more than one color of illumination, as different
colored LED boards can be snapped onto any socket on the tower.
This not only results in the ability to represent multiple shapes at once, but
for the SnapCAD to become a platform for all manner of multi-player interactions
(e.g. games, puzzles, shape matching contests), with each `player' assigned a
unique color. To this end, we have created a simple `3D Tic-Tac-Toe'
implementation on the SnapCAD. Additional changes to the software include
supporting multiple but separate convex hulls of different colors, the ability
to create and export shapes created from the minimal spanning tree of a set of
input points, and the ability to adjust the width of the segments in the
knot/path and minimal spanning tree modes.
The click-and-drag editing mode now includes the knot/sequential path and
minimal spanning tree modes as well as the convex hull mode. We also adjusted
the knot-forming algorithm to handle paths that cross or self-intersect, as well
as providing a `close knot' button to complete a circuit in a shape, allowing
for even more kinds of 3D-printable objects. While significant work has been
done to bring the UCube and SnapCAD to their current states, we believe not only
that there is room for additional improvements to be made, but that, as opposed
to focusing on a incremental but essentially similar interface as the subject of
a thesis, it is far more intellectually interesting to focus on a class of
objects that demonstrate multiple incarnations of a set of ideas.

% \section{Proposed Work: Technical Additions}
% The proposed work is in two sections: technical additions and
% evaluation. This section deals with the technical additions to the proposed
% devices: SnapCAD and PopCAD.
% 
% \subsection{SnapCAD}
% 
% With 343 potential points, a click-and-drag editing tool, and three separate
% modeling modes (convex hull, knot/path, and minimal spanning tree) the SnapCAD
% is capable of generating countless 3-dimensional forms.
% Although the potential for additional modeling tools is certainly a possibility
% (we have yet to experiment with curved surfaces, for instance) we believe that
% the multi-player `platform' aspects of the SnapCAD system are the most ripe for
% development. We already have two colors of LED boards, the ability to display
% two colors of convex hulls, and a 3D implementation of a two-player tic-tac-toe
% game. Displaying multiple colors of the path/knot and minimal spanning tree
% modes should be fairly straightforward to implement.
% We would also like to expand and change the colors currently being used - we
% currently use red and green LEDs, which would be problematic for anyone with
% red/green color blindness. We propose using three colors: red, blue, and purple.
% This not only allows for up to three-player interactions, but could help to
% solve a deeper problem: representing in hardware a node occupied by two players.
% Using an idiom where solely-occupied nodes are either blue or red, and a jointly
% occupied node is purple, we can then expand the types of games, puzzles, or
% modeling activities the SnapCAD system can support. Once these improvements are
% made we can expand the activities supported on SnapCAD. Developments include
% two-to-three player games like tic-tac-toe as well as games built
% off of the modeling capabilities of the SnapCAD (e.g. match the model generated
% by the computer, model a sequential path through a generated maze, place points
% on or interior to the convex hull until there are no more to be found). Colors
% can also be used for certain as-yet unexplored modeling operations (e.g., the
% set union, intersection, or difference). While some of these operations may
% prove difficult or even impossible, these are all avenues worth exploring, as
% they all point towards the extensibility and potential expressiveness of SnapCAD
% as a platform for future development.

\section{PopCAD}

Our motivations for creating alternative interfaces to the UCube and SnapCAD
stem from the desire to explore this intellectual space more generally; it is
far more interesting to discuss a \emph{class} of tangible interfaces for
scaffolding digital fabrication than it is to discuss a singular device. To this
end, we looked at some of the weaknesses of SnapCAD and towards technologies we
had yet to explore. While SnapCAD can admirably perform a number of modeling
tasks, it was always envisioned as one device amongst an `ecosystem' of next
generation fabrication tools. It has strengths, but obvious weaknesses as well;
in particular, the SnapCAD hardware was expensive to produce, and so would be a
difficult proposition for some schools or fab labs; it is also rather unwieldy
and unportable - it moderately heavy, fairly large, and has many separate pieces
that could break or go missing. Thus, an interface with cheaper and more
portable materials was desirable.

To address these issues we chose to build a pop-up book combining traditional
paper-crafts and paper-friendly electronics such as copper tape.
In recent years, revolutionary work has been done in combining electronics and
paper
crafting\cite{Qi:2010:EPE:1709886.1709909}\cite{Mellis:2013:MMC:2460625.2460638},
leading to new techniques and new uses for traditional materials. Paper is
inexpensive (especially when compared to circuit boards), light, and easily
portable, making it an ideal material choice for a device that would not suffer
the same limitations present in the SnapCAD. Although we often think of `paper'
as a rather static material, there are in fact many variations in the size,
weight, color, transparency, and composition of contemporary paper products. For
the initial prototype, we used a simple construction paper as it provided a
balance between strength and flexibility as well as having a consistency
well-suited to laser etching and cutting.
The pop-up book (named PopCAD) has a 3x3x3 array of 27 points which are evenly
spaced 3 inches apart on a 12'' x 18'' paper surface.
The book folds on a single center crease making the closed footprint of the book
roughly 12'' x 9''.

\begin{figure}[ht] \begin{center}$
\begin{array}{cc}
\includegraphics[width=.45\linewidth]{images/popup1} &
\includegraphics[width=.45\linewidth]{images/popup2}
\end{array}$
\end{center}
\caption{Two views of the pop-up book prototype, showing the paper towers and 
LEDs in both open and closed states.}
\label{fig:popup}
\end{figure}

Each tower has a copper tape circuit consisting of three LEDs on the front face
and three corresponding capacitive touch sensors on the left face. The copper
tape acts as a paper-friendly conductive material to connect the electronic
components together much like traditional wire. The LEDs are soldered onto the
copper tape for greater stability. The capacitive sensors are simply a piece of
copper tape which is connected to a pin on a microcontroller (in the first
version, this is an Arduino Mega Pro). By bringing the internal pull-up resistor
connected to the pin `LOW' (to ground) and then timing how long it takes to get
back to a `HIGH' state we can tell if the connection is being influenced by a
capacitive force. For example, if there is no interference on the circuit, the
timer will normally only get to `1' before the resistor is back to a HIGH state;
if a finger is placed on the copper tape, the reading will be
much higher (typically around `17'). Based on this change, we can detect which
switch was touched and toggle the associated LED on or off. The hollow interior
of each paper tower is used to solder thin 30-gauge wire to the three LEDs, the
three switches, and ground. These seven wires are soldered to a row of headers
that stick through the bottom of the first layer of the pop-up book. Wires are
then run along the backside of the top layer of paper from these headers to the
microcontroller. The entire circuit in then encased in a cloth-covered cardboard
binder that acts as a book cover as well as a means to protect and hide the
electronics.

The software originally written for the UCube and SnapCAD was adapted to work
with the pop-up book, making it capable of similar types of algorithmic modeling
and stereolithography output for 3D printing. As the grid is 3x3x3, it also
makes sense to adapt some of the game-playing aspects of the larger devices
(e.g., it would still be possible to play 3D tic-tac-toe). In addition to adding
this functionality, there are several improvements and finishing touches to be
made on the book itself. Additionally, the current hardware setup for the pop-up
book does not allow for the LED's to be snapped on or off, making certain
multi-player or multi-shape operations impossible. Weather or not this
functionality is crucial to the pop-up book will determine if changes need to be
made.  

Given the different medium of the pop-up book (paper as opposed to circuit
boards), it is worth exploring the possibilities afforded by a cheaper, more
flexible material. For instance, the flexibility of paper might provide the
means for new types of modeling actions. It is plausible to imagine paper tabs
or other mechanisms that perturb the LEDs off the integer lattice, or alter the
overall topology in such a way that new shapes are possible (e.g. by deforming
an equidistant grid into a spherical shape). There may be additional sensors or
hardware that could be embedded into the book to provide new functionality
(rotation, proximity, pressure). Additionally, due the inexpensive and portable
nature of the pop-up book, it is worth exploring the sorts of interactions that
could occur between several pop-up books (e.g., extending the input field to
include two or more grids, networked interactions like cooperative modeling
tasks, or competitive games like 3D-battleship). By using paper as a material to
think with, we may find further possibilities as development continues.

\section{Software}
Put stuff about software development here. Details. Screenshots.
\chapter{Related Work}
\label{related}

% from proposal
The belief that tangible objects\footnote{It is worth noting the difference in
this work between `tangible objects' of the sort that a child might play with
(e.g. Lego) and `tangible user interfaces' (TUIs) that a child might interact
with - typically a peripheral device (apart from the keyboard and mouse) that
communicates physical interactions to a computer.} play an important role in
children's education is relatively recent. Friedrich Froebel's use of 20 wooden
forms he dubbed `gifts' in the first Kindergarten was in 1837\cite{froebel}. It
took until 1907 before an extension of Froebel's ideas and a focus on physical,
manipulative objects and tasks was implemented by Maria Montessori in the first
Casa Dei Bambini\cite{montessori}.
The interest in children's learning incorporating the use of manipulatives
progressed steadily, most notably by Jean Piaget and his work on `genetic
epistemology'. Piaget wrote extensively on the stages of development during
which certain kinds of knowledge emerged\cite{Piaget}, including
logical-mathematical knowledge related to the kind we wish to foster.
Additionally, by using our devices as an assessment vehicle for children's
spatial reasoning, one can position our work as part of a tradition (dating back
at least to Piaget \cite{piaget1967child}) in understanding spatial thinking and
its development (cf. also \cite{newcombe2003making} for a more recent treatment
of the subject). While Piaget's specific theories have been strongly
challenged\cite{Esther}\cite{Repacholi}, his influence was (and is!) extremely
important. Seymour Papert, one of Piaget's intellectual descendants, published
Mindstorms\cite{mindstorms} in 1980 and with it introduced his own ideas about
constructivism. Combined with the advent of the physical Logo turtle, Papert
brought many constructivist ideas into the modern age and opened the door for a
technical and cognitive exploration of how computation and interactive objects
could be combined to examine the link between tangibles and children's learning.

% It should also be noted, along these lines, that our early pilot test experience
% suggested a use for our work as an assessment vehicle for children's spatial
% cognition. We have, for instance, given children a pattern of lights
% and ask them to match that pattern to one of a set of physical or pictorial
% solid representations as well as ask children to recreate a variety of
% physical solids (such as a plastic prism or tetrahedron) by selecting the
% appropriate set of lights, and note their development and difficulties in doing
% so. By using the UCube as an experimental device in this fashion, one can
% position this work as part of a tradition (dating back at least to Piaget
% \cite{piaget1967child}) in understanding spatial thinking and its development
% (cf. also \cite{newcombe2003making} for a more recent treatment of the subject).

While a rich and diverse lineage of tangible and embedded user interfaces has
progressed since (and partially because of) Papert, the genealogy of the
proposed work derives from an interest not only in constructivist-like
activities, but in theories about how interaction with physical objects may be
beneficial to learning. In cognitive science, the area of embodied cognition
examines the ways in which our interactions with the physical world shape our
cognitive experiences from a body-centric point of view. More specifically,
embodied cognition holds that our cognitive processes are `deeply rooted in the
body's interactions with the world'\cite{Wilson}. This is in stark contrast to
decades of research in cognitive science wherein the mind was viewed as a sort
of central but detached information processing unit where motor-sensory
functions were more-or-less secondary inputs and outputs to a main
system\cite{clark1998being}.

Although there are several different tenets of this body-centric view, the
primary conclusion relevant to our proposal is that interactions with physical
objects can shape, clarify, and reinforce our cognitive processes in scores of
disparate areas. Of keen interest for this work in particular is a domain
referred to as embodied mathematics.
Lakoff and Nu\~nez\cite{lakoff} give a fascinating account of the origins of
mathematics from an embodied point of view. They propose that humans, by virtue
of their interactions with the physical world, inevitably form certain
intuitions of a mathematical nature. Recognizing small numbers of objects (e.g.
the pre-verbal ability to do arithmetic with less than five objects),
estimation, and simple comparisons are a few of the examples given
in\cite{lakoff}. From these basics, they argue that four kinds of physical
operations (object collection, object construction, using a measuring stick, and
movement along a path) form the basis of simple arithmetic. Although the book
postulates about concepts as ungrounded and seemingly abstract as infinity, for
our work it is enough to suggest that the interactions present in our designs
follow from these four operations and may in fact contribute to the
solidification of more complex mathematical ideas in 3D modeling and digital
fabrication (e.g. forming correct mental models of 3-dimensional objects). Such
notions of embodied mathematics have�even before the Lakoff/Nu\~nez text�played
a role in discussions of the development or instruction of mathematical ideas.
The link between physical experience and mathematical growth was a strong
element, for instance, in Montessori's work (see, e.g.,
\cite{hainstock1978essential}); much of the motivation behind traditional
mathematical ``manipulatives'' such as number rods and balance beams can also be
traced to this intellectual tradition. More recently, theoretical discussions of
embodied cognition have given rise to fine-grained observations of the
connections between bodily activity and mathematical learning:
Goldin-Meadow\cite{goldin2005hearing}, for instance, describes a fascinating
line of research in which children's nonverbal gestures appear to both reflect
and, in some cases, anticipate their verbal understanding of concepts such as
conservation and ``inverse operations''. In other work, Ehrlich, Levine and
Goldin-Meadow show that through an analysis of hand gestures, one is not only
able to predict a subject's `readiness' to learn mathematical
concepts\cite{goldin} but that the kinds of gestures children make (those
relating to movement, for example) are correlated with spatial reasoning
ability\cite{ehrlich2006importance} and performance on mental transformation
tasks.
 
Pedagogical research in embodied mathematics has, moreover, proceeded
hand-in-hand with the development of desktop, embedded, or portable
technological artifacts to support the link between bodily actions and
mathematical conceptualization. Papert's discussions of the Logo computer
language \cite{mindstorms} reveal this connection early in the history of
children's computing: Papert discussed, for example, the way in which the
program for a Logo circle resonated with children's bodily understanding of
moving in a circular path. More recently, Nemirovsky et al.
\cite{nemirovsky1998body} describe the use of a computer-based motion detector
system to assist children in the development of intuitions behind graphing;
Howison et al. \cite{howison2011mathematical} used a device based on a
Nintendo Wii remote to assess children's understanding of ratio (the children
attempt to move their arms in a manner illustrating a target ratio); Bakker et al.
\cite{bakker2011moso} created a collection of handheld objects (``MoSo
Tangibles'') with embedded sensors to help children learn about musical ideas
via hand motions such as waving, squeezing (pressing hands together), and
shaking up and down, among others; Mickelson and Ju \cite{mickelson2011math} use
sophisticated video and projection equipment as the basis of activities through
which children can learn about mathematical ideas (e.g., symmetry, rotation
angles) via large-scale physical movements.
 
In their section on `Thinking Through Doing', Klemmer et
al.\cite{Klemmer:2006:BMF:1142405.1142429} give a particularly poignant summary
of why we ought to consider the body as instrumental in any human-computer
interaction design, stepping through many of the concepts outlined above. In
fact, the marriage of ideas derived from Papert's work with the conclusions of
embodied cognition are not new, and appear to substantiate our motivations to
produce tangible, manipulative interfaces as opposed to purely 2-dimensional
screen-based work. In the mid-to-late 1990's, research examining the ways in
which physical objects might be infused with computational ability started to
coalesce around several themes\cite{Eisenberg:1996:RMV:257089.257230}. Resnick's
work with ``digital
manipulatives"\cite{Resnick:1998:DMN:274644.274684}\cite{Zuckerman:2005:ETI:1054972.1055093}
specifically references the contributions of Froebel and Montessori in the
design of a series of ``programmable bricks'' with computational ability whose aim
is to make certain specific concepts (e.g. systems-level thinking) more salient
for the user. Ishii's work on breaking down the divide between physical and
virtual worlds into `tangible
bits'\cite{Ishii:1997:TBT:258549.258715}\cite{Ishii:2008:TBB:1347390.1347392}
has subsequently set the stage for a new family of tangible interface designs
that support the kind of embodied interactions that our work seeks to produce.
By constructing environments and artifacts that focus on the possible physical
representations of computational components, these works (among others) created
the philosophical space to delve into how tangible objects might affect users at
a cognitive level. Our proposal is a confluence of both tangible and cognitive
design; as Resnick states, `We are interested in Things That Think only if they
also serve as Things To Think With'\cite{Resnick:1998:DMN:274644.274684}.

Having shown several PopCAD prototypes in Chapter 2 representative of a
``renaissance'' in papercrafting by infusing it with electronics, it is worth
situating that work in relation to that of other researchers in this (still
embryonic) field. The blending of traditional papercrafts with emerging
technology is in fact still a relatively novel technique, but there is a
remarkable community of researchers beginning to explore this area. For us, a
special debt is owed to Leah Buechley's High-Low Technology group at the MIT
Media Lab; that group first (to our knowledge) introduced conductive ink and
copper tape into paper-based projects. Early (c. 2008) use of conductive ink
with microcontrollers on a paper substrate can be found in
\cite{buechley2009paints} and \cite{Eisenberg:2009:CPR:1551788.1551790} with the
development of paper-based Arduino processors and simple electronic components
(e.g. LEDs, toy motors, switches) that could be placed onto conductive paint to
form an electronic connection. This work culminated with a paper application
usually reserved for home remodeling: a ``living wallpaper''\cite{Buechley2010}
where passers-by could trigger light, movement, and sound by interacting with
different parts of the surface (see Figure \ref{fig:paperCrafts}).

\begin{figure}[ht]
\begin{center}$
\begin{array}{cc}
\includegraphics[width=.45\linewidth]{images/elect_popables}&
\includegraphics[width=.45\linewidth]{images/living_wall}
\end{array}$
\end{center}
\caption{Examples of paper-based electronics: Electric Popables (left) is a
pop-up book infused with a variety of paper-friendly electronics. The Living
Wall (right) is a complete interactive environment embedded in wallpaper,
reacting with light, sound, and movement.}
\label{fig:paperCrafts}
\end{figure}


These early efforts in turn spawned developments that further refined the
expressive potential of paper-based electronics, infusing traditional
papercrafts with new elements and abilities. An electronic pop-up book by Qi and
Buechley\cite{Qi:2010:EPE:1709886.1709909} re-imagined the traditional pop-up by
infusing each page with paper-friendly, interactive circuitry (e.g. by using a
copper tape circuit to power LEDs in a pop-up cityscape), and from which PopCAD
certainly owes some debt. Other projects in this vein include techniques to
animate origami structures through shape-memory alloy (SMA)\cite{Qi2012}, using
SMAs in the design and fabrication of printable paper-based devices (e.g.
speakers and lamps)\cite{Saul2010a}, storytelling and craft-making through
electronically-enhanced storybooks and workshops
\cite{Jacoby2013a}\cite{Buechley2012}\cite{Sylla2012} and the use of small
microcontrollers incorporated into programmable paper-based
sculptures\cite{Mellis:2013:MMC:2460625.2460638}.

These efforts have focused on the creation of compelling(either electronically
or digitally enhanced) papercrafts. As noted in the introduction, there are
numerous technological developments that, in combination, serve to accelerate
the development of paper mechatronics. For instance, Kawahara et
al.\cite{Kawahara2013} describe how inkjet-ready conductive ink can allow
circuits to be printed easily and directly onto paper; and Koizumi et
al.\cite{Koizumi2010a} present a toolkit for wireless control of movable paper
toys, Zhu et al.\cite{Zhu2011a} describe a method for wireless power transfer
for paper computing, and Coelho et al.\cite{Coelho2009} have achieved the direct
embedding of conductive components during the papermaking process.

Of particular interest for the current work are explorations focusing on 3D
modeling and perception with tangible interfaces. Prime examples include
software that allows for 3D shapes to be flattened into paper-printable,
origami-esque polyhedra\cite{Eisenberg:1997:HUT:238218.238312}, a construction
kit with kinetic memory so as to record and playback certain user-generated
manipulations\cite{Raffle:2004:TCA:985692.985774}, as well as several variations
of ``smart-cube'' interfaces
\cite{Watanabe:2004:SAI:1037851.1037874}\cite{Schweikardt:2006:RRC:1180995.1181010}
that encourage spatial and logical reasoning in order to make use of the
computational aspects of the cubes. While diverse in their implementation, these
kits point to ways in which interface design can tease out the kind of
3-dimensional problem-solving and exploration present in the proposed work.

\begin{figure}[ht]
\begin{center}$
\begin{array}{cc}
\includegraphics[width=.54\linewidth]{images/activecube}&
\includegraphics[width=.37\linewidth]{images/roblocks}
\end{array}$
\end{center}
\caption{Left: The ActiveCube system. Right: The Roblocks system.}
\label{fig:activecube}
\end{figure}

Related contributions focus more on the cognitive processes involved when
exploring embodied interfaces with children. Research on supporting creative
problem solving with children\cite{Bevans:2011:SCC:1979742.1979838}, arguing for
a kindergarten-influenced approach to creative
thinking\cite{Resnick:2007:IRN:1254960.1254961}, embodied approaches to
analyzing children's interactions with smart
objects\cite{Antle:2009:THE:1520340.1520612}, as well as the embodied design of
interfaces for introducing mathematical concepts to
kids\cite{Abrahamson:2011:TED:1999030.1999031} have shown a great degree of
correlation between physical interaction and learning in children.


\begin{figure}[ht]
\begin{center}$
\begin{array}{cc}
\includegraphics[width=.40\linewidth]{images/interactiveLaser}&
\includegraphics[width=.54\linewidth]{images/interactiveFab}
\end{array}$
\end{center}
\caption{Examples of interactive fabrication interfaces: Constructable (left)
allows users to control a laser cutter with a set of physical tools as opposed
to a pre-defined design file. Shaper (right), and interactive fabrication tool
using expanding polyurethane foam.}
\label{fig:interactiveFab}
\end{figure}



\begin{figure}[ht]
\begin{center}$
\begin{array}{cc}
\includegraphics[width=.45\linewidth]{images/kidcad}&
\includegraphics[width=.45\linewidth]{images/easigami}
\end{array}$
\end{center}
\caption{Left: The KidCAD interface showing a model Zebra and its 2.5D
impression on screen.
Right: The Easigami system, showing a series of connected polygonal faces with
smart-hinges and embedded electronics.}
\label{fig:kidcad}
\end{figure}

Yet so far, there have been few attempts to design embodied interfaces for
children that specifically address the growing presence and availability of
digital fabrication tools.
KidCAD\cite{Follmer:2012:KDR:2207676.2208403}, a deformable pad that captures
the 2.5D geometry of depressions made on the underside of the surface, was a
very promising idea in that it allowed very young children to take small objects
from their surroundings (or their hands) and `stamp' them into the pad - an
intuitive and satisfying experience. Unfortunately, the authors intentions to be
able to output the geometry to 3D printers has not yet manifested.
Easigami\cite{Huang:2012:EVC:2148131.2148143} is a set of interchangeable and
interlocking polyhedral faces with smart `hinges' that can reproduce the
morphology of a set of connected faces while connected to a computer. In
contrast, Easigami \emph{is} able to export this morphology to a
stereolithography file ready for 3D printing. There are several other interfaces
that deal with `interactive fabrication'\cite{Willis:2010:IFN:1935701.1935716};
devices that manipulate materials interactively based on various input from a
user, such as controlling a laser cutter with a laser pointer (instead of
through a CAD program)\cite{Mueller:2012:ICI:2380116.2380191}, or a wearable
device that takes in a CAD file and provides haptic feedback to make the
physical creation of the device by hand easier, even for a
non-fabricator\cite{Zoran:2013:FFD:2470654.2481361}.
These projects, as well as several others that deal specifically with digital
fabrication for laser
cutting\cite{Johnson:2012:SMS:2212776.2212390}\cite{Willis:2010:SSB:1709886.1709890},
are examples of the subset of tangible interfaces to which this work belongs -
namely, those concerned with providing a means to engage with digital
fabrication technologies in a more intuitive, embodied fashion. However, with
the exception of KidCAD and Easigami these designs are not made with children in
mind, nor do they cover the range of possibilities for child-friendly input
devices that focus on 3D-printing. Thus, we argue that there is room in this
area for the work described in the thesis, as well as a lineage that suggests
meaningful results may follow from continuing to explore the incorporation of
tangible interfaces with embodied design.

Specifically, we see our devices as part of a larger, burgeoning ``technological
ecosystem'' around the activity of three-dimensional printing. The introduction
chapter to this work noted several prominent researchers who argue for the
democratization of this technology, and for its applications to education.
Indeed, exciting early work has been done in applying 3D printing to education
in fields such as architecture \cite{breen2003tangible}, solid geometry
\cite{hart2008procedural}, and mechanical design \cite{lipson20053}.
Our devices are specifically designed so that they can be employed by younger
students � younger, for instance, than the typical (undergraduate-age)
architecture student - and certainly less skilled or experienced with
traditional 3D modeling software. The devices were created to enable children to
specify and identify three-dimensional shapes by hand motions (instead of, by
contrast, using symbolic commands directed at a two-dimensional screen display).
At the same time, they are not simply devices for mathematical instruction,
nor even a general tool for mathematical design - but as a suite of
experiential, embodied interfaces for engaging youth in a variety of spatial
design activities aimed not only at learning but at democratizing authorship
for 3D printing as well.


% It should also be noted, along these lines, that our early pilot test experience
% suggests a potentially fruitful use for the UCube as an assessment device for
% children's spatial cognition. (The young subject who suggested that it could be
% made into a "puzzle game" is anticipating our thoughts here!) A researcher
% could, for instance, give children a pattern of lights and ask them to match
% that pattern to one of a set of physical or pictorial solid representations; or
% one might ask children to recreate a variety of physical solids (such as a
% plastic prism or tetrahedron) by selecting the appropriate set of lights, and
% note their development and difficulties in doing so. By using the UCube as an
% experimental device in this fashion, one can position this work as part of a
% tradition (dating back at least to Piaget \cite{piaget1967child}) in
% understanding spatial thinking and its development (cf. also
% \cite{newcombe2003making} for a more recent treatment of the subject).

% It should also be noted, along these lines, that our early pilot test experience
% suggests a potentially fruitful use for the UCube as an assessment device for
% children's spatial cognition. (The young subject who suggested that it could be
% made into a "puzzle game" is anticipating our thoughts here!) A researcher
% could, for instance, give children a pattern of lights and ask them to match
% that pattern to one of a set of physical or pictorial solid representations; or
% one might ask children to recreate a variety of physical solids (such as a
% plastic prism or tetrahedron) by selecting the appropriate set of lights, and
% note their development and difficulties in doing so. By using the UCube as an
% experimental device in this fashion, one can position this work as part of a
% tradition (dating back at least to Piaget \cite{piaget1967child}) in
% understanding spatial thinking and its development (cf. also
% \cite{newcombe2003making} for a more recent treatment of the subject).



% from IDC 2011
% There are several strands of research that have strongly influenced the design
% (and motivation) for the UCube. Perhaps the most fundamental of these is in the
% area of "embodied mathematics"� that is, the notion that mathematical thinking
% and learning are affected by, and perhaps grounded in, metaphors derived from
% bodily experience. The most thorough and discursive (though largely theoretical)
% discussion of these ideas is in the foundational text by Lakoff and Nu�ez
% \cite{lakoff}:
% the authors discuss physically-derived metaphors that underlie such essential
% mathematical ideas as numbers, operations, and sets. Such notions of embodied
% mathematics have�even before the Lakoff/Nu�ez text�played a role in discussions
% of the development or instruction of mathematical ideas. The link between
% physical experience and mathematical growth was a strong element, for instance,
% in Montessori's work (see, e.g., \cite{hainstock1978essential}); much of the
% motivation behind traditional mathematical ``manipulatives'' such as number rods
% and balance beams can also be traced to this intellectual tradition. More
% recently, theoretical discussions of embodied cognition have given rise to
% fine-grained observations of the connections between bodily activity and
% mathematical learning: Goldin-Meadow\cite{goldin2005hearing}, for instance,
% describes a fascinating line of research in which children's nonverbal gestures
% appear to both reflect and, in some cases, anticipate their verbal understanding
% of concepts such as conservation and ``inverse operations''.


% Pedagogical research in embodied mathematics has, moreover, proceeded
% hand-in-hand with the development of desktop, embedded, or portable
% technological artifacts to support the link between bodily actions and
% mathematical conceptualization. Papert's discussions of the Logo computer
% language \cite{mindstorms} reveal this connection early in the history of
% children's computing: Papert discussed, for example, the way in which the
% program for a Logo circle resonated with children's bodily understanding of
% moving in a circular path. More recently, Nemirovsky et al.
% \cite{nemirovsky1998body} describe the use of a computer-based motion detector
% system to assist children in the development of intuitions behind graphing;
% Howison et al. \cite{howison2011mathematical} used a device based on a Wii
% remote to assess children's understanding of ratio (the children attempt to move
% their arms in a manner illustrating a target ratio); Bakker et al.
% \cite{bakker2011moso} created a collection of handheld objects (``MoSo
% Tangibles'') with embedded sensors to help children learn about musical ideas
% via hand motions such as waving, squeezing (pressing hands together), and
% shaking up and down, among others; Mickelson and Ju \cite{mickelson2011math} use
% sophisticated video and projection equipment as the basis of activities through
% which children can learn about mathematical ideas (e.g., symmetry, rotation
% angles) via large- scale physical movements.
% The development of the UCube follows within this tradition, in that the device
% was created to enable children to specify and identify three-dimensional shapes
% by hand motions (instead of, by contrast, using symbolic commands directed at a
% two-dimensional screen display). At the same time, the UCube is not simply a
% device for mathematical instruction, but is more generally a tool for
% mathematical design. As noted at the outset of this paper, the intent of the
% UCube is to enable youngsters not only to learn about but also to build
% mathematical shapes.
% Specifically, we see the device as part of a larger, burgeoning ``technological
% ecosystem'' around the activity of three-dimensional printing. The first section
% of this paper noted several prominent researchers who argue for the
% democratization of this technology, and for its applications to education.
% Indeed, exciting early work has been done in applying 3D printing to education
% in fields such as architecture \cite{breen2003tangible}, solid geometry
% \cite{hart2008procedural}, and mechanical design \cite{lipson20053}. The UCube
% is designed so that it can be employed by younger students�younger, for
% instance, than the typical (undergraduate-age) architecture student. At the same
% time, we see no reason at all why the device could not be used by adult or
% professional-level students�particularly if (as we anticipate) the device and
% software are made more expressive or powerful in future iterations.
% It should also be noted, along these lines, that our early pilot test experience
% suggests a potentially fruitful use for the UCube as an assessment device for
% children's spatial cognition. (The young subject who suggested that it could be
% made into a "puzzle game" is anticipating our thoughts here!) A researcher
% could, for instance, give children a pattern of lights and ask them to match
% that pattern to one of a set of physical or pictorial solid representations; or
% one might ask children to recreate a variety of physical solids (such as a
% plastic prism or tetrahedron) by selecting the appropriate set of lights, and
% note their development and difficulties in doing so. By using the UCube as an
% experimental device in this fashion, one can position this work as part of a
% tradition (dating back at least to Piaget \cite{piaget1967child}) in
% understanding spatial thinking and its development (cf. also
% \cite{newcombe2003making} for a more recent treatment of the subject).


%From IDC 2014 on paper-based electronics
% Having shown several representative prototypes of our own work in paper
% mechatronics, it is now worth situating that work in relation to that of other
% researchers in this (still embryonic) field. The blending of traditional
% papercrafts with emerging technology is in fact still a relatively novel
% technique, but there is a remarkable community of researchers beginning to
% explore this area. For us, a special debt is owed to Leah Buechley's High-Low
% Technology group at the MIT Media Lab; that group first (to our knowledge)
% introduced conductive ink and copper tape into paper-based projects. Early (c.
% 2008) use of conductive ink with microcontrollers on a paper substrate can be
% found in \cite{buechley2009paints} and \cite{Eisenberg:2009:CPR:1551788.1551790}
% with the development of paper-based Arduino processors and simple electronic
% components (e.g. LEDs, toy motors, switches) that could be placed onto
% conductive paint to form an electronic connection. This work culminated with a
% paper application usually reserved for home remodeling: a ``living
% wallpaper''\cite{Buechley2010} where passers-by could trigger light, movement,
% and sound by interacting with different parts of the surface.
% 
% 
% These early efforts in turn spawned developments that further refined the
% expressive potential of paper-based electronics, infusing traditional
% papercrafts with new elements and abilities. An electronic pop-up book by Qi and
% Buechley\cite{Qi:2010:EPE:1709886.1709909} re-imagined the traditional pop-up by
% infusing each page with paper-friendly, interactive circuitry (e.g. by using
% a copper tape circuit to power LEDs in a pop-up cityscape). Other projects in
% this vein include techniques to animate origami structures through shape-memory alloy
% (SMA)\cite{Qi2012}, using SMAs in the design and fabrication of
% printable paper-based devices (e.g. speakers and lamps)\cite{Saul2010a},
% storytelling and craft-making through electronically-enhanced storybooks and
% workshops \cite{Jacoby2013a}\cite{Buechley2012}\cite{Sylla2012} and the use
% of small microcontrollers incorporated into programmable
% paper-based sculptures\cite{Mellis:2013:MMC:2460625.2460638}.
% 
% These efforts have focused on the creation of compelling(either electronically
% or digitally enhanced) papercrafts. As noted in the introduction, there are
% numerous technological developments that, in combination, serve to accelerate
% the development of paper mechatronics. For instance, Mueller et
% al.\cite{Mueller:2012:ICI:2380116.2380191} describe the use of a laser cutter to
% produce origami figures; Kawahara et al.\cite{Kawahara2013} describe how
% inkjet-ready conductive ink can allow circuits to be printed easily and directly
% onto paper; and Koizumi et al.\cite{Koizumi2010a} present a toolkit for wireless
% control of movable paper toys, Zhu et al.\cite{Zhu2011a} describe a method for
% wireless power transfer for paper computing, and Coelho et al.\cite{Coelho2009}
% have achieved the direct embedding of conductive components during the
% papermaking process.

% These last efforts are powerful examples of expanding techniques--they signal
% the emergence of a new territory within which to explore paper-based electronics.
% Our own prototypes are intended to continue this communal development of
% techniques and examples, but there are several factors that distinguish our work
% from that of other efforts. First, several of our own prototypes (e.g., PopCAD,
% the bicycle rider, and the cherry blossom painting) may be seen as incorporating
% paper elements as portions of larger, composite systems. PopCAD is a paper-based
% input device; one might think of it as one early foothold in an unexplored
% landscape of paper input devices for children's activities. The bicycle rider is
% an artifact that combines a desktop computer screen with a paper model; again,
% one could think of it as an exemplar of blending papercrafts with (e.g.)
% high-resolution graphics, or one could imagine websites designed to work as
% ``background graphics'' for electronically-controlled paper constructions. The
% cherry blossom painting is (in a sense) the ``flip side'' of PopCAD; whereas
% PopCAD is a paper-based input device, the cherry blossom painting is a
% paper-based display for output. And once more, one could take the example still
% further: paper mechanisms or models could be moved or controlled as components
% of extended output displays that combine physical and screen-based elements.

% More generally, we see our prototypes as (still-early) pointers toward a new
% genre of activities for children. In the final section, we turn our attention to
% the mechatronic future of children's papercrafts.







\chapter{Technical Implementation}
\label{technical}
\chapter{Evaluation}
\label{evaluation}

This section is devoted to the description and discussion of three separate user
studies with the devices discussed in Chapter 2. Two studies were performed with
the original UCube device (one more informal than the other), while a longer
study involved both the SnapCAD and PopCAD systems.

\section{UCube Pilot}
\subsection{Procedure}
Early in 2011, we conducted an initial (and informal) pilot test of the UCube
with a group of 12-14 year olds. Fourteen participants, consisting of five girls
and nine boys, were divided into six groups (five groups of two, one group of
four). Participants were asked to model a sequence of five shapes of increasing
complexity using the UCube along with the companion software. The target shapes
were displayed on one half of a computer screen, while the UCube software
showing the live model was displayed on the other half as in
\autoref{fig:ucube_test1}.
The first shape that participants were asked to model was a straight vertical
line; after this, the requested shapes were a diagonal line, a cube, a
triangular prism, and finally an irregular polyhedral object. No shape required
more than four towers to complete, and shapes were always presented in the same
order.

\begin{figure}[h]
\begin{center}$
\begin{array}{cc}
\includegraphics[width=.8\linewidth]{images/ucubescreentest}
\end{array}$
\end{center}
\caption{A screenshot of the testing setup, with the live output from the UCube
on the right and the target shape on the left.}
\label{fig:ucube_test1}
\end{figure}

Participants were instructed to place the poles on the board (but not shown
how), and were told that the software model could be rotated and filled in using
the keyboard and mouse, should that help them complete the task. The
participants were not given any hints as to how to complete the shapes and were
not told when they had the correct configuration (they had to indicate their
belief that the model was done). Participants were also instructed to 'think
aloud' about their actions. The main purpose of the pilot study was to get an
initial impression of how the UCube would act as an accessible 3D modeling
tool�how well it could help ``3D novices'' overcome the ``2D bottleneck''.

\subsection{Results and Discussion} 
Of the six groups who participated, four groups successfully modeled all five
shapes, one group ran out of time after three shapes, and one group finished one
shape. Sessions lasted between 17 and 30 minutes.
A variety of problem-solving strategies were observed during testing, as the
participants tended to treat the exercise as a sort of puzzle to be solved.
Simple methods equivalent to ``try and see'' were common, and seemed to serve as
a base point from which to draw conclusions about the relationship between the
3D model and 2D on-screen representation (e.g. ``No, not there, up one''). More
sophisticated strategies were also observed� ``deconstructing'' more complex
shapes into smaller, easier-to- model shapes (e.g. thinking of one side of a
cube as a square) was observed from several groups. Another popular technique
was to systematically match the on-screen perspective from the live model with
the shape they were attempting to model (e.g. ``Okay, first let's do the top
view, and then go from the side''). By orienting the two models similarly,
participants were able to make more accurate modeling decisions as well as check
their model against the on-screen shape. Counting distance in terms of spaces on
the board, between switches, or between dots on the screen was also a very
common technique of reasoning about and describing position. For example, by
counting that two vertices of a shape were separated by ``two dots over and one
down'' on the screen, subjects were able to count the distance out on the
physical UCube board. A few of the more mathematically-advanced participants
used terms such as ``axis'' and ``origin'' to orient themselves and describe
various positions on the board to their partners.
Another revealing observation in the pilot study was that, in the few instances
of mechanical failure (certain switches not lighting up, towers not plugging in
properly, or points not showing up on screen) the participants were still able
(with a high degree of certainty) to complete the assigned tasks. This appears
to indicate that, as opposed to arbitrarily moving the towers around until the
two sides of the computer screen looked the same, participants had formed a more
substantial mental model of the relationship between the UCube interface and the
2D representations on the screen. That opens the possibility that by performing
the embodied interactions necessary to operate the UCube, participants had
actually strengthened their understanding of how 3-dimensional space is
typically represented on a 2D screen. Although further testing and observation
is needed, this finding would strengthen the argument for using the UCube in an
educational setting to improve understanding of 3D space, as well as providing a
gateway for youngsters to move on to more complex modeling software.
While the variety of problem-solving techniques we witnessed is a testament to
the participants' ingenuity, it is also indicative of the fact that parts of the
UCube are not immediately intuitive. While none of the participants had trouble
understanding how to place the towers on the platform, the positions of the
towers and switches had to be reasoned out explicitly. It was common for groups
to clear the board of any poles when starting a new shape, even in cases where
an overlap of points or tower positions existed. (Figure XXXX, for example�shown
earlier in the context of explaining the UCube's operation�depicts one of the
students placing a tower and checking the screen to see whether the tower
placement is appropriate.) Although most groups completed all the shapes (or ran
out of time), there were some expressions along the way of the difficulty of the
task (e.g. ``This is hard'', or ``This is like a puzzle''). This indicates that
design changes can be made in future iterations to help clarify the
correspondence between positions on the UCube platform and the on-screen
representation; for example, labeling the both the physical and software grid
with a simple alphanumeric system.
Despite these drawbacks as well as the inherent limitations of the UCube design,
these early results indicate a promising ability of youngsters to effectively
engage with the UCube interface. In fact, despite various levels of success in
completing the assigned tasks, the vast majority of participants exhibited a
high level of engagement with the UCube. For example, although the group that
completed only one shape seemed unmotivated to attempt to model the other
shapes, they continued to play with the interface and observe the results, even
stating ``this is fun'' and ``I like the switches''. Participants also saw
potential uses for the UCube outside of the specific exercise we assigned.
Comments (unsolicited) included, ``you should use this to teach geometry'' and
``you could make this a puzzle game''.
At the very least, these early results indicate that the majority of
participants were able to take a 2-dimensional representation on the screen and
model its 3-dimensional equivalent using the UCube, a very encouraging result in
our eyes.

\section{Further UCube Study}

Early in 2012, we conducted a further user study of the UCube with a group of
11-13 year olds. The group consisted of ten participants, eight boys and two
girls, from a local middle school multimedia class. Every participant was
individually led through two separate exercises (outlined below) using the
UCube. 

\subsection{Procedure: Modeling}
Participants were handed a 3D-printed shape (modeled and printed from the UCube)
and were instructed to attempt to model the shape using the UCube. The
participant was initially allowed to hold the shape for approximately 10
seconds, after which they would hand the shape back to the facilitator and
attempt to model the shape from memory. Participants were instructed that they
may ask to hold the shape again, at which point they were allowed to hold it
throughout the duration of the modeling task. Additionally, users were
instructed that they had the option to skip a shape and return to it at a later
point in the exercise.
The five physical shapes presented were: a cube, a tetrahedron, a diamond, a
``house'' (a cube with a pyramid on top), and a complex irregular polyhedron.
The models were presented to the user starting with the cube (as this was deemed
to be the most basic shape with regard to modeling complexity). To avoid an
ordering bias, we randomized the presentation sequence of the next four shapes
using an online random order generator. If, after skipping a shape and returning
to it, the participant was still having difficulty, we offered them the
opportunity to attempt modeling the shape with the help of the UCube software,
the effects of which are discussed in the results section. Participants were
given a total of 25 minutes for the modeling exercise. We recorded, but did not
limit the modeling time per shape, only the total time for all five shapes.

\subsection{Procedure: Matching}
Participants were instructed to face away from the UCube while the facilitator
modeled a set of lights on the UCube corresponding to one shape among a set of
physical models laid out on the table next to the UCube.
Once the lights on the UCube were set up, the participant was instructed to turn
around, and indicate which physical object they thought the set of lights on the
UCube corresponded to.
There were nine physical models presented on the table, and consisted of a cube,
a tetrahedron, the �house� shape, a diamond, a triangular prism, an elongated
hexagon, a parallelogram, a trapezoid, and an irregular polyhedron (see
\autoref{fig:ucube_shapes} for a picture of all the models). The shapes were
always presented on the table in the same order and orientation to avoid
discrepancies in perception or association.
Of the nine shapes, the participants were asked to match five of them (the cube,
the triangular prism, the parallelogram, the elongated hexagon, and the
trapezoid). Thus, only the cube was presented in both the matching and modeling
exercises. As with the modeling exercise, the cube was presented first, with the
remaining four shapes presented in a computer-generated randomized
order.Participants were given a total of ten minutes for the matching exercise,
corresponding to two minutes per shape, and were instructed to think aloud
during the process.

\begin{figure}[h]
\begin{center}$
\begin{array}{cc}
\includegraphics[width=.8\linewidth]{images/ucube_shapes}
\end{array}$
\end{center}
\caption{The nine 3D-printed models used in the modeling and matching tasks
described in this section.}
\label{fig:ucube_shapes}
\end{figure}


\subsection{Results} 
While many established forms of 3D modeling systems can be confounding and
operationally too complex for a child to navigate, the UCube was positively
received and system instruction was accomplished with just a minor introduction
and demonstration (system instruction and demonstration lasted approximately 2-3
minutes). We found this first instance of system comprehension to offer some
validation that the UCube worked well as a user-friendly 3D modeling device.
This section will detail the outcome of both the modeling and matching tasks
performed.

\subsubsection{Exercise 1: Modeling} 
Modeling occurred under three conditions:recreate the object from memory,
construction of the object while it was in the participant�s possession, and
modeling the shape with the help of the UCube software. Overall, 21 of 50 shapes
were completed from memory, 12 of 50 were completed while holding the shape, and a
further 8 of 50 were completed with the aid of the UCube software, for a total
of 41 out of 50 shapes modeled successfully (82\%). Of the nine missed shapes,
seven were of the same shape, the complex polyhedron. The remaining two misses
were from the same participant, who ran out of time before completion.
Of the 10 participants, 8 were able to recreate the cube from memory, whereas
only 4 were able to recreate the diamond and the tetrahedron from memory. Half
of the participants constructed the house from memory, and no participants were
able to complete the irregular polyhedron from memory. However, once shown the
software the majority of the participants found the modeling task significantly
easier to perform. The irregular polyhedron was by far the hardest shape and was
only able to be completed by 3 of the 10 participants either after continued
possession of the shape or using the software.

\begin{figure}[h]
\begin{center}$
\begin{array}{cc}
\includegraphics[width=.45\linewidth, height=1.75in]{images/modeling1}&
\includegraphics[width=.45\linewidth, height=1.75in]{images/modeling2}
\end{array}$
\end{center}
\caption{Results of the modeling task, showing total modeling time spent per
participant (left) and average modeling time spent per shape across
participants (right).}
\label{fig:modeling}
\end{figure}


\autoref{fig:modeling} represents the total completion times per participant (on
the left) and average time per shape (right). Two exceptional completion times
were observed, where participants finished modeling all the shapes in under 10
minutes. However, the majority of participants finished the task in the 19-25
minute range. Only one of the participants ran out of time. Once participants
had been introduced to the software, 9 of 10 of participants were able to
complete all but the irregular polyhedron. It is interesting to note that of the
10 participants, the child that had the most difficult time modeling, the lowest
shape completion rate, and the longest completion time during the matching
exercise was the youngest participant.


\subsubsection{Exercise 2: Matching} 
Out of 50 matching tasks (five per participant), all but three tasks were
completed in 20 seconds or less. \autoref{fig:matching} displays the total time
spent on the matching task per participant (left) and the average completion
times for each shape (right).
No participant selected the wrong shape (a few preliminary ``mis-selections''
were made that the participants quickly corrected), and all participants
completed the task in well under the allotted 10 minutes. The lack of errors in
the matching task is highly encouraging as a basis from which to reason about
youngsters' abilities to perceive and reason about convex hulls as a set of lit
vertices in space, meaning that this kind of 3D modeling interface might be
applied to other domains (e.g., as a cognitive assessment tool, a puzzle game,
etc.) with some optimism.

\begin{figure}[h]
\begin{center}$
\begin{array}{cc}
\includegraphics[width=.45\linewidth, height=1.75in]{images/matching1}&
\includegraphics[width=.45\linewidth, height=1.75in]{images/matching2}
\end{array}$
\end{center}
\caption{Results of the matching task, showing total time spent per
participant (left) and average time spent per shape across
participants (right).}
\label{fig:matching}
\end{figure}


\subsubsection{Observations} 
Modeling trends as well as distinct modeling behaviors were documented in the
process. Common observations included building from the ground up (lowest
vertices first), building in the orientation that the object had been presented
in, not clearing the poles/lights from the UCube before starting to model a new
shape, and modeling a shape by breaking it up into discrete parts (e.g. a
participant building a house would commonly build a cube first and then add on a
vertex to the top; a participant constructing the diamond might combine two
opposite facing triangles.).

Unique behaviors were exhibited in the modeling process as well, reflecting a
type of user-specific construction-based problem- solving. One participant used
their arm to connect the red lights of the UCube for shape definition. A few
participants oriented the object differently than how it had been
presented�typically this occurred for the modeling of those objects with a
pyramidal apex (tetrahedron, house, diamond). Apex formation was perhaps one of
the most difficult concepts for most participants to grasp, as it required them
to strategically align the base on a 3x3 grid so there was a middle plug for
them to create the apex. If participants were fixated on designing from a 4x4
grid then there was no center plug for them to create a midpoint. Some
participants ended up building an oblong polyhedron as opposed to a cube, or an
oblique polyhedron as opposed to an equilateral tetrahedron. Other observed
behaviors included a participant who modeled shapes by turning on lights for an
entire shape edge, as opposed to just the corners and a participant who built
shapes that were floating, as opposed to resting on the base of the UCube.
There were also some notable behaviors regarding physical and gestural actions
of the participants. Many participants modeled with both hands simultaneously,
placing towers and flipping switches without a clear preference for a dominant
hand. Participants would often gesture with their arms following an arc in
parallel with a face of the object they were currently modeling. This �tracing�
behavior was also noticed when participants were holding a physical model and
tracing a side of the object with their fingertip, often while rotating the
object with the other hand. Finally, during object possession phase three
participants actually placed the 3D object on top of the UCube in the modeling
space while they reasoned out the construction (see
\autoref{fig:user_placedModel} for an example).

\begin{figure}[h]
\begin{center}$
\begin{array}{cc}
\includegraphics[width=.45\linewidth]{images/idc3}
\end{array}$
\end{center}
\caption{A participant modeling with the UCube, using a strategy of placing the
physical model on top of the UCube while modeling, as well as using both hands
simultaneously to manipulate the towers.}
\label{fig:user_placedModel}
\end{figure}



\section{SnapCAD and PopCAD}
The study will comprise several stages, the first being a pre-assessment of
spatial reasoning skills (all spatial reasoning assessment will be done using
the 'Children's Mental Transformation Task' designed by Susan Levine - see
http://silccenter.org/index.php/testsainstruments#MRT for the instruments, see
http://www.spatialintelligence.org/publications_pdfs/Ehrlich\%20Levine\%20\%20Goldin-Meadow\%20\%282006\%29.pdf
 p.1260-1261 for a good description of the study procedure.). After the
pre-assessment, participants will be split into two groups (~10 students each),
with group A modeling first on the PopCAD and group B modeling first on the
SnapCAD (a similar modeling device made from more rigid materials, and
representing a more expressive potential modeling space with a different means
of selecting points in space).
The evaluating of each device will consist of three modeling exercises,
evaluating the three main modeling modes in the software: convex hull, path, and
minimal spanning tree.  The basic operation and a brief explanation of each mode
will be given to the participants before each new modeling mode. Four 3D-printed
models representative of each mode will be presented to the user in a random
order, for a total of 12 modeling tasks. The random order shall be determined by
a computer program, and each task will be cut off at 10 minutes. Successful
completion (or lack thereof), time to completion, and observational notes shall
be recorded for each task. Participants will be asked to �think aloud� about
their process, difficulties, modeling choices, etc. Video will be recorded for
the purposes of analyzing user gestures when asked about how they arrived at a
certain solution.
A freehand activity will occur before and after the modeling tasks (10 minutes
each time) to help to get a sense of the kinds of shapes and modes used, the
overall usability of the interface, and a sense of the expressive range of the
relative point arrays, (PopCAD has a 3x3x3 equidistant grid of 27 individual
points, SnapCAD has a 7x7x7 grid of 343 points). By asking participants to think
aloud about their intentions and thinking processes, a deeper understanding may
be gained of the strengths and weaknesses of the system. Additionally, the
modeling modes used, relative time spent using each mode, and complexity (in
number of points) of the final object modeled will be recorded.
After the first set of modeling tasks, users will be given a mid-term assessment
using the same Mental Transformation Task as before (although different
individual problems will be presented). A week or so later, participants will
model on the device they have not used. Exercises will be consistent by device,
but may differ slightly between devices in order to tease out some of the
potential differences in each design. Once modeling on the second device is
completed, users will take final assessment to gauge if any improvement in
spatial reasoning skills has occurred throughout the study.
Around 20 participants will be enrolled in the study. Each individual session
should take 2 hours or less, and the overall duration of the study will be three
weeks. No control group will be used.

After the consent and assent forms are completed and the study session starts,
participants will be seated and will go through an explanation of the entire
study and what they will be asked to do. The pre-assessment will then be given,
consisting of 10 brief exercises where the participant attempts to match a
spatially transformed shape, printed on paper along with several incorrect
choices, to its original counterpart.

Participants will then be seated in front of a desk on which either the PopCAD
interface (a pop-up book) or SnapCAD (a tower and magnetic light based
interface) has been placed. The device will be hooked up to a laptop on the
desk. Participants will be given a brief demo of how to interact with the device
and how the software responds to those interactions.

The tasks that follow are the same for each device:

Task 1: Convex Hull Modeling

The participant will be given a brief demo of how the �convex hull� modeling
mode interprets the points from the device. The user will then be presented with
a series of four (4) plastic, 3D-printed models that were modeled on the device
using �convex hull� mode. The 4 shapes will be presented in a computer-generated
random order. For each of these shapes, the participant will attempt to recreate
the shape using the modeling abilities of the device. The user will be
instructed to indicate when they believe they are done, as well as to think
aloud about their modeling process. Each modeling task will be capped at 10
minutes. The time to completion (of lack thereof), observational notes, and
video shall be recorded.

Task 2: Path Modeling

The participant will be given a brief demo of how the �path� modeling mode
interprets the points from the device. The user will then be presented with a
series of four (4) plastic, 3D-printed models that were modeled on the device
using the �path� mode. The 4 shapes will be presented in a computer-generated
random order. For each of these shapes, the participant will attempt to recreate
the shape using the modeling abilities of the device. The user will be
instructed to indicate when they believe they are done, as well as to think
aloud about their modeling process. Each modeling task will be capped at 10
minutes. The time to completion (of lack thereof), observational notes, and
video shall be recorded.

Task 3: Minimal Spanning Tree Modeling

The participant will be given a brief demo of how the �minimal spanning tree�
(aka �tree�) modeling mode interprets the points from the device. The user will
then be presented with a series of four (4) plastic, 3D-printed models that were
modeled on the device using the �tree mode. The 4 shapes will be presented in a
computer-generated random order. For each of these shapes, the participant will
attempt to recreate the shape using the modeling abilities of the device. The
user will be instructed to indicate when they believe they are done, as well as
to think aloud about their modeling process. Each modeling task will be capped
at 10 minutes. The time to completion (of lack thereof), observational notes,
and video shall be recorded.

Task 4: Freehand Modeling

The freehand activity will occur before and after the modeling tasks and will
serve to get a sense of the overall usability of the interface as well as a
sense of the expressive range of each device. By asking participants to think
aloud about their intentions and thinking processes, a deeper understanding may
be gained of the strengths and weaknesses of the system. Additionally, the
modeling modes used, relative time spent using each mode, and complexity (in
number of points) of the objects modeled will be recorded.




\input{5_vision.tex}
\input{6_future_work.tex}
%%%%%%%%%   then the Bibliography, if any   %%%%%%%%%
\bibliographystyle{plain}	% or "siam", or "alpha", etc.
\nocite{*}		% list all refs in database, cited or not
\bibliography{refs}		% Bib database in "refs.bib"

%%%%%%%%%   then the Appendices, if any   %%%%%%%%%
\appendix

\input {appendixA.tex}
\chapter{Selected Transcriptions}

This appendix contains several illustrative transcriptions from the third user
study with the PopCAD and SnapCAD, where we recorded the speech and gesture
expressions of users explaining their modeling strategy. These excerpts contain
quotes from users generated while explaining their modeling strategy for a given
shape, the observed gestures that occurred during the spoken explanation, and
the speech and gesture codes generated from those expressions. A reminder of the
codes and examples used are provided in Table \ref{GcodingStrategy}.

\begin{table}[!ht]
\small
    \caption[Coding rubric for speech and gesture during user explanation of
    modeling strategy]{ The various coding strategies used in the video
    analysis of subjects' modeling strategy explanations. Borrowed and adapted
    from \cite{ehrlich2006importance}.}
    \begin{center}
    \begin{tabular}{| p{1.5cm} | p{4.2cm} | p{4.2cm} | p{4.2cm} |} \hline
	$Category$ & $Definition$ &   $Speech$ $Examples$  & $Gesture$ $Examples$ \\
	\hline Movement & Any indication of movement & ``Just slide them together and then it
	looks like that'' (S.M) & Miming movement with the hands (G.M) \\ \hline 
	Perceptual Features & Focus on a particular feature of the model & ``Because
	there is a little bend in here and a point thing here'' (S.PF) & Pointing to a
	specific feature on the model (G.PF) \\ \hline 
	Perceptual Whole & Any indication of seeing the model as a whole & ``It looks
	like an arrow!'' (S.PW) & Gesture indicating inclusion of the whole shape 
	(G.PW) \\
	\hline Vague & An expression of strategy that the coder cannot decipher & ``Because I
	looked at that and I looked at the differences'' (S.V) & Waving gestures above
	the computer device that do not indicate any specific strategy (G.V) \\ \hline 
	Other & Any strategy not listed above & ``And here is like half of it.
	But so and two halves make a whole'' (S.O) & Using the hand to form a straight
	line through the middle of the whole shape to represent the line of symmetry
	(G.O)
	\\
	\hline
	\end{tabular}
   \\ \rule{0mm}{5mm}
\end{center}
\label{GcodingStrategy}
\end{table}


\emph{Excerpt 1}

\begin{quotation}
``It goes upward, downward, over, up,across, over downward, over, and doesn't
go up again."
\end{quotation}
Gesture: (Pointing along sides of the shape)\\
Codes: S.M, G.PF \\ 
- User3, Shape P2, Session 1\\


\emph{Excerpt 2}

\begin{quotation}
``I saw that is was a series of L shapes\ldots I just thought that if I
could make a series of L, I could soon make it." 
\end{quotation} 
Gesture: (rotating model in hand repeatedly)\\
Codes: S.PW, S.PF, G.V\\ 
- User 4, Shape P2, Session 1\\

\emph{Excerpt 3}

\begin{quotation}
``Well it's based on the points, so the shape has 6 points, so needed 5
towers for the 6 points\ldots because of how to works you get one point there, you go
down, then another point, then you go over, on the same kind of layer\ldots and
then in the middle, so you don't just have a flat square, and you're pulling it up, you get
two points, so it is pulling the whole thing."
\end{quotation} 
Gesture: (pulls ``up'' with hands, points to towers, makes flat hand for
layers)\\
Codes: S.PW, S.PF, S.M, S.O, G.PF, G.O, G.M\\ 
- User 11, Shape CH1, Session 1\\

\emph{Excerpt 4}

\begin{quotation}
``Because the lines are the same way, as, the, uh, that shape."
\end{quotation}
Gesture: (shrugs)\\
Codes: S.V, G.V\\ 
- User 8, Shape P2, Session 1\\

\emph{Excerpt 5}

\begin{quotation}
``By taking each point, each side, and creating it into a point on the
graph" "on this shape, uh, this is about, uh 3 points wide, and 2 points tall, like, on
this, and then also, like 2 points out."
\end{quotation}
Gesture: (at `like this' points to two lights on the device one after the
other.)\\
Codes: S.O, S.PF, G.PF \\
- User 7, Shape T2, Session 1\\

\emph{Excerpt 6}

\begin{quotation}
``If it was hull then I knew it would automatically connect and fill in between
the points, so all I really needed to do was make two triangles, and it would
fill in the rest" 
\end{quotation}
Gesture: (gestures with hands to indicate two parts moving together)\\
Codes: S.O, S.PF, G.M, G.PF\\
- User 11, Shape CH1, Session 2\\

\emph{Excerpt 7}

\begin{quotation}
``Because, it didn't look like a pyramid, but like a right pyramid\ldots like a
right angle, one side is a right\ldots I put four there, in each corner\ldots I
put one on top because it was the point, a vertex\ldots and then I looked at the laptop and it
looked like the shape." 
\end{quotation}
Gesture: (points to different corners of the shape) \\
Codes: S.PW, S.PF, S.O, G.PF, G.V\\
- User 17, Shape CH4, Session 2\\

\emph{Excerpt 8}

\begin{quotation}
``When I was trying to make this part, I was trying to make it small, from the
middle\ldots I made it look like it's supposed to look, I made it down,
straight, up, right, down, straight, up\ldots so I try to just mimic the shape
and see how it works."
\end{quotation}
Gesture: (gestures along the path on the device that the user is describing)\\
Codes: S.PF, S.PW, S.M, G.M, G.PF\\
- User 1, Shape T3, Session 2\\

\emph{Excerpt 9}

\begin{quotation}
``There's two U's, and um, like an entrance to a doorway."
\end{quotation}
Gesture: (traces with finger along different parts of the shape)\\
Codes: S.PF, S.O, G.PF\\
- User 5, Shape T3, Session 1\\

\emph{Excerpt 10}

\begin{quotation}
``Basically drawing it out with my mind." 
\end{quotation}
Gesture: (makes drawing motion with hand, then a sort of vague waving motion)\\
Codes:  S.O, S.V, G.M, G.V\\
- User 2, Shape P3, Session 2\\





\end{document}

