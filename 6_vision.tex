\chapter{Vision}
\label{vision}


Given the different medium of the pop-up book (paper as opposed to circuit
boards), it is worth exploring the possibilities afforded by a cheaper, more
flexible material. For instance, the flexibility of paper might provide the
means for new types of modeling actions. It is plausible to imagine paper tabs
or other mechanisms that perturb the LEDs off the integer lattice, or alter the
overall topology in such a way that new shapes are possible (e.g. by deforming
an equidistant grid into a spherical shape). There may be additional sensors or
hardware that could be embedded into the book to provide new functionality
(rotation, proximity, pressure). Additionally, due the inexpensive and portable
nature of the pop-up book, it is worth exploring the sorts of interactions that
could occur between several pop-up books (e.g., extending the input field to
include two or more grids, networked interactions like cooperative modeling
tasks, or competitive games like 3D-battleship). By using paper as a material to
think with, we may find further possibilities as development continues.